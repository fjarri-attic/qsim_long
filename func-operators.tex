% =============================================================================
\section{Field operators and restricted basis}
% =============================================================================

Multimode fields are described by operators $\Psiop_j^{\dagger}(\xvec)$ and $\Psiop_j(\xvec)$, where $\Psiop_j^{\dagger}(\xvec)$ creates a bosonic atom of spin $j$ at location $\xvec$, and $\Psiop_j(\xvec)$ destroys one; the commutators are
\begin{eqn}
\label{eqn:func-aux:commutators}
	[ \Psiop_j(\xvec), \Psiop_k^{\dagger}(\xvec^\prime) ]
	= \delta_{jk} \delta(\xvec^\prime-\xvec).
\end{eqn}
Field operators can be decomposed using a single-particle basis:
\begin{eqn}
	\Psiop_j(\xvec) = \sum_{\nvec} \phi_{\nvec}(\xvec) \hat{a}_{j,\nvec}.
\end{eqn}
Single mode operators $\hat{a}_{j,\nvec}$ obey bosonic commutation relations, the pair $j,\nvec$ serving as a mode identifier.

Projection transformation can be extended to work on operators.
The expression remains the same, and the type becomes
\begin{eqn}
	\hat{\mathcal{P}} :: \mathbb{FH} \rightarrow \mathbb{FH}_L,
\end{eqn}
where $\mathbb{FH}_L \equiv (\mathbb{R}^D \rightarrow \mathbb{H}_L)$, and $\mathbb{H}_L$ is the Hilbert space of the restricted subset of modes.
Being applied to the annihilation operator $\Psiop_j$, this transformation returns the restricted annihilation operator
\begin{eqn}
	\hat{\mathcal{P}} [\Psiop_j]
	= \sum_{\nvec \in L} \phi_{\nvec} (\xvec) \hat{a}_{j,\nvec}
	= \Psiop_{jL} (\xvec),
\end{eqn}
containing only modes from subset $L$.
Same as with functions, we will consider all field operators to be restricted and omit the index $L$.

Because of the restricted nature of the operator, commutation relations~\eqnref{func-aux:commutators} no longer apply.
The following ones should be used instead:
\begin{eqn}
\label{eqn:func-aux:restricted-commutators}
	\left[ \Psiop_j(\xvec), \Psiop_k(\xvec^\prime) \right]
	& = \left[ \Psiop_j^\dagger(\xvec), \Psiop_k^\dagger(\xvec^\prime) \right] = 0, \\
	\left[ \Psiop_j(\xvec), \Psiop_k^\dagger(\xvec^\prime) \right]
	& = \delta_{jk} \delta_L(\xvec^\prime, \xvec).
\end{eqn}

Let us now find the expression for high-order commutators of restricted field operators, analogous to the similar one for single-mode operators~\cite{Louisell1990}.

\begin{lemma}
	Abbreviating $\Psiop \equiv \Psiop(\xvec)$ and $\Psiop^\prime \equiv \Psiop(\xvec^\prime)$:
	\begin{eqn*}
		\left[ \Psiop, ( \Psiop^{\prime\dagger} )^l \right]
		& = l \delta_P (\xvec^\prime - \xvec) ( \Psiop^{\prime\dagger} )^{l-1}, \\
		\left[ \Psiop^\dagger, ( \Psiop^\prime )^l \right]
		& = - l \delta_P^* (\xvec^\prime - \xvec) ( \Psiop^\prime )^{l-1}.
	\end{eqn*}
\end{lemma}
\begin{proof}
Proved by induction.
\end{proof}

A further generalisation of these relations is

\begin{lemma}
\label{lmm:functional-commutators}
	\begin{eqn*}
		\left[ \Psiop, f( \Psiop^\prime, \Psiop^{\prime\dagger} ) \right]
		& = \delta_P (\xvec^\prime - \xvec) \frac{\partial f}{\partial \Psiop^{\prime\dagger}} \\
		\left[ \Psiop^\dagger, f( \Psiop^\prime, \Psiop^{\prime\dagger} ) \right]
		& = -\delta_P^* (\xvec^\prime - \xvec) \frac{\partial f}{\partial \Psiop^\prime},
	\end{eqn*}
	where $f(z, z^*)$ is a function that can be expanded in the power series of $z$ and $z^*$.
\end{lemma}
