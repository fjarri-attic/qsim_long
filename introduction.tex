% =============================================================================
\section{Introduction}
% =============================================================================

\todo{Introductory words?}

Wigner representation~\cite{Gardiner2004} is a convenient and effective method of simulating the dynamics of Bose-Einstein condensates (BECs).
It works best in the limit of large particle number, where direct diagonalization approaches like~\cite{Sakmann2009} become computationally impossible.
Large particle number usually implies large amount of field modes with significant population, which makes two-mode variational approaches~\cite{Li2008,Li2009,Sinatra2011} less accurate.

Phase-space treatment of multimode problems can be simplified by working with multimode field operators instead of single-mode operators.
This approach was initially introduced by Graham~\cite{Graham1970,Graham1970a}.
Later it was used in a number of works~\cite{Steel1998,Isella2006,Norrie2006a} without formally defining and generalizing corresponding transformation, or accompanying theorems.
In order to numerically calculate the evolution of the Wigner function of a system, one has to employ Wigner truncation~\cite{Drummond1993,Steel1998,Sinatra2002}, which further complicates the formal description of the method.

In this paper we present a formal description of the application of truncated Wigner representation to simulating the multi-mode dynamics of Bose-Einstein condensates (BECs).
We successively reduce the problem in its initial form, the master equation for bosonic field operators, to the system of stochastic differential equations, which have significantly lower computational complexity (at a price of making several approximations).

\todo{Possibly worth mentioning:}
Positive-P and Wigner application:~\cite{Deuar2007}.
Multimode Wigner using decomposition into modes (no losses):~\cite{Norrie2005,Norrie2006}.
Functional Wigner, finite-temperature:~\cite{Steel1998,Isella2006}.
Applications of method from this paper: Ramsey interferometry simulations~\cite{Egorov2011}, short letter on Ramsey and squeezing~\cite{Opanchuk2012}, 4-mode truncated Wigner and losses in entanglement calculations~\cite{Opanchuk2012a}.
