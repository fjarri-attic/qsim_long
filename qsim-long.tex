\documentclass[12pt,aip,jmp,amssymb,amsmath]{revtex4-1}
\usepackage{amsthm}
\usepackage{dsfont}


\usepackage{color}

\newtheorem{theorem}{Theorem}
\newtheorem{definition}{Definition}
\newtheorem{lemma}{Lemma}

\begin{document}
\title{Wigner representation of BEC}

\author{B. Opanchuk}
\email{bogdan@opanchuk.net}
\affiliation{Centre for Atom Optics and Ultrafast Spectroscopy, Swinburne University of Technology, Hawthorn, VIC 3122, Australia}
\author{P. D. Drummond}
\affiliation{Centre for Atom Optics and Ultrafast Spectroscopy, Swinburne University of Technology, Hawthorn, VIC 3122, Australia}

\date{\today}
\begin{abstract}
We develop a method of simulating full quantum dynamics of multi-mode multi-component Bose-Einstein condensate in a trap.
We use truncated Wigner representation to produce stochastic equations that can be solved using conventional methods.
Our approach describes spatial evolution of spinor components and properly accounts for nonlinear losses.
\end{abstract}

% Uncomment to set PACs
%\pacs{}

% uncomment if the separate page for title is needed
\maketitle



% =============================================================================
\section{Introduction}
% =============================================================================

[TODO: Introductory words?]

Wigner representation~\cite{Gardiner2004} is a convenient and effective method of simulating the dynamics of Bose-Einstein condensates (BECs).
It works best in the limit of large particle number, where direct diagonalization approaches like~\cite{Sakmann2009} become computationally impossible.
Large particle number usually implies large amount of field modes with significant population, which makes two-mode variational approaches~\cite{Li2008,Li2009,Sinatra2011} less accurate.

Phase-space treatment of multimode problems can be simplified by working with multimode field operators instead of single-mode operators.
This approach was initially introduced by Graham~\cite{Graham1970,Graham1970a}.
Later it was used in a number of works~\cite{Steel1998,Isella2006,Norrie2006a} without formally defining and generalizing corresponding transformation, or accompanying theorems.
In order to numerically calculate the evolution of the Wigner function of a system, one has to employ Wigner truncation~\cite{Drummond1993,Steel1998,Sinatra2002}, which further complicates the formal description of the method.

In this paper we present a formal description of the application of truncated Wigner representation to simulating the multi-mode dynamics of Bose-Einstein condensates (BECs).
We successively reduce the problem in its initial form, the master equation for bosonic field operators, to the system of stochastic differential equations, which have significantly lower computational complexity (at a price of making several approximations).

[TODO: Possibly worth mentioning:]
Positive-P and Wigner application:~\cite{Deuar2007}.
Multimode Wigner using decomposition into modes (no losses):~\cite{Norrie2005,Norrie2006}.
Functional Wigner, finite-temperature:~\cite{Steel1998,Isella2006}.
Applications of method from this paper: Ramsey interferometry simulations~\cite{Egorov2011}, short letter on Ramsey and squeezing~\cite{Opanchuk2012}, 4-mode truncated Wigner and losses in entanglement calculations~\cite{Opanchuk2012a}.


% =============================================================================
\section{Master equation}
% =============================================================================

In this paper we consider a zero-temperature $C$-component BEC in $D$ effective dimensions.
The Hamiltonian for this system is expressed in terms of bosonic field creation and annihilation operators $\hat{\Psi}_j^{\dagger}(\boldsymbol{x})$ and $\hat{\Psi}_j(\boldsymbol{x})$, $j = 1 \ldots C$, which obey standard bosonic commutation relation
\begin{eqnarray}\label{eqn:master-eqn:commutators}
    [ \hat{\Psi}_j, \hat{\Psi}_k^{\prime\dagger} ]
    = \delta_{jk} \delta(\boldsymbol{x}^\prime-\boldsymbol{x}),
\end{eqnarray}
where $\boldsymbol{x} \in \mathbb{R}^D$ is a $D$-dimensional coordinate vector, and we dubbed $\hat{\Psi}_j \equiv \hat{\Psi}_j(\boldsymbol{x})$ and $\hat{\Psi}_k^\prime \equiv \hat{\Psi}_k(\boldsymbol{x}^\prime)$ for brevity (hereinafter the same abbreviation will be used for all functions of coordinates).
The second-quantized Hamiltonian for the system is
\begin{eqnarray}\label{eqn:master-eqn:hamiltonian}
    \hat{H}
    = \int d\boldsymbol{x} \left\{
        \hat{\Psi}_j^{\dagger} K_{jk} \hat{\Psi}_k
        + \frac{1}{2} \int d\boldsymbol{x}^\prime
            \hat{\Psi}_j^\dagger \hat{\Psi}_k^{\prime\dagger}
            U_{jk}(\boldsymbol{x} - \boldsymbol{x}^\prime)
            \hat{\Psi}_j^\prime \hat{\Psi}_k
    \right\},
\end{eqnarray}
where $U_{jk}$ is the two-body scattering potential, and the single-particle Hamiltonian $K_{jk}$ is
\begin{eqnarray}\label{eqn:master-eqn:single-particle}
    K_{jk} = \left(
            -\frac{\hbar^2}{2m} \nabla^2 + \hbar \omega_j + V_j(\boldsymbol{x})
        \right) \delta_{jk}
        + \hbar \Omega_{jk}(t),
\end{eqnarray}
where $m$ is the atomic mass, $V_j$ is the external trapping potential for spin $j$, $\hbar \omega_j$ is the internal energy of spin $j$, and $\Omega_{jk}$ represents a time-dependent coupling that is used to rotate one spin projection into another.

If we impose an energy cutoff $\epsilon_{\mathrm{cut}}$ and only take into account low-energy modes, the non-local scattering potential $U_{jk}(\boldsymbol{x} - \boldsymbol{x}^\prime)$ can be replaced by the contact potential $U_{jk} \delta(\boldsymbol{x} - \boldsymbol{x}^\prime)$~\cite{Morgan2000}, giving the effective Hamiltonian
\begin{eqnarray}\label{eqn:master-eqn:effective-H}
    \hat{H}
    = \int d\boldsymbol{x} \left\{
        \hat{\Psi}_j^{\dagger} K_{jk} \hat{\Psi}_k
        + \frac{U_{jk}}{2} \hat{\Psi}_j^\dagger \hat{\Psi}_k^\dagger \hat{\Psi}_j \hat{\Psi}_k
    \right\},
\end{eqnarray}
where $\hat{\Psi}_j^{\dagger}$ and $\hat{\Psi}_j$ are field operators in the new restricted basis of low-energy modes, which is described in detail in the next section.
For the $s$-wave scattering in three dimensions the coefficient is $U_{jk} = 4 \pi \hbar^2 a_{jk} / m$, where $a_{jk}$ is the scattering length.
Note that in general case the coefficient must be renormalised depending on the grid~\cite{Sinatra2002}, but the change is small if $dx \gg a_{jk}$.

The Markovian master equation for the system with the inclusion of losses can be written as~\cite{Jack2002}
\begin{eqnarray}\label{eqn:master-eqn:master-eqn}
    \frac{d\hat{\rho}}{dt} =
        - \frac{i}{\hbar} \left[ \hat{H}, \hat{\rho} \right]
        + \sum_{\boldsymbol{l}} \kappa_{\boldsymbol{l}} \int d\boldsymbol{x}
            \mathcal{L}_{\boldsymbol{l}} \left[ \hat{\rho} \right],
\end{eqnarray}
where $\boldsymbol{l} = (l_1, l_2, \ldots, l_C)$ is a tuple indicating the number of atoms from each component involved in the inelastic interaction, $C$ being the total number of components, and we have introduced local Liouville loss terms,
\begin{eqnarray}\label{eqn:master-eqn:loss-term}
    \mathcal{L}_{\boldsymbol{l}} \left[ \hat{\rho} \right] =
        2\hat{O}_{\boldsymbol{l}} \hat{\rho} \hat{O}_{\boldsymbol{l}}^\dagger
        - \hat{O}_{\boldsymbol{l}}^\dagger \hat{O}_{\boldsymbol{l}} \hat{\rho}
        - \hat{\rho} \hat{O}_{\boldsymbol{l}}^\dagger \hat{O}_{\boldsymbol{l}}.
\end{eqnarray}
The reservoir coupling operators $\hat{O}_{\boldsymbol{l}}$ are products of local field annihilation operators:
\begin{eqnarray}
    \hat{O}_{\boldsymbol{l}}
    \equiv \hat{O}_{\boldsymbol{l}} (\hat{\boldsymbol{\Psi}})
    = \prod_{c=1}^C \hat{\Psi}_c^{l_c} (\boldsymbol{x}),
\end{eqnarray}
describing local $\left( \sum_{c=1}^C l_c \right)$-body collision losses.



% =============================================================================
\section{Field operators and restricted basis}
% =============================================================================
\label{sec:func-operators}

For each component $j$ we define an orthonormal basis consisting of $\phi_{j,\boldsymbol{n}}(\boldsymbol{x})$, where $\boldsymbol{n} \in \mathbb{B}_j$ is a mode identifier.
Orthonormality and completeness conditions for basis functions are, respectively,
\begin{eqnarray}
    \int\limits_A \phi_{j,\boldsymbol{n}}^* \phi_{j,\boldsymbol{m}} d\boldsymbol{x} = \delta_{\boldsymbol{n}\boldsymbol{m}}, \\
    \sum_{\boldsymbol{n}} \phi_{j,\boldsymbol{n}}^* \phi_{j,\boldsymbol{n}}^\prime = \delta(\boldsymbol{x}^\prime - \boldsymbol{x}),
\end{eqnarray}
where the exact nature of integration area $A$ depends on the basis set (for example, $A$ is the whole space for harmonic oscillator modes, or a box for plane waves).
Hereinafter we assume that the integration $\int d\boldsymbol{x}$ is always performed over $A$.

Standard bosonic field operators from~((\ref{eqn:master-eqn:commutators})) which project coordinates to the full Hilbert space $\mathbb{H}$ can be decomposed as
\begin{eqnarray}
    \hat{\Psi}_j \in (\mathbb{R}^D \rightarrow \mathbb{H}): \qquad
    \hat{\Psi}_j = \sum_{\boldsymbol{n} \in \mathbb{B}_j} \phi_{j,\boldsymbol{n}} \hat{a}_{j,\boldsymbol{n}},
\end{eqnarray}
where single mode operators $\hat{a}_{j,\boldsymbol{n}}$ obey bosonic commutation relations, the pair $j,\boldsymbol{n}$ serving as a mode identifier.
The energy cutoff mentioned in the previous section will result in operating with some fixed subset of each component's basis.
Let $\mathbb{M}_j \subseteq \mathbb{B}_j$ be this subset with cardinality $|\mathbb{M}|$.
Restricted field operators contain only modes from the subset $\mathbb{M}_j$:
\begin{eqnarray}
    \hat{\Psi}_j \in (\mathbb{R}^D \rightarrow \mathbb{H}_{\mathbb{M}_j}): \qquad
    \hat{\Psi}_j = \sum_{\boldsymbol{n} \in \mathbb{M}_j} \phi_{j,\boldsymbol{n}} \hat{a}_{j,\boldsymbol{n}}.
\end{eqnarray}
In this paper we will use them without explicit indices in order to avoid clutter [TODO: Should we explicitly mark ``full basis'' operators in the previous section?].
Restricted field operators have the type $\mathbb{FH}_{\mathbb{M}_j} \equiv (\mathbb{R}^D \rightarrow \mathbb{H}_{\mathbb{M}_j})$, where $\mathbb{H}_{\mathbb{M}_j}$ is the Hilbert space of the restricted subset of modes.

Because of the restricted nature of the operator, commutation relations~((\ref{eqn:master-eqn:commutators})) no longer apply.
The following ones should be used instead:
\begin{eqnarray}\label{eqn:func-operators:restricted-commutators}
    \left[ \hat{\Psi}_j, \hat{\Psi}_k^\prime \right]
    & = \left[ \hat{\Psi}_j^\dagger, \hat{\Psi}_k^{\prime\dagger} \right] = 0, \\
    \left[ \hat{\Psi}_j, \hat{\Psi}_k^{\prime\dagger} \right]
    & = \delta_{jk} \delta_{\mathbb{M}_j}(\boldsymbol{x}^\prime, \boldsymbol{x}),
\end{eqnarray}
where $\delta_{\mathbb{M}_j}$ is the restricted delta-function from Definition~\ref{def:func-calculus:restricted-delta}.



% =============================================================================
\section{Functional Wigner representation}
% =============================================================================

The single-mode Wigner transformation of the operator $\hat{A}$ is defined as
\begin{eqnarray}
    \mathcal{W}_{\mathrm{sm}}[\hat{A}]
    = \frac{1}{\pi^2} \int d^2 \lambda \exp(-\lambda \alpha^* + \lambda^* \alpha)
        \mathrm{Tr} \left\{ \hat{A} \hat{D}(\lambda, \lambda^*) \right\},
\end{eqnarray}
where the displacement operator $\hat{D}(\lambda, \lambda^*) = \exp(\lambda \hat{a}^\dagger - \lambda^* \hat{a})$ was first introduced by Weyl~\cite{Weyl1950}.
The detailed description of the Wigner function $W(\alpha, \alpha^*) \equiv \mathcal{W}_{\mathrm{sm}}[\hat{\rho}]$ can be found in~\cite{Gardiner2004}.
In this section we will extend this definition to the multimode case.

The important part of the definition is the functional analogue of the displacement operator.
[TODO: Need to explain why we write $F[\Lambda, \Lambda^*]$ and not $F[\Lambda]$?]

\begin{definition}
    Functional displacement operator
    \begin{eqnarray*}
        & \hat{D}_j :: \mathbb{F}_{\mathbb{M}_j} \rightarrow \mathbb{H}_{\mathbb{M}_j} \\
        & \hat{D}_j[\Lambda, \Lambda^*] = \exp \int d\boldsymbol{x} \left(
            \Lambda \hat{\Psi}_j^\dagger - \Lambda^* \hat{\Psi}_j
        \right).
    \end{eqnarray*}
    It is also convenient to define the displacement functional as
    \begin{eqnarray*}
        & D :: \mathbb{F}_{\mathbb{M}_j} \rightarrow \mathbb{F}_{\mathbb{M}_j} \rightarrow \mathbb{C} \\
        & D[\Lambda, \Lambda^*, \Psi, \Psi^*] = \exp \int d\boldsymbol{x} \left(
            -\Lambda \Psi^* + \Lambda^* \Psi
        \right).
    \end{eqnarray*}
\end{definition}

It can be shown that the functional displacement operator has properties similar to its single-mode equivalent.

\begin{lemma}
\label{lmm:func-wigner:displacement-derivatives}
    \begin{eqnarray*}
        \frac{\delta}{\delta \Lambda^\prime} \hat{D}_j[\Lambda, \Lambda^*]
        = \hat{D}_j[\Lambda, \Lambda^*] (\hat{\Psi}_j^{\prime\dagger} + \frac{1}{2} \Lambda^{\prime*})
        = (\hat{\Psi}_j^{\prime\dagger} - \frac{1}{2} \Lambda^{\prime*}) \hat{D}_j[\Lambda, \Lambda^*], \\
        -\frac{\delta}{\delta \Lambda^{\prime*}} \hat{D}_j[\Lambda, \Lambda^*]
        = \hat{D}_j(\Lambda, \Lambda^*) (\hat{\Psi}_j^\prime + \frac{1}{2} \Lambda^\prime)
        = (\hat{\Psi}_j^\prime - \frac{1}{2} \Lambda^\prime) \hat{D}_j[\Lambda, \Lambda^*].
    \end{eqnarray*}
\end{lemma}
\begin{proof}
Proved using Baker-Hausdorff theorem and evaluating integrals.
\end{proof}

It is convenient to first define a general Wigner transformation.

\begin{definition}
\label{def:func-wigner:w-transformation}
    Multi-component functional Wigner transformation $\mathcal{W}$ is defined as
    \begin{eqnarray*}
    & &   \mathcal{W} :: \left( \mathbb{R}^D \rightarrow \prod_{j=1}^C \mathbb{H}_{\mathbb{M}_j} \right)
            \rightarrow \prod_{j=1}^C \mathbb{F}_{\mathbb{M}_j}
            \rightarrow \mathbb{C} \\
    & &   \mathcal{W}[\hat{A}]
        = \frac{1}{\pi^{2 \sum|\mathbb{M}_j|}} \int \delta^2 \boldsymbol{\Lambda}
            \left( \prod_{j=1}^C D[\Lambda_j, \Lambda_j^*, \Psi_j, \Psi_j^*] \right)
            \mathrm{Tr} \left\{ \hat{A} \prod_{j=1}^C \hat{D}_j[\Lambda_j, \Lambda_j^*] \right\},
    \end{eqnarray*}
    where $\Lambda_j \in \mathbb{F}_{\mathbb{M}_j}$, and $\int \delta^2 \boldsymbol{\Lambda} \equiv \int \delta^2 \Lambda_1 \ldots \delta^2 \Lambda_C$.
    It transforms a coordinate-dependent operator $\hat{A}$ on a restricted subset of a Hilbert space to a functional $(\mathcal{W}[\hat{A}])[\boldsymbol{\Psi}, \boldsymbol{\Psi}^*]$.
    [TODO: Add expression for Weyl transformation?]
\end{definition}

Wigner functional is a special case of Wigner transformation.

\begin{definition}
\label{def:func-wigner:w-functional}
    The Wigner functional is
    \begin{eqnarray*}
    &    W :: \prod_{j=1}^C \mathbb{F}_{\mathbb{M}_j} \rightarrow \mathbb{R} \\
    &    W [\boldsymbol{\Psi}, \boldsymbol{\Psi}^*]
        \equiv \mathcal{W}[\hat{\rho}]
        = \frac{1}{\pi^{2 \sum|\mathbb{M}_j|}} \int \delta^2 \boldsymbol{\Lambda}
            \left( \prod_{j=1}^C D[\Lambda_j, \Lambda_j^*, \Psi_j, \Psi_j^*] \right) \chi_W,
    \end{eqnarray*}
    where $\chi_W [\boldsymbol{\Lambda}, \boldsymbol{\Lambda}^*]$ is the characteristic functional
    \begin{eqnarray*}
        \chi_W [\boldsymbol{\Lambda}, \boldsymbol{\Lambda}^*]
        = \mathrm{Tr} \left\{ \hat{\rho} \prod_{j=1}^C \hat{D}_j[\Lambda_j, \Lambda_j^*] \right\}.
    \end{eqnarray*}
    [TODO: Need proof that $W$ is real-valued?
    Or just an explicit statement that $\mathcal{W}[\hat{A}]$ is real when $\hat{A}$ is Hermitian.]
\end{definition}

The Wigner functional has two important properties analogous to the single-mode case.
First one is used to successively transform operator products.

\begin{theorem}[Functional correspondences]
\label{thm:func-wigner:correspondences}
    If $\mathcal{W} [ \hat{A} ] \equiv (\mathcal{W} [ \hat{A} ]) [\boldsymbol{\Psi}, \boldsymbol{\Psi}^*]$, then
    \begin{eqnarray*}
        \mathcal{W} [ \hat{\Psi}_j \hat{A} ]
            & = \left( \Psi_j + \frac{1}{2} \frac{\delta}{\delta \Psi_j^*} \right) \mathcal{W}[\hat{A}],
        \qquad
        \mathcal{W} [ \hat{\Psi}_j^\dagger \hat{A} ]
            = \left( \Psi_j^* - \frac{1}{2} \frac{\delta}{\delta \Psi_j} \right) \mathcal{W}[\hat{A}], \\
        \mathcal{W} [ \hat{A} \hat{\Psi}_j ]
            & = \left( \Psi_j - \frac{1}{2} \frac{\delta}{\delta \Psi_j^*} \right) \mathcal{W}[\hat{A}],
        \qquad
        \mathcal{W} [ \hat{A} \hat{\Psi}_j^\dagger ]
            = \left( \Psi_j^* + \frac{1}{2} \frac{\delta}{\partial \Psi_j} \right) \mathcal{W}[\hat{A}].
    \end{eqnarray*}
\end{theorem}
\begin{proof}
The proof uses Lemma~\ref{lmm:func-wigner:displacement-derivatives} to transform the $\hat{A} \prod_j \hat{D}_j$ product inside the trace, and Lemma~\ref{lmm:func-calculus:zero-integrals} to integrate by parts, effectively moving the differentials to intended places.
\end{proof}

The second property complements the first one, providing the way to obtain expectations of operator products given the Wigner function.
Again, it requires a supplementary lemma.

\begin{lemma}
\label{lmm:func-wigner:moments-from-chi}
    For any non-negative integer $r$ and $s$:
    \begin{eqnarray*}
        \langle \left\{ (\hat{\Psi}_j^\prime)^r (\hat{\Psi}_j^{\prime\dagger})^s \right\}_{\mathrm{sym}} \rangle
        = \left.
            \left( \frac{\delta}{\delta \Lambda_j^\prime} \right)^s
            \left( -\frac{\delta}{\delta \Lambda_j^{\prime*}} \right)^r
            \chi_W [\boldsymbol{\Lambda}, \boldsymbol{\Lambda}^*]
        \right|_{\boldsymbol{\Lambda} \equiv 0}.
    \end{eqnarray*}
\end{lemma}
\begin{proof}
The factor corresponding to $j$-th component in the displacement operator can be expanded as
\begin{eqnarray}
    \exp (\Lambda_j \hat{\Psi}_j^\dagger - \Lambda_j^* \hat{\Psi}_j)
    = \sum_{r,s}
        \frac{
            \left\{
                \left( \int d\boldsymbol{x} \Lambda_j \hat{\Psi}_j^\dagger \right)^r
                \left( -\int d\boldsymbol{x} \Lambda_j^* \hat{\Psi}_j \right)^s
            \right\}_{\mathrm{sym}}
        }
        {r!s!}.
\end{eqnarray}
We can swap functional derivative with both integration and multiplication by independent function, so:
\begin{eqnarray}
    \frac{\delta}{\delta \Lambda_j^\prime} \left( \int d\boldsymbol{x} \Lambda_j \hat{\Psi}_j^\dagger \right)^r
    = r \hat{\Psi}_j^{\prime\dagger} \left( \int d\boldsymbol{x} \Lambda_j \hat{\Psi}_j^\dagger \right)^{r-1},
\end{eqnarray}
and multiple application of the differential gives us
\begin{eqnarray}
    \left( \frac{\delta}{\delta \Lambda_j^\prime} \right)^r
    \left( \int d\boldsymbol{x} \Lambda_j \hat{\Psi}_j^\dagger \right)^r
    = r! ( \hat{\Psi}_j^{\prime\dagger} )^r.
\end{eqnarray}
Similarly for the other differential:
\begin{eqnarray}
    \left( -\frac{\delta}{\delta \Lambda_j^{\prime*}} \right)^s
    \left( -\int d\boldsymbol{x} \Lambda_j \hat{\Psi}_j^\dagger \right)^s
    = s! ( \hat{\Psi}_j^{\prime\dagger} )^s.
\end{eqnarray}

Thus, differentiation will eliminate all lower order terms in the expansion, and all higher order terms will be eliminated by setting $\Lambda_j \equiv 0$ for every $j$, leaving only one operator product with required order.
\end{proof}

\begin{theorem}[Calculation of expectations]
\label{thm:func-wigner:moments}
    For any non-negative integer $r_j$, $s_j$
    \begin{eqnarray*}
        \langle \left\{ \prod_{j=1}^C \hat{\Psi}_j^{r_j} (\hat{\Psi}_j^\dagger)^{s_j} \right\}_{\mathrm{sym}} \rangle
        = \int \delta^2 \boldsymbol{\Psi}
            \left( \prod_{j=1}^C \Psi_j^{r_j} (\Psi_j^*)^{s_j} \right) W[\boldsymbol{\Psi}, \boldsymbol{\Psi}^*].
    \end{eqnarray*}
\end{theorem}
\begin{proof}
By definition of the Wigner functional:
\begin{eqnarray}
    & \int \delta^2\boldsymbol{\Psi} \left( \prod_{j=1}^C \Psi_j^{r_j} (\Psi_j^*)^{s_j} \right) W[\boldsymbol{\Psi}, \boldsymbol{\Psi}^*] \\
    & = \frac{1}{\pi^{2\sum|\mathbb{M}_j|}} \mathrm{Tr} \left\{ \hat{\rho}
        \prod_{j=1}^C
            \int \delta^2 \Lambda_j \left(
                \int \delta^2 \Psi_j\, \Psi_j^{r_j} (\Psi_j^*)^{s_j}
                D[\Lambda_j, \Lambda_j^*, \Psi_j, \Psi_j^*]
            \right)
            \hat{D}_j[\Lambda_j, \Lambda_j^*]
    \right\}.
\end{eqnarray}
Evaluating the integral over $\Psi_j$ using Lemma~\ref{lmm:func-calculus:fourier-of-moments}:
\begin{eqnarray}
    = \mathrm{Tr} \left\{ \hat{\rho}
        \prod_{j=1}^C
            \int \delta^2 \Lambda_j \left(
                \left( -\frac{\delta}{\delta \Lambda_j^*} \right)^{r_j}
                \left( \frac{\delta}{\delta \Lambda_j} \right)^{s_j}
                \Delta_{\mathbb{M}_j}[\Lambda_j]
            \right)
            \hat{D}_j[\Lambda_j, \Lambda_j^*]
    \right\}.
\end{eqnarray}
Integrating by parts for each component in turn and eliminating terms which fit Lemma~\ref{lmm:func-calculus:zero-delta-integrals}:
\begin{eqnarray}
    & = \mathrm{Tr} \left\{ \hat{\rho}
        \prod_{j=1}^C \int \delta^2 \Lambda_j
            \Delta_{\mathbb{M}_j}[\Lambda_j]
            \left( -\frac{\delta}{\delta \Lambda_j^*} \right)^{r_j}
            \left( \frac{\delta}{\delta \Lambda_j} \right)^{s_j}
            \hat{D}_j[\Lambda_j, \Lambda_j^*]
    \right\} \\
    & = \left.
        \left(
            \prod_{j=1}^C
            \left( \frac{\delta}{\delta \Lambda_j} \right)^{s_j}
            \left( -\frac{\delta}{\delta \Lambda_j^*} \right)^{r_j}
        \right)
        \chi_W [\boldsymbol{\Lambda}, \boldsymbol{\Lambda}^*]
    \right|_{\boldsymbol{\Lambda} \equiv 0},
\end{eqnarray}
where $\Delta_{\mathbb{M}_j}[\Lambda_j]$ is a delta functional from Definition~\ref{def:func-calculus:delta-functional}.
Now, recognising the final expression as a part of Lemma~\ref{lmm:func-wigner:moments-from-chi},
we immediately get the statement of the theorem.
\end{proof}



% =============================================================================
\section{Specific cases of transformations}
% =============================================================================

In order to Wigner transform the master equation~((\ref{eqn:master-eqn:master-eqn})), we will need several theorems about transformations of specific operator sequences.

First, we will need an expression for high-order commutators of restricted field operators.
They look somewhat similar to those for single-mode bosonic operators, or standard field operators from~\cite{Louisell1990}.

\begin{lemma}
    \begin{eqnarray*}
        \left[ \hat{\Psi}, ( \hat{\Psi}^{\prime\dagger} )^l \right]
        & = l \delta_{\mathbb{M}} (\boldsymbol{x}^\prime - \boldsymbol{x}) ( \hat{\Psi}^{\prime\dagger} )^{l-1}, \\
        \left[ \hat{\Psi}^\dagger, ( \hat{\Psi}^\prime )^l \right]
        & = - l \delta_{\mathbb{M}}^* (\boldsymbol{x}^\prime - \boldsymbol{x}) ( \hat{\Psi}^\prime )^{l-1}.
    \end{eqnarray*}
\end{lemma}
\begin{proof}
Proved by induction.
\end{proof}

A further generalisation of these relations is

\begin{lemma}
\label{lmm:func-operators:functional-commutators}
    \begin{eqnarray*}
        \left[ \hat{\Psi}, f( \hat{\Psi}^\prime, \hat{\Psi}^{\prime\dagger} ) \right]
        & = \delta_{\mathbb{M}} (\boldsymbol{x}^\prime - \boldsymbol{x}) \frac{\partial f}{\partial \hat{\Psi}^{\prime\dagger}} \\
        \left[ \hat{\Psi}^\dagger, f( \hat{\Psi}^\prime, \hat{\Psi}^{\prime\dagger} ) \right]
        & = -\delta_{\mathbb{M}}^* (\boldsymbol{x}^\prime - \boldsymbol{x}) \frac{\partial f}{\partial \hat{\Psi}^\prime},
    \end{eqnarray*}
    where $f(z, z^*)$ is a function that can be expanded into the power series of $z$ and $z^*$.
\end{lemma}

First theorem describes the transformation of the linear part of the Hamiltonian~((\ref{eqn:master-eqn:hamiltonian})).

\begin{theorem}
\label{thm:transformations:w-commutator1}
    \begin{eqnarray*}
        \mathcal{W} \left[ [\int d\boldsymbol{x} \hat{\Psi}_j^\dagger \hat{\Psi}_k, \hat{A}] \right]
        = \int d\boldsymbol{x} \left(
            - \frac{\delta}{\delta \Psi_j} \Psi_k
            + \frac{\delta}{\delta \Psi_k^*} \Psi_j^*
        \right) \mathcal{W}[\hat{A}].
    \end{eqnarray*}
\end{theorem}
\begin{proof}
Proved straightforwardly using Theorem~\ref{thm:func-wigner:correspondences} and the relation
\begin{eqnarray}
    \Psi_k \frac{\delta}{\delta \Psi_j} \mathcal{F}
    = \left(
        \frac{\delta}{\delta \Psi_j} \Psi_k
        - \delta_{jk} \delta_{\mathbb{M}}(\boldsymbol{x}, \boldsymbol{x})
    \right) \mathcal{F}.
\end{eqnarray}
\end{proof}

Commutators with the Laplacian inside require somewhat special treatment, because it acts on basis functions and, in general, cannot be dragged around like a constant.
But for our purposes we only need one specific case, and, fortunately, in this case it does act like a constant.

\begin{theorem}
\label{thm:transformations:w-laplacian-commutator1}
    \begin{eqnarray*}
        \mathcal{W} \left[
            \int d\boldsymbol{x} [\hat{\Psi}^\dagger \nabla^2 \hat{\Psi}, \hat{A}]
        \right]
        = \int d\boldsymbol{x} \left(
            - \frac{\delta}{\delta \Psi} \nabla^2 \Psi
            + \frac{\delta}{\delta \Psi^*} \nabla^2 \Psi^*
        \right) \mathcal{W}[\hat{A}].
    \end{eqnarray*}
\end{theorem}
\begin{proof}
Proved using Theorem~\ref{thm:func-wigner:correspondences} and Lemma~\ref{lmm:func-calculus:move-laplacian}.
\end{proof}

Next theorem describes the transformation of the non-linear part of the Hamiltonian.
[TODO: Can be simplified by assuming the locality of the interaction, thus getting rid of $\boldsymbol{x}^\prime$.]

\begin{theorem}
\label{thm:transformations:w-commutator2}
    \begin{eqnarray*}
        & \mathcal{W} \left[
            [
                \int d\boldsymbol{x} \int d\boldsymbol{x}^\prime
                \hat{\Psi}_j^\dagger \hat{\Psi}_k^{\prime\dagger} \hat{\Psi}_j^\prime \hat{\Psi}_k,
                \hat{A}
            ]
        \right] \\
        & = \int d\boldsymbol{x} \int d\boldsymbol{x}^\prime \left(
            \frac{\delta}{\delta \Psi_j} \left(
                - \Psi_j^\prime \Psi_k \Psi_k^{\prime*}
                + \frac{1}{2} \delta_{jk} \delta_{\mathbb{M}_j}(\boldsymbol{x}^\prime, \boldsymbol{x}^\prime) \Psi_k
                + \frac{1}{2} \delta_{\mathbb{M}_j}(\boldsymbol{x}, \boldsymbol{x}^\prime) \Psi_j^\prime
            \right) \right . \\
        &   \left. + \frac{\delta}{\delta \Psi_j^{\prime*}} \left(
                \Psi_j^* \Psi_k \Psi_k^{\prime*}
                - \frac{1}{2} \delta_{jk} \delta_{\mathbb{M}_j}(\boldsymbol{x}, \boldsymbol{x}) \Psi_k^{\prime*}
                - \frac{1}{2} \delta_{\mathbb{M}_j}(\boldsymbol{x}, \boldsymbol{x}^\prime) \Psi_j^*
            \right) \right. \\
        &   \left. + \frac{\delta}{\delta \Psi_k^\prime} \left(
                - \Psi_j^\prime \Psi_j^* \Psi_k
                + \frac{1}{2} \delta_{jk} \delta_{\mathbb{M}_j}(\boldsymbol{x}, \boldsymbol{x}) \Psi_j^\prime
                + \frac{1}{2} \delta_{\mathbb{M}_k}(\boldsymbol{x}^\prime, \boldsymbol{x}) \Psi_k
            \right) \right .\\
        &   \left. + \frac{\delta}{\delta \Psi_k^*} \left(
                \Psi_j^\prime \Psi_j^* \Psi_k^{\prime*}
                - \frac{1}{2} \delta_{jk} \delta_{\mathbb{M}_j}(\boldsymbol{x}^\prime, \boldsymbol{x}^\prime) \Psi_j^*
                - \frac{1}{2} \delta_{\mathbb{M}_k}(\boldsymbol{x}^\prime, \boldsymbol{x}) \Psi_k^{\prime*}
            \right) \right. \\
        &   \left.
                + \frac{\delta}{\delta \Psi_j}
                \frac{\delta}{\delta \Psi_j^{\prime*}}
                \frac{\delta}{\delta \Psi_k^\prime}
                \frac{1}{4} \Psi_k
                - \frac{\delta}{\delta \Psi_j}
                \frac{\delta}{\delta \Psi_j^{\prime*}}
                \frac{\delta}{\delta \Psi_k^*}
                \frac{1}{4} \Psi_k^{\prime*}
            \right. \\
        &   \left.
                + \frac{\delta}{\delta \Psi_k^\prime}
                \frac{\delta}{\delta \Psi_k^*}
                \frac{\delta}{\delta \Psi_j}
                \frac{1}{4} \Psi_j^\prime
                - \frac{\delta}{\delta \Psi_k^\prime}
                \frac{\delta}{\delta \Psi_k^*}
                \frac{\delta}{\delta \Psi_j^{\prime*}}
                \frac{1}{4} \Psi_j^*
        \right) \mathcal{W}[\hat{A}].
    \end{eqnarray*}
\end{theorem}
\begin{proof}
Proof is the same as in case of Theorem~\ref{thm:transformations:w-commutator1}.
\end{proof}

Finally, the transformation of loss terms~((\ref{eqn:master-eqn:loss-term})) requires more work.
The proof makes use of two auxiliary lemmas.
First one will help us move functional differentials to their intended places (i.e., to the left).

\begin{lemma}
\label{lmm:transformations:swap-differential}
    For $\mathcal{F} \in \mathbb{F}_{\mathbb{M}} \rightarrow \mathbb{F}$ and any non-negative integer $a$, $b$:
    \begin{eqnarray*}
    &    \Psi^a \left( \frac{\delta}{\delta \Psi} \right)^b \mathcal{F}[\Psi, \Psi^*] \\
    &    = \sum_{j=0}^{\min(a, b)}
            \binom{b}{j} \frac{(-1)^j a!}{(a - j)!}
            \delta_{\mathbb{M}}^j(\boldsymbol{x}, \boldsymbol{x})
            \left( \frac{\delta}{\delta \Psi} \right)^{b - j}
            \Psi^{a - j}
            \mathcal{F}[\Psi, \Psi^*]
    \end{eqnarray*}
\end{lemma}
\begin{proof}
Proved straightforwardly by induction.
\end{proof}

Second lemma gives a way to simplify sums obtained from the application of the previous lemma.

\begin{lemma}[Sum rearrangement]
\label{lmm:transformations:sum-rearrangement}
    For any non-negative integer $l$, $u$:
    \begin{eqnarray*}
        \sum_{j=0}^l \sum_{k=0}^{\min(l-u,j)} x^{j-k} Q(j, k)
        = \sum_{v=0}^l x^v \sum_{k=0}^{l-\max(u,v)} Q(v + k, k).
    \end{eqnarray*}
\end{lemma}
\begin{proof}
Can be proved either by formal manipulation with sets, or by drawing a picture.
\end{proof}

Finally, the loss transformation theorem can be proved.

\begin{theorem}
\label{thm:transformations:w-losses}
    The Wigner transformation of loss term~(\ref{eqn:master-eqn:loss-term}) is
    \begin{eqnarray*}
        \mathcal{W} \left[ \int d\boldsymbol{x} \hat{\mathcal{L}}_{\boldsymbol{l}} [\hat{A}] \right]
        = \int d\boldsymbol{x}
            \sum_{j_1=0}^{l_1} \sum_{k_1=0}^{l_1} \ldots
            \sum_{j_C=0}^{l_C} \sum_{k_C=0}^{l_C}
                \left(
                    \prod_{c=1}^C
                        \left( \frac{\delta}{\delta \Psi_c^*} \right)^{j_c}
                        \left( \frac{\delta}{\delta \Psi_c} \right)^{k_c}
                \right)
                L_{\boldsymbol{l}, \boldsymbol{j}, \boldsymbol{k}}
            \mathcal{W}[\hat{A}],
    \end{eqnarray*}
    where
    \begin{eqnarray*}
        L_{\boldsymbol{l}, \boldsymbol{j}, \boldsymbol{k}}
        ={} & \left( 2 - (-1)^{\sum_c j_c} - (-1)^{\sum_c k_c} \right) \\
        &   \prod_{c=1}^C \left(
                \sum_{m_c=0}^{l_c - \max(j_c, k_c)}
                Q(l_c, j_c, k_c, m_c)
                \delta_{\mathbb{M}_c}^{m_c}(\boldsymbol{x}, \boldsymbol{x})
                \Psi_c^{l_c - j_c - m_c}
                (\Psi_c^*)^{l_c - k_c - m_c}
            \right),
    \end{eqnarray*}
    and
    \begin{eqnarray*}
        Q(l, j, k, m)
        = \frac{(-1)^m}{2^{j + k + m}}
            \frac{(l!)^2}{m! j! k! (l - k - m)! (l - j - m)!}.
    \end{eqnarray*}
\end{theorem}
\begin{proof}
Proved by applying Theorem~\ref{thm:func-wigner:correspondences}, expanding products using binomial theorem, using Lemma~\ref{lmm:transformations:swap-differential} to move differentials to front, and applying Lemma~\ref{lmm:transformations:sum-rearrangement} to transform summations.
\end{proof}



% =============================================================================
\section{Wigner truncation and Fokker-Planck equation}
% =============================================================================

Now we have all necessary tools to transform the master equation~(\ref{eqn:master-eqn:master-eqn}) with the Wigner transformation from Definition~\ref{def:func-wigner:w-transformation} to the partial differential equation.

Namely, the single-particle term~(\ref{eqn:master-eqn:single-particle}) is transformed using Theorem~\ref{thm:transformations:w-commutator1} and Theorem~\ref{thm:transformations:w-laplacian-commutator1} (since $K_j$ is basically a sum of Laplacian operator and functions of $\boldsymbol{x}$):
\begin{eqnarray}
    \mathcal{W} \left[ [ \int d\boldsymbol{x} \hat{\Psi}_j^\dagger K_{jk} \hat{\Psi}_k, \hat{\rho} ] \right]
    = \int d\boldsymbol{x} \left(
            - \frac{\delta}{\delta \Psi_j} K_{jk} \Psi_k
            + \frac{\delta}{\delta \Psi_k^*} K_{jk} \Psi_j^*
        \right)
        W,
\end{eqnarray}
where Wigner function $W = \mathcal{W}[\hat{\rho}]$.
Nonlinear term is transformed with Theorem~\ref{thm:transformations:w-commutator2} (minding the locality of interaction and assuming $U_{kj} = U_{jk}$):
\begin{eqnarray}
    \mathcal{W} \left[
        [
            \int d\boldsymbol{x} \frac{U_{jk}}{2}
                \hat{\Psi}_j^\dagger \hat{\Psi}_k^\dagger \hat{\Psi}_j \hat{\Psi}_k,
            \hat{\rho}
        ]
    \right]
    = & \int d\boldsymbol{x} U_{jk} \left(
        \frac{\delta}{\delta \Psi_j} \left(
            - \Psi_j \Psi_k \Psi_k^*
            + \frac{\delta_{\mathbb{M}}(\boldsymbol{x}, \boldsymbol{x})}{2} ( \delta_{jk} \Psi_k + \Psi_j )
        \right) \right. \\
    &   \left. + \frac{\delta}{\delta \Psi_j^*} \left(
            \Psi_j^* \Psi_k \Psi_k^*
            - \frac{\delta_{\mathbb{M}}(\boldsymbol{x}, \boldsymbol{x})}{2} ( \delta_{jk} \Psi_k^* + \Psi_j^* )
        \right) \right. \\
    &   \left.
            + \frac{\delta}{\delta \Psi_j}
            \frac{\delta}{\delta \Psi_j^*}
            \frac{\delta}{\delta \Psi_k}
            \frac{1}{4} \Psi_k
            - \frac{\delta}{\delta \Psi_j}
            \frac{\delta}{\delta \Psi_j^*}
            \frac{\delta}{\delta \Psi_k^*}
            \frac{1}{4} \Psi_k^*
        \right) W.
\end{eqnarray}

Loss terms~(\ref{eqn:master-eqn:loss-term}) are transformed with Theorem~\ref{thm:transformations:w-losses}.
[TODO: Not writing the resulting expression here, because with the absence of truncation it is too long, and is practically the same as in theorem statement.]

Assuming that $K_{jk}$, $U_{jk}$ and $\kappa_{\boldsymbol{l}}$ are real-valued, all the transformations described above result in a partial differential equation for $W$ of the form
\begin{eqnarray}
    \frac{\partial W}{\partial t} = \int d\boldsymbol{x} \left\{
        - \sum_{j=1}^C \frac{\delta}{\delta \Psi_j} \mathcal{A}_j
        - \sum_{j=1}^C \frac{\delta}{\delta \Psi_j^*} \mathcal{A}_j^*
        + \sum_{j=1}^C \sum_{k=1}^C \frac{\delta^2}{\delta \Psi_j^* \delta \Psi_k} \mathcal{D}_{jk}
        + \mbox{O} \left[ \frac{\delta^3}{\delta\Psi_j^3} \right]
    \right\} W.
\end{eqnarray}
The terms of order higher than 2 are produced both by the nonlinear term in the Hamiltonian and loss terms.
Such an equation could be solved without additional approximations if there were only orders up to 3 (which means the absence of losses)~\cite{Polkovnikov2003}, but in most cases all terms except for first- and second-order ones are truncated.
In the assumption of the state being coherent, the condition for truncation can be shown to be~\cite{Sinatra2002}
\begin{eqnarray}
    N \gg |\mathbb{M}|,
\end{eqnarray}
where $N$ is the number of atoms.
This condition is equivalent to~\cite{Norrie2006}
\begin{eqnarray}
    \delta_{\mathbb{M}_j}(\boldsymbol{x}, \boldsymbol{x}) \ll | \Psi_j |^2.
\end{eqnarray}

Wigner truncation allows us to simplify the results of Theorem~\ref{thm:transformations:w-commutator2} and Theorem~\ref{thm:transformations:w-losses}.

\begin{lemma}
    Assuming the conditions for Wigner truncation are satisfied,
    the result of Wigner transformation of the nonlinear term can be written as
    \begin{eqnarray*}
        \mathcal{W} \left[
            [
                \frac{U_{jk}}{2}
                    \hat{\Psi}_j^\dagger \hat{\Psi}_k^\dagger \hat{\Psi}_j \hat{\Psi}_k,
                \hat{\rho}
            ]
        \right]
        = U_{jk} \left(
            \frac{\delta}{\delta \Psi_j^*} \Psi_j^* \Psi_k \Psi_k^*
            - \frac{\delta}{\delta \Psi_j} \Psi_j \Psi_k \Psi_k^*
        \right) W.
    \end{eqnarray*}
\end{lemma}

\begin{lemma}
    Assuming the conditions for Wigner truncation are satisfied, the result of Wigner transformation of the loss term can be written as
    \begin{eqnarray*}
        \mathcal{W}[\mathcal{L}_{\boldsymbol{l}}[\hat{\rho}]]
        = \sum_{n=1}^C
                \frac{\delta}{\delta \Psi_n^*} \frac{\partial O_{\boldsymbol{l}}}{\partial \Psi_n} O_{\boldsymbol{l}}^*
        + \sum_{n=1}^C
            \frac{\delta}{\delta \Psi_n} \frac{\partial O_{\boldsymbol{l}}^*}{\partial \Psi_n^*} O_{\boldsymbol{l}}
        + \sum_{n=1}^C \sum_{p=1}^C
            \frac{\delta^2}{\delta \Psi_n^* \delta \Psi_p}
            \frac{\partial O_{\boldsymbol{l}}}{\partial \Psi_n}
            \frac{\partial O_{\boldsymbol{l}}^*}{\partial \Psi_p^*},
    \end{eqnarray*}
    where $O_{\boldsymbol{l}} \equiv O_{\boldsymbol{l}}[\boldsymbol{\Psi}] = \prod_{c=1}^C \Psi_c^{l_c}$.
\end{lemma}
\begin{proof}
The proof is basically a simplification of the result of Theorem~\ref{thm:transformations:w-losses} under certain conditions.
First, we are neglecting all occurrences of $\delta_{\mathbb{M}}$, which means setting $m_c = 0$ for every $c$.
Second, we are dropping all terms with high order differentials,
which can be expressed as limiting $\sum j_c + \sum k_c \le 2$.
The only combinations of $j_c$ and $k_c$ for which $Z(\boldsymbol{j}, \boldsymbol{k})$ is not zero are thus
$\{ j_c = \delta_{cn}, k_c = 0, n \in [1, C] \}$,
$\{ j_c = 0, k_c = \delta_{cn}, n \in [1, C] \}$ and
$\{ j_c = \delta_{cn}, k_c = \delta_{cp}, n \in [1, C], p \in [1, C] \}$.
These combinations produce terms with $\delta/\delta \Psi_n^*$,
$\delta/\delta \Psi_n$ and
$\delta^2/\delta \Psi_p \delta \Psi_n^*$ respectively.
Applying these conditions one can get the statement of the theorem.
\end{proof}

Thus the truncated Fokker-Planck equation is
\begin{eqnarray}
    \frac{dW}{dt}
    = \int d\boldsymbol{x} \left(
        - \sum_{j=1}^C \frac{\delta}{\delta \Psi_j} \mathcal{A}_j
        - \sum_{j=1}^C \frac{\delta}{\delta \Psi_j^*} \mathcal{A}_j^*
        + \sum_{j=1}^C \sum_{k=1}^C \frac{\delta^2}{\delta \Psi_j^* \delta \Psi_k} \mathcal{D}_{jk}
    \right) W,
\end{eqnarray}
where
\begin{eqnarray}
    \mathcal{A}_j = -\frac{i}{\hbar} \left(
            \sum_{k=1}^C K_{jk} \Psi_k
            + \sum_{k=1}^C U_{jk} \Psi_j \Psi_k \Psi_k^*
        \right)
        - \sum_{\boldsymbol{l}} \kappa_{\boldsymbol{l}} \frac{\partial O_{\boldsymbol{l}}^*}{\partial \Psi_j^*} O_{\boldsymbol{l}},
\end{eqnarray}
and
\begin{eqnarray}
    \mathcal{D}_{jk} = \sum_{\boldsymbol{l}} \kappa_{\boldsymbol{l}}
        \frac{\partial O_{\boldsymbol{l}}}{\partial \Psi_j}
        \frac{\partial O_{\boldsymbol{l}}^*}{\partial \Psi_k^*}.
\end{eqnarray}

Since the diffusion matrix is positive-definite, the truncated Wigner function $W$ is a probability distribution
Therefore the equation can be further transformed to the equivalent set of stochastic differential equations in It\^{o} form as described by Theorem~\ref{thm:app-fpe:fpe-sde-func}:
\begin{eqnarray}\label{eqn:fpe:sdes}
    d\Psi_j = \mathcal{P}_{\mathbb{M}_j} \left[
        \mathcal{A}_j dt + \sum_{\boldsymbol{l}} \mathcal{B}_{j \boldsymbol{l}} Q_{\boldsymbol{l}}
    \right],
\end{eqnarray}
where
\begin{eqnarray}
    \mathcal{B}_{j \boldsymbol{l}} = \sqrt{\kappa_{\boldsymbol{l}}} \frac{\partial O_{\boldsymbol{l}}^*}{\partial \Psi_j^*},
\end{eqnarray}
and $Q_{\boldsymbol{l}}$ is a functional Wiener process:
\begin{eqnarray}
    Q_{\boldsymbol{l}} = \sum_{\boldsymbol{n} \in \mathbb{B}} \phi_j Z_{\boldsymbol{l},\boldsymbol{n}},
\end{eqnarray}
and $Z_{\boldsymbol{l},\boldsymbol{n}}$ are, in turn, independent complex-valued Wiener processes.
Alternatively, in Stratonovich form the SDEs look like
\begin{eqnarray}
    d\Psi_j = \mathcal{P}_{\mathbb{M}_j} \left[
        (\mathcal{A}_j - \mathcal{S}_j) dt + \sum_{\boldsymbol{l}} B_{j \boldsymbol{l}} Q_{\boldsymbol{l}}
    \right],
\end{eqnarray}
where the Stratonovich term is
\begin{eqnarray}
    \mathcal{S}_j
    = \sum_{n=1}^C \sum_{\boldsymbol{l}} \kappa_{\boldsymbol{l}}
        \frac{\partial O_{\boldsymbol{l}}}{\partial \Psi_n}
        \left(\frac{\partial^2 O_{\boldsymbol{l}}}{\partial \Psi_n \partial \Psi_j} \right)^*
        \delta_{\mathbb{M}_n} (\boldsymbol{x}, \boldsymbol{x}).
\end{eqnarray}

These equations can now be solved using conventional methods, and any required expectations symmetrically ordered operator products can be obtained from their solution using Theorem~\ref{thm:func-wigner:moments}:
\begin{eqnarray}
    \langle \left\{
        \prod_{c=1}^C \hat{\Psi}_c^{r_c} (\hat{\Psi}_c^\dagger)^{s_c}
    \right\}_{\mathrm{sym}} \rangle
    & = \int \delta \boldsymbol{\Psi} \left(
            \prod_{c=1}^C \Psi_c^{r_c} (\Psi_c^*)^{s_c}
        \right) W \\
    & \approx \left\{
        \prod_{c=1}^C \Psi_c^{r_c} (\Psi_c^*)^{s_c}
    \right\}_{\mathrm{paths}},
\end{eqnarray}
where $r_c$ and $s_c$ is some set of non-negative integers, and $\left\{\right\}_{\mathrm{paths}}$ stands for the average over the simulation paths.



% =============================================================================
\section{Initial states}
% =============================================================================

Initial values for the numerical integration of equations~(\ref{eqn:fpe:sdes}) are obtained by finding the Wigner transformation of the density matrix for the desired initial state, and then sampling the initial values according to the resulting Wigner function.
As an example, consider the simple case with a coherent initial state.

\begin{theorem}
    The Wigner distribution for a multi-mode coherent state with expectation values
    $\alpha_{\boldsymbol{n}}(0) = \alpha_{\boldsymbol{n}}^{(0)}$, $\boldsymbol{n} \in \mathbb{M}$ is
    \begin{eqnarray*}
        W_c (\boldsymbol{\alpha}^{(0)})
        = \left( \frac{2}{\pi} \right)^{|\mathbb{M}|} \prod_{\boldsymbol{n} \in \mathbb{M}}
            \exp(-2 |\alpha_{\boldsymbol{n}} - \alpha_{\boldsymbol{n}}^{(0)}|^2).
    \end{eqnarray*}
\end{theorem}
\begin{proof}
The density matrix of the state is
\begin{eqnarray}
    \hat{\rho}
    = \vert \alpha_{\boldsymbol{n}}^{(0)},\, \boldsymbol{n} \in \mathbb{M} \rangle
        \langle \alpha_{\boldsymbol{n}}^{(0)},\, \boldsymbol{n} \in \mathbb{M} \vert
    = \left( \prod_{\boldsymbol{n} \in \mathbb{M}} \vert \alpha_{\boldsymbol{n}}^{(0)} \rangle \right)
        \left( \prod_{\boldsymbol{n} \in \mathbb{M}} \langle \alpha_{\boldsymbol{n}}^{(0)} \vert \right).
\end{eqnarray}
Then the characteristic function is
\begin{eqnarray}
    \chi_W (\boldsymbol{\alpha}^{(0)})
    = \prod_{\boldsymbol{n} \in \mathbb{M}}
        \langle \alpha_{\boldsymbol{n}}^{(0)} \vert
        \hat{D}_{\boldsymbol{n}} (\lambda_{\boldsymbol{n}}, \lambda_{\boldsymbol{n}}^*)
        \vert \alpha_{\boldsymbol{n}}^{(0)} \rangle
\end{eqnarray}
Using the properties of the displacement operator, this can be transformed to
\begin{eqnarray}
    \chi_W (\boldsymbol{\alpha}^{(0)})
    = \prod_{\boldsymbol{n} \in \mathbb{M}}
        \exp(
            - \lambda_{\boldsymbol{n}}^* \alpha_{\boldsymbol{n}}^{(0)}
            + \lambda_{\boldsymbol{n}} (\alpha_{\boldsymbol{n}}^{(0)})^*
            - \frac{1}{2} |\lambda|^2
        ).
\end{eqnarray}
Finally, Wigner function is
\begin{eqnarray}
    W_c (\boldsymbol{\alpha}^{(0)})
    & = \frac{1}{\pi^{2|\mathbb{M}|}} \prod_{\boldsymbol{n} \in \mathbb{M}} \left(
        \int d^2\lambda_{\boldsymbol{n}}
            \exp(
                - \lambda_{\boldsymbol{n}} (\alpha_{\boldsymbol{n}}^* - (\alpha_{\boldsymbol{n}}^{(0)})^*)
                + \lambda_{\boldsymbol{n}}^* (\alpha_{\boldsymbol{n}} - \alpha_{\boldsymbol{n}}^{(0)})
                - \frac{1}{2} |\lambda|^2
            )
    \right) \\
    & = \left( \frac{2}{\pi} \right)^{|\mathbb{M}|} \prod_{\boldsymbol{n} \in \mathbb{M}}
        \exp(-2 |\alpha_{\boldsymbol{n}} - \alpha_{\boldsymbol{n}}^{(0)}|^2).
    \qedhere
\end{eqnarray}
\end{proof}

The resulting Wigner distribution is a product of independent complex-valued Gaussian distributions for each mode,
with the expectation value equal to the expectation value of the mode,
and the variance equal to $\frac{1}{2}$.
Therefore the initial state can be sampled as
\begin{eqnarray}
    \alpha_{\boldsymbol{n}} = \alpha_{\boldsymbol{n}}^{(0)} + \frac{1}{\sqrt{2}} \eta_{\boldsymbol{n}},
\end{eqnarray}
where $\eta_{\boldsymbol{n}}$ are normally distributed complex random numbers with zero mean,
$\langle \eta_{\boldsymbol{m}} \eta_{\boldsymbol{n}} \rangle = 0$ and
$\langle \eta_{\boldsymbol{m}} \eta_{\boldsymbol{n}}^* \rangle = \delta_{\boldsymbol{m},\boldsymbol{n}}$
(in other words, with components distributed independently with variance $\frac{1}{2}$).
This looks like adding half a ``vacuum particle'' to each mode.
In functional form this can be written as
\begin{eqnarray}
    \Psi_j(\boldsymbol{x}, 0)
    = \Psi_j^{(0)}(\boldsymbol{x}, 0)
        + \sum_{\boldsymbol{n} \in \mathbb{M}_j} \frac{\eta_{j,\boldsymbol{n}}}{\sqrt{2}} \phi_{j,\boldsymbol{n}},
\end{eqnarray}
where $\Psi_j^{(0)}(\boldsymbol{x}, 0)$ is the ``classical'' ground state of the system.

[TODO: We can describe the Wigner function for the number state here, as an example of non-positive Wigner function.]



% =============================================================================
\section{Conclusion}
% =============================================================================

[TODO: Conclusion goes here.]


\appendix



% =============================================================================
\section{Wirtinger differentiation}
% =============================================================================

In this paper we are using differentiation of complex functions extensively.
Instead of the classical definition of differential which only works for holomorphic functions, we use Wirtinger differentiation~\cite{Wirtinger1927}.
One can find thorough description of these rules, for example, in~\cite{Kreutz-Delgado2009}; in this section we will only outline the basics.

\begin{definition}
    For a complex variable $z = x + iy$ and a function $f(z) = u(x, y) + iv(x, y)$ the Wirtinger differential is
    \begin{eqnarray*}
        \frac{\partial f(z)}{\partial z}
        = \frac{1}{2} \left(
            \frac{\partial f}{\partial x} - i \frac{\partial f}{\partial y}
        \right).
    \end{eqnarray*}
\end{definition}

One can easily prove that if $f(z)$ is holomorphic, then the above definition coincides with the classical differential for complex functions.
Wirtinger differential obeys sum, product, quotient, and chain differentiation rules (the former one is applied as if $f(z) \equiv f(z, z^*)$).

In addition, we will need a non-standard integration over the complex variable:

\begin{definition}
    For a complex variable $z = x + iy$ the integral
    \begin{eqnarray*}
        \int d^2 z \equiv \int_{-\infty}^{\infty} \int_{-\infty}^{\infty} dx\, dy,
    \end{eqnarray*}
    or, in other words, stands for the two-dimensional integral over the complex plane.
\end{definition}

Such integration has a property similar to a Fourier transformation in real space.

\begin{lemma}
\label{lmm:c-numbers:fourier-of-moments}
    If $\lambda$ is a complex variable, then for any non-negative integers $r$ and $s$:
    \begin{eqnarray*}
        \int d^2\alpha\, \alpha^r (\alpha^*)^s \exp(-\lambda \alpha^* + \lambda^* \alpha)
        = \pi^2
            \left( -\frac{\partial}{\partial \lambda^*} \right)^r
            \left( \frac{\partial}{\partial \lambda} \right)^s
            \delta(\mathrm{Re} \lambda) \delta(\mathrm{Im} \lambda)
    \end{eqnarray*}
\end{lemma}
\begin{proof}
First, using known Fourier transform relations, it is easy to prove that for real $x$ and $v$, and non-negative integer $n$
\begin{eqnarray*}
    \int\limits_{-\infty}^{\infty} dv\, v^n \exp(\pm 2 i x v)
    = \pi (\mp i / 2)^n \delta^{(n)}(x).
\end{eqnarray*}
Substituting $\alpha = x + iy$, expanding the $\alpha^r (\alpha^*)^s$ term using binomial theorem and employing the above property, one can reach the statement of the lemma.
\end{proof}

Another important property is used extensively throughout the paper.

\begin{lemma}
\label{lmm:c-numbers:zero-integrals}
    If $f(\lambda, \lambda^*)$ is bounded, then for any complex $\alpha$:
    \begin{eqnarray*}
        \int d^2\lambda
            \frac{\partial}{\partial \lambda} \left(
                \exp(-\lambda \alpha^* + \lambda^* \alpha)
                f(\lambda, \lambda^*)
            \right)
        & = 0 \\
        \int d^2\lambda
            \frac{\partial}{\partial \lambda^*}
            \left(
                \exp(-\lambda \alpha^* + \lambda^* \alpha)
                f(\lambda, \lambda^*)
            \right)
        & = 0.
    \end{eqnarray*}
\end{lemma}
\begin{proof}
[TODO: Needs proof?]
\end{proof}



% =============================================================================
\section{Functional calculus}
% =============================================================================

This section outlines the functional calculus, which is heavily used throughout the paper.
Detailed description is given in~\cite{Dalton2011}, and here we only provide some important definitions and results which are used later in the paper.
In this section we will use the definitions from the Section~\ref{sec:func-operators}, namely the full basis $\mathbb{B}$ and the restricted basis $\mathbb{M}$.
Given the basis, we can define the correspondence between some function of coordinates and its representation in mode space.

\begin{definition}
    Let $\mathbb{F}$ be a space of all functions of coordinates, which consists only of modes from $\mathbb{M}$: $\mathbb{F}_{\mathbb{M}} \equiv (\mathbb{R}^D \rightarrow \mathbb{C})_{\mathbb{M}}$ (restricted functions).
    Composition transformation creates a function from a vector of mode populations:
    \begin{eqnarray*}
        & \mathcal{C}_{\mathbb{M}} :: \mathbb{C}^{|\mathbb{M}|} \rightarrow \mathbb{F}_{\mathbb{M}} \\
        & \mathcal{C}_{\mathbb{M}}(\boldsymbol{\alpha}) = \sum_{\boldsymbol{n} \in \mathbb{M}} \phi_{\boldsymbol{n}} \alpha_{\boldsymbol{n}}.
    \end{eqnarray*}
    Decomposition transformation, correspondingly, creates a vector of populations out of a function:
    \begin{eqnarray*}
        & \mathcal{C}_{\mathbb{M}}^{-1} :: \mathbb{F} \rightarrow \mathbb{C}^{|\mathbb{M}|} \\
        & (\mathcal{C}_{\mathbb{M}}^{-1}[f])_{\boldsymbol{n}}
        = \int d\boldsymbol{x} \phi_{\boldsymbol{n}}^* f,\,{\boldsymbol{n}} \in \mathbb{M}.
    \end{eqnarray*}
    Note that for any $f \in \mathbb{F}_{\mathbb{M}}$, $\mathcal{C}_{\mathbb{M}}(\mathcal{C}_{\mathbb{M}}^{-1}[f]) \equiv f$.
\end{definition}

The result of any non-linear transformation of a function $f \in \mathbb{F}_{\mathbb{M}}$ is not guaranteed to belong to $\mathbb{F}_{\mathbb{M}}$ and requires explicit projection to be used with other restricted functions.
This applies to the delta function of coordinates.
To avoid confusion with the common delta function, we introduce the restricted delta function.

\begin{definition}
\label{def:func-calculus:restricted-delta}
    The restricted delta function $\delta_{\mathbb{M}} \in \mathbb{F}_{\mathbb{M}}$ is defined as
    \begin{eqnarray*}
        \delta_{\mathbb{M}}(\boldsymbol{x}^\prime, \boldsymbol{x})
        = \sum_{\boldsymbol{n} \in \mathbb{M}} \phi_{\boldsymbol{n}}^{\prime*} \phi_{\boldsymbol{n}}.
    \end{eqnarray*}
    Note that $\delta_{\mathbb{M}}^*(\boldsymbol{x}^\prime, \boldsymbol{x}) = \delta_{\mathbb{M}}(\boldsymbol{x}, \boldsymbol{x}^\prime)$.
\end{definition}

Any function can be projected to $\mathbb{M}$ using the projection transformation.

\begin{definition}
\label{def:func-calculus:projector}
    Projection transformation
    \begin{eqnarray*}
        & \mathcal{P}_{\mathbb{M}} ::
        \mathbb{F} \rightarrow \mathbb{F}_{\mathbb{M}} \\
        & \mathcal{P}_{\mathbb{M}}[f](\boldsymbol{x})
        = (\mathcal{C}_{\mathbb{M}}(\mathcal{C}_{\mathbb{M}}^{-1}[f])) (\boldsymbol{x})
        = \sum_{\boldsymbol{n} \in \mathbb{M}} \phi_{\boldsymbol{n}} \int
            d\boldsymbol{x}^\prime\, \phi_{\boldsymbol{n}}^{\prime*} f^\prime
        = \int d\boldsymbol{x}^\prime \delta_{\mathbb{M}}(\boldsymbol{x}^\prime, \boldsymbol{x}) f^\prime.
    \end{eqnarray*}
    Obviously, $\mathcal{P}_{\mathbb{B}} \equiv \mathds{1}$.
\end{definition}

The conjugate of $\mathcal{P}_{\mathbb{M}}$ is thus defined as
\begin{eqnarray}
    (\mathcal{P}_{\mathbb{M}}[f](\boldsymbol{x}))^*
    = \int d\boldsymbol{x}^\prime \delta_{\mathbb{M}}^*(\boldsymbol{x}^\prime, \boldsymbol{x}) f^{\prime*}
    = \mathcal{P}_{\mathbb{M}}^* [f^*](\boldsymbol{x}).
\end{eqnarray}

Let $\mathcal{F}[f] :: \mathbb{F}_{\mathbb{M}} \rightarrow \mathbb{F}$ be some transformation (note that the result is not guaranteed to belong to the restricted basis).
Because of the bijection between $\mathbb{F}_{\mathbb{M}}$ and $\mathbb{C}^{|\mathbb{M}|}$, $\mathcal{F}$ can be alternatively treated as a function of a vector of complex numbers:
\begin{eqnarray}
    & \mathcal{F} :: \mathbb{C}^{|\mathbb{M}|} \rightarrow \mathbb{C}^\infty \\
    & \mathcal{F}(\boldsymbol{\alpha}) \equiv \mathcal{C}_{\mathbb{M}}^{-1}[\mathcal{F}[\mathcal{C}_{\mathbb{M}}(\boldsymbol{\alpha})]].
\end{eqnarray}
Using this correspondence, we can define the functional differentiation.

\begin{definition}
\label{def:func-calculus:func-diff}
    Functional derivative is defined as
    \begin{eqnarray*}
        & \frac{\delta}{\delta f^\prime} ::
        \left(
            \mathbb{F}_{\mathbb{M}} \rightarrow \mathbb{F}
        \right)
        \rightarrow
        \left(
            \mathbb{R}^D \rightarrow \mathbb{F}_{\mathbb{M}} \rightarrow \mathbb{F}
        \right) \\
        & \frac{\delta \mathcal{F}[f]}{\delta f^\prime}
        = \sum_{\boldsymbol{n} \in \mathbb{M}} \phi_{\boldsymbol{n}}^{\prime*}
            \frac{\partial \mathcal{F}(\boldsymbol{\alpha})}{\partial \alpha_{\boldsymbol{n}}}.
    \end{eqnarray*}
\end{definition}

Note that the transformation being returned differs from the one which was taken: the result of the new transformation is a function of the additional variable from $\mathbb{R}^D$ ($\boldsymbol{x}^\prime$).
This variable comes from the function we are differentiating by.

Functional derivatives behave in many ways similar to Wirtinger derivatives.
The detailed treatment can be found in~\cite{Dalton2011}.
In particular, the following useful lemma gives us the ability to differentiate functionals based on the intuition for common functions:

\begin{lemma}
    If $g(z)$ is a function of complex variable that can be expanded into series of $z^n (z^*)^m$, and functional $\mathcal{F}[f, f^*] \equiv g(f, f^*)$, $\mathcal{F} \in \mathbb{F}_{\mathbb{M}} \rightarrow \mathbb{F}$, then $\delta \mathcal{F} / \delta f^\prime$ and $\delta \mathcal{F} / \delta f^{\prime*}$ can be treated as partial differentiation of the functional of two independent variables $f$ and $f^*$.
    In other words:
    \begin{eqnarray*}
        \frac{\delta \mathcal{F}}{\delta f^\prime}
        = \delta_{\mathbb{M}}(\boldsymbol{x}^\prime, \boldsymbol{x})
            \frac{\partial g(f, f^*)}{\partial f},
        \qquad
        \frac{\delta \mathcal{F}}{\delta f^{\prime*}}
        = \delta_{\mathbb{M}}^*(\boldsymbol{x}^\prime, \boldsymbol{x})
            \frac{\partial g(f, f^*)}{\partial f^*}
    \end{eqnarray*}
\end{lemma}

Functional integration is defined as

\begin{definition}
    \begin{eqnarray*}
        & \int \delta^2 f :: (\mathbb{F}_{\mathbb{M}} \rightarrow \mathbb{F}) \rightarrow \mathbb{C} \\
        & \int \delta^2 f \mathcal{F}[f]
        = \int d^2\boldsymbol{\alpha} \mathcal{F}(\boldsymbol{\alpha})
        = \left(
            \prod_{\boldsymbol{n} \in \mathbb{M}} \int d^2\alpha_{\boldsymbol{n}}
        \right) \mathcal{F}(\boldsymbol{\alpha}).
    \end{eqnarray*}
    If the basis contains an infinite number of modes, the integral is treated as a limit $|\mathbb{M}| \rightarrow \infty$.
    [TODO: Product of integrals means successive applications of those integrals --- do we need to state it explicitly?]
\end{definition}

Functional integration has the Fourier-like property analogous to Lemma~Lemma~\ref{lmm:c-numbers:fourier-of-moments}, but its statement requires the definition of the delta functional:

\begin{definition}
\label{def:func-calculus:delta-functional}
    For a function $\Lambda \in \mathbb{F}_{\mathbb{M}}$ the delta functional is
    \begin{eqnarray*}
        \Delta_{\mathbb{M}}[\Lambda]
        \equiv \prod_{\boldsymbol{n} \in \mathbb{M}} \delta(\mathrm{Re} \lambda_{\boldsymbol{n}}) \delta(\mathrm{Im} \lambda_{\boldsymbol{n}}),
    \end{eqnarray*}
    where $\boldsymbol{\lambda} = \mathcal{C}_{\mathbb{M}}^{-1}[\Lambda]$.
\end{definition}

The delta functional has the same property as the common delta function:
\begin{eqnarray}
    \int \delta^2 \Lambda \mathcal{F}[\Lambda] \Delta_{\mathbb{M}}[\Lambda]
    & = \left(
            \prod_{\boldsymbol{n} \in \mathbb{M}} \int d^2\lambda_{\boldsymbol{n}}
        \right)
        \mathcal{F}(\boldsymbol{\lambda})
        \prod_{\boldsymbol{n} \in \mathbb{M}} \delta(\mathrm{Re} \lambda_{\boldsymbol{n}}) \delta(\mathrm{Im} \lambda_{\boldsymbol{n}}) \\
    & = \left. \mathcal{F}(\boldsymbol{\lambda}) \right|_{\forall \boldsymbol{n} \in \mathbb{M}\, \lambda_{\boldsymbol{n}} = 0} \\
    & = \left. \mathcal{F}[\Lambda] \right|_{\Lambda \equiv 0}
\end{eqnarray}

\begin{lemma}[Functional extension of Lemma~\ref{lmm:c-numbers:fourier-of-moments}]
\label{lmm:func-calculus:fourier-of-moments}
    For $\Psi \in \mathbb{F}_{\mathbb{M}}$ and $\Lambda \in \mathbb{F}_{\mathbb{M}}$, and for any non-negative integers $r$ and $s$:
    \begin{eqnarray*}
        \int \delta^2\Psi\, \Psi^r (\Psi^*)^s \exp \left(
                \int d\boldsymbol{x} \left( -\Lambda \Psi^* + \Lambda^* \Psi \right)
            \right) \\
        = \pi^{2|\mathbb{M}|}
            \left( -\frac{\delta}{\delta \Lambda^*} \right)^r
            \left( \frac{\delta}{\delta \Lambda} \right)^s
            \Delta_{\mathbb{M}}[\Lambda]
    \end{eqnarray*}
\end{lemma}
\begin{proof}
The proof consists of expanding functions into sums of modes and applying Lemma~\ref{lmm:c-numbers:fourier-of-moments} $|\mathbb{M}|$ times.
\end{proof}

\begin{lemma}
\label{lmm:func-calculus:zero-integrals}
    For a bounded functional $F(\boldsymbol{\lambda}, \boldsymbol{\lambda}^*)$
    \begin{eqnarray*}
        \int \delta^2\Lambda
            \frac{\delta}{\delta \Lambda^\prime} \left(
                D[\Lambda, \Lambda^*, \Psi, \Psi^*]
                F[\Lambda, \Lambda^*]
            \right)
        & = 0 \\
        \int \delta^2\Lambda
            \frac{\delta}{\delta \Lambda^{\prime*}}
            \left(
                D[\Lambda, \Lambda^*, \Psi, \Psi^*]
                F[\Lambda, \Lambda^*]
            \right)
        & = 0.
    \end{eqnarray*}
\end{lemma}
\begin{proof}
Proved by expanding integrals and differentials into modes and applying Lemma~\ref{lmm:c-numbers:zero-integrals}.
\end{proof}

\begin{lemma}
\label{lmm:func-calculus:zero-delta-integrals}
    For $\Lambda \in \mathbb{F}_{\mathbb{M}}$ [TODO: Any limitations on $F$?]
    \begin{eqnarray*}
        \int \delta^2\Lambda
            \frac{\delta}{\delta \Lambda} \left(
                \left(
                    \left( \frac{\delta}{\delta \Lambda} \right)^s
                    \left( -\frac{\delta}{\delta \Lambda^*} \right)^r
                    \Delta_{\mathbb{M}}[\Lambda]
                \right)
                F[\lambda, \lambda^*]
            \right)
        & = 0 \\
        \int \delta^2\Lambda
            \frac{\delta}{\delta \Lambda^*} \left(
                \left(
                    \left( \frac{\delta}{\delta \Lambda} \right)^s
                    \left( -\frac{\delta}{\delta \Lambda^*} \right)^r
                    \Delta_{\mathbb{M}}[\Lambda]
                \right)
                F[\lambda, \lambda^*]
            \right)
        & = 0 \\
    \end{eqnarray*}
\end{lemma}
\begin{proof}
Proved by expanding functional integration and differentials into modes and integrating separately over each $\lambda_{\boldsymbol{n}}$, using the fact that any differential of the delta function is zero on the infinity.
\end{proof}

In order to perform transformations of master equations, we will need a lemma that justifies the ``relocation'' of the Laplacian (which is a part of the kinetic term in the Hamiltonian) inside the functional integral.

\begin{lemma}
\label{lmm:func-calculus:move-laplacian}
    If $\mathcal{F} \in \mathbb{F}_{\mathbb{M}} \rightarrow \mathbb{F}$, and $\forall \boldsymbol{n} \in \mathbb{M}, \boldsymbol{x} \in \partial A$ $\phi_{\boldsymbol{n}}(\boldsymbol{x}) = 0$, then
    \begin{eqnarray*}
        \int\limits_A d\boldsymbol{x} \left(
            \nabla^2 \frac{\delta}{\delta \Psi}
        \right) \Psi \mathcal{F}[\Psi, \Psi^*]
        = \int\limits_A d\boldsymbol{x} \frac{\delta}{\delta \Psi}
        ( \nabla^2 \Psi ) \mathcal{F}[\Psi, \Psi^*]
    \end{eqnarray*}
\end{lemma}
\begin{proof}
The proof consists of a function expansion into a mode sum and an application of Green's first identity.
\end{proof}

Note that the above lemma imposes an additional requirement for basis functions, but in practical applications it is always satisfied.
For example, in plane wave basis eigenfunctions are equal to zero at the border of the bounding box, and in harmonic oscillator basis they are equal to zero on the infinity (which can be considered the boundary of their integration area).
Hereinafter we will assume that this condition is true for any basis we work with.



% =============================================================================
\section{Functional Fokker-Planck equation}
% =============================================================================

The general approach to numerical solution of the Fokker-Planck equation is to transform it to the equivalent set of stochastic differential equations (SDEs).
In the textbooks this transformation is defined for real variables only~\cite{Risken1996}, while we have functional FPE with complex-valued functions.

Our starting point is the reformulation of the theorem for real-valued multivariable FPE from~\cite{Risken1996} in terms of vectors and matrices:

\begin{lemma}[FPE--SDEs correspondence in convenient form.]
\label{lmm:app-fpe:fpe-sde-real}
    If $\boldsymbol{z}^T \equiv (z_1 \ldots z_M)$ is a set of real-valued variables,
    Fokker-Planck equation
    \begin{eqnarray*}
        \frac{dW}{dt}
        = -\boldsymbol{\partial}_{\boldsymbol{z}}^T \boldsymbol{a} W
        + \frac{1}{2} \mathrm{Tr} \left\{ \boldsymbol{\partial}_{\boldsymbol{z}} \boldsymbol{\partial}_{\boldsymbol{z}}^T B B^T \right\} W
    \end{eqnarray*}
    is equivalent to a set of stochastic differential equations in It\^{o} form
    \begin{eqnarray*}
        d\boldsymbol{z} = \boldsymbol{a} dt + B d\boldsymbol{Z}
    \end{eqnarray*}
    and to a set of stochastic differential equations in Stratonovich form
    \begin{eqnarray*}
        d\boldsymbol{z} = (\boldsymbol{a} - \boldsymbol{s})dt + B d\boldsymbol{Z},
    \end{eqnarray*}
    where the noise-induced (or spurious) drift vector $\boldsymbol{s}$ has elements
    \begin{eqnarray*}
        s_i
        = \sum_{k,j} B_{kj} \frac{\partial}{\partial z_k} B_{ij}
        = \mathrm{Tr} \left\{ B^T \boldsymbol{\partial}_z \boldsymbol{e}_i^T B \right\},
    \end{eqnarray*}
    $\boldsymbol{e}_i$ being the unit vector with elements $(\boldsymbol{e}_i)_j = \delta_{ij}$.
    Here $W \equiv W(\boldsymbol{z})$ is a probability distribution,
    $\boldsymbol{a} \equiv \boldsymbol{a}(\boldsymbol{z})$ is a vector function,
    $B \equiv B(\boldsymbol{z})$ is a matrix function ($B$ having size $M \times L$, where $L$ corresponds to the number of noise sources),
    $\boldsymbol{\partial}_{\boldsymbol{z}}^T \equiv (\partial_{z_1} \ldots \partial_{z_M})$ is a vector differential,
    and $d\boldsymbol{Z}$ is a standard $L$-dimensional real-valued Wiener process.
\end{lemma}
\begin{proof}
For details see~\cite{Risken1996}, sections 3.3 and 3.4.
\end{proof}

\begin{theorem}
\label{thm:app-fpe:fpe-sde-complex}
    If $\boldsymbol{\alpha}^T \equiv (\alpha_1 \ldots \alpha_M)$ is a set of complex-valued variables,
    Fokker-Planck equation
    \begin{eqnarray*}
        \frac{dW}{dt}
        = -\boldsymbol{\partial}_{\boldsymbol{\alpha}}^T \boldsymbol{a} W - \boldsymbol{\partial}_{\boldsymbol{\alpha}^*}^T \boldsymbol{a}^* W
        + \mathrm{Tr} \left\{ \boldsymbol{\partial}_{\boldsymbol{\alpha}^*} \boldsymbol{\partial}_{\boldsymbol{\alpha}}^T B B^H \right\} W
    \end{eqnarray*}
    is equivalent to a set of stochastic differential equations in It\^{o} form
    \begin{eqnarray*}
        d\boldsymbol{\alpha} = \boldsymbol{a} dt + B d\boldsymbol{Z},
    \end{eqnarray*}
    or to Stratonovich form
    \begin{eqnarray*}
        d\boldsymbol{\alpha} = (\boldsymbol{a} - \boldsymbol{s}) dt + B d\boldsymbol{Z},
    \end{eqnarray*}
    where noise-induced drift term is
    \begin{eqnarray*}
        s_j = \mathrm{Tr} \left\{ B^H \boldsymbol{\partial}_{\boldsymbol{\alpha}^*} \boldsymbol{e}_j^T B \right\},
    \end{eqnarray*}
    and $d\boldsymbol{Z} = (d\boldsymbol{X} + id\boldsymbol{Y}) / \sqrt{2}$ is an $M$-dimensional complex-valued Wiener process,
    containing two real-valued $L$-dimensional Wiener processes $d\boldsymbol{X}$ and $d\boldsymbol{Y}$.
\end{theorem}
\begin{proof}
Proved straightforwardly by transforming the equation to real variables and applying Lemma~\ref{lmm:app-fpe:fpe-sde-real}.
\end{proof}

\begin{theorem}[Multi-component extension of Theorem~\ref{thm:app-fpe:fpe-sde-complex}]
\label{thm:app-fpe:mc-fpe-sde}
    [TODO: Perhaps not necessary.]
    If $\boldsymbol{\alpha}^{(c)},\, c = 1..C$ are $C$ sets of complex variables $\boldsymbol{\alpha}^{(c)} \equiv (\alpha_1^{(c)} \ldots \alpha_M^{(c)})$, then the Fokker-Planck equation
    \begin{eqnarray}
        \frac{dW}{dt}
        = & - \sum_{c=1}^C \boldsymbol{\partial}_{\boldsymbol{\alpha}^{(c)}}^T \boldsymbol{a}^{(c)} W
        - \sum_{c=1}^C \boldsymbol{\partial}_{(\boldsymbol{\alpha}^{(c)})^*}^T (\boldsymbol{a}^{(c)})^* W \\
        & + \sum_{m=1}^c \sum_{n=1}^c
            \mathrm{Tr} \left\{
                \boldsymbol{\partial}_{(\boldsymbol{\alpha}^{(m)})^*}
                \boldsymbol{\partial}_{\boldsymbol{\alpha}^{(n)}}^T
                B^{(n)} (B^{(m)})^H
            \right\} W
    \end{eqnarray}
    is equivalent to a set of stochastic differential equations in It\^{o} form
    \begin{eqnarray}
        d\boldsymbol{\alpha}^{(c)} = \boldsymbol{a}^{(c)} dt + B^{(c)} d\boldsymbol{Z},\, c = 1..C
    \end{eqnarray}
    or to Stratonovich form
    \begin{eqnarray*}
        d\boldsymbol{\alpha}^{(c)} = (\boldsymbol{a}^{(c)} - \boldsymbol{s}^{(c)}) dt + B^{(c)} d\boldsymbol{Z},
    \end{eqnarray*}
    where noise-induced drift term is
    \begin{eqnarray*}
        s_j^{(c)} = \sum_{d=1}^C
            \mathrm{Tr} \left\{ (B^{(d)})^H \boldsymbol{\partial}_{(\boldsymbol{\alpha}^{(d)})^*} \boldsymbol{e}_j^T B^{(c)} \right\},
    \end{eqnarray*}
    and $d\boldsymbol{Z}$ is an $L$-dimensional complex-valued Wiener process.
\end{theorem}
\begin{proof}
Proved by joining vectors from all components into one vector and applying Theorem~\ref{thm:app-fpe:fpe-sde-complex}.
\end{proof}

\begin{theorem}
\label{thm:app-fpe:fpe-sde-func}
    For the probability distribution $W[\boldsymbol{\Psi}, \boldsymbol{\Psi}^*] \in (\mathbb{F}_{\mathbb{M}}^C \rightarrow \mathbb{R})$ the functional FPE
    \begin{eqnarray*}
        \frac{dW}{dt}
        = \int d\boldsymbol{x} \left(
            - \sum_{j=1}^C \frac{\delta}{\delta \Psi_j} \mathcal{A}_j
            - \sum_{j=1}^C \frac{\delta}{\delta \Psi_j^*} \mathcal{A}_j^*
            + \sum_{j=1}^C \sum_{k=1}^C \frac{\delta^2}{\delta \Psi_j^* \delta \Psi_k}
                \sum_{\boldsymbol{l}} \mathcal{B}_{k \boldsymbol{l}} \mathcal{B}_{j \boldsymbol{l}}^*
        \right) W
    \end{eqnarray*}
    [TODO: alternatively, in matrix form]
    \begin{eqnarray*}
        \frac{dW}{dt}
        = \int d\boldsymbol{x} \left(
            - 2 \mathrm{Re} \left( \boldsymbol{\delta}_{\boldsymbol{\Psi}} \cdot \vec{\mathcal{A}} \right)
            + \mathrm{Tr} \left\{ \boldsymbol{\delta}_{\boldsymbol{\Psi}^*} \boldsymbol{\delta}_{\boldsymbol{\Psi}}^T \mathcal{B} \mathcal{B}^H \right\}
        \right) W
    \end{eqnarray*}
    is equivalent to the set of SDEs in It\^{o} form
    \begin{eqnarray*}
        d\Psi_j = \mathcal{P}_{\mathbb{M}_j} \left[
            \mathcal{A}_j dt + \sum_{\boldsymbol{l}} \mathcal{B}_{j \boldsymbol{l}} dQ_{\boldsymbol{l}}
        \right],
    \end{eqnarray*}
    [TODO: alternatively, in matrix form]
    \begin{eqnarray*}
        d\boldsymbol{\Psi} = \vec{\mathcal{P}} \left[
            \vec{\mathcal{A}} dt + \mathcal{B} d\boldsymbol{Q}
        \right]
    \end{eqnarray*}
    or in Stratonovich form
    \begin{eqnarray*}
        d\Psi_j = \mathcal{P}_{\mathbb{M}_j} \left[
            (\mathcal{A}_j - \mathcal{S}_j) dt + \sum_{\boldsymbol{l}} \mathcal{B}_{j \boldsymbol{l}} dQ_{\boldsymbol{l}}
        \right],
    \end{eqnarray*}
    where
    \begin{eqnarray*}
        \mathcal{S}_j = \sum_{c=1}^C \sum_{\boldsymbol{l}}
            \mathcal{B}_{c \boldsymbol{l}}^*
            \frac{\delta}{\delta \Psi_c^*}
            \mathcal{B}_{j \boldsymbol{l}},
    \end{eqnarray*}
    [TODO: alternatively, in matrix form]
    \begin{eqnarray*}
        \mathcal{S}_j = \mathrm{Tr} \left\{ \mathcal{B}^H \boldsymbol{\partial}_{\boldsymbol{\Psi}^*} \boldsymbol{e}_j^T \mathcal{B} \right\},
    \end{eqnarray*}
    and $Q_{\boldsymbol{l}}$ is a functional Wiener process:
    \begin{eqnarray*}
        Q_{\boldsymbol{l}} = \sum_{\boldsymbol{n} \in \mathbb{B}} \phi_{\boldsymbol{n}} Z_{\boldsymbol{l},\boldsymbol{n}}.
    \end{eqnarray*}
\end{theorem}
\begin{proof}
Proved by expanding functional derivatives and applying Theorem~\ref{thm:app-fpe:mc-fpe-sde}.
The diffusion term has to be transformed in order to conform to the theorem:
\begin{eqnarray}
    \int d\boldsymbol{x} \phi_{j,\boldsymbol{m}} \phi_{k,\boldsymbol{n}}^* \sum_{\boldsymbol{l}} \mathcal{B}_{\boldsymbol{l}}^{(k)} (\mathcal{B}_{\boldsymbol{l}}^{(j)})^*
    & = \int d\boldsymbol{x} \int d\boldsymbol{x}^\prime
            \phi_{j,\boldsymbol{m}}^\prime \phi_{k,\boldsymbol{n}}^*
            \sum_{\boldsymbol{l}} (\mathcal{B}_{\boldsymbol{l}}^{(j)})^{\prime *} \mathcal{B}_{\boldsymbol{l}}^{(k)}
            \delta(\boldsymbol{x} - \boldsymbol{x}^\prime) \\
    & = \int d\boldsymbol{x} \int d\boldsymbol{x}^\prime
            \phi_{j,\boldsymbol{m}}^\prime \phi_{k,\boldsymbol{n}}^*
            \sum_{\boldsymbol{l}} (\mathcal{B}_{\boldsymbol{l}}^{(j)})^{\prime *} \mathcal{B}_{\boldsymbol{l}}^{(k)}
            \sum_{\boldsymbol{p} \in \mathbb{B}} \phi_{\boldsymbol{p}}^{\prime*} \phi_{\boldsymbol{p}} \\
    & = \sum_{\boldsymbol{p} \in \mathbb{B}, \boldsymbol{l}}
        \int d\boldsymbol{x}
            \phi_{j,\boldsymbol{m}} (\mathcal{B}_{\boldsymbol{l}}^{(j)})^* \phi_{\boldsymbol{p}}^*
        \int d\boldsymbol{x}
            \phi_{k,\boldsymbol{n}}^* \mathcal{B}_{\boldsymbol{l}}^{(k)} \phi_{\boldsymbol{p}}
\end{eqnarray}
Grouping terms back and recognising the definition of projection transformation, one gets the statement of the theorem.
\end{proof}



\section*{References}
%\bibliographystyle{aipsamp}
\bibliography{qsim-long}

\end{document}
