\documentclass[12pt]{iopart}
\usepackage{iopams}
\usepackage{amsthm}
\usepackage{dsfont}
\usepackage[pdftex]{graphicx}
\usepackage{bbm}
\usepackage{braket}
\usepackage{wasysym}


\usepackage{color}
\newcommand{\todo}[1]{\textcolor{red}{[#1]}}

\newcommand{\jvec}{\boldsymbol{j}}
\newcommand{\kvec}{\boldsymbol{k}}
\newcommand{\lvec}{\boldsymbol{l}}
\newcommand{\mvec}{\boldsymbol{m}}
\newcommand{\nvec}{\boldsymbol{n}}
\newcommand{\pvec}{\boldsymbol{p}}
\newcommand{\xvec}{\boldsymbol{x}}
\newcommand{\zvec}{\boldsymbol{z}}
\newcommand{\Zvec}{\boldsymbol{Z}}

%\newcommand{\Tr}{\operatorname{Tr}}
\newcommand{\Trace}[1]{\Tr \left\{ #1 \right\}}

\newcommand{\symprod}[1]{\left\{ #1 \right\}_{\mathrm{sym}}}
\newcommand{\pathavg}[1]{\langle #1 \rangle_{\mathrm{paths}}}
\newcommand{\Real}{\mathrm{Re}}
\newcommand{\Imag}{\mathrm{Im}}

\newcommand{\Psivec}{\boldsymbol{\Psi}}
\newcommand{\Psiop}{\hat{\Psi}}
\newcommand{\Psiopvec}{\hat{\boldsymbol{\Psi}}}

\newcommand{\ecut}{\epsilon_{\mathrm{cut}}}
\newcommand{\fullbasis}{\mathbb{B}}
\newcommand{\restbasis}{\mathbb{M}}

\def\starteqalign#1\end{\eqalign{#1}\end} % magic to wrap \eqalign{} in the environment
\newenvironment{eqn}
	{\begin{eqnarray}\starteqalign}
	{\end{eqnarray}}
\newenvironment{eqns}
	{\begin{eqnarray}}
	{\end{eqnarray}}
\newenvironment{eqn*}
	{\begin{eqnarray*}}
	{\end{eqnarray*}}

\newcommand{\binom}[2]{{#1 \choose #2}}

\newcommand{\eqnref}[1]{\eref{eqn:#1}}
\newcommand{\figref}[1]{Fig.~\ref{fig:#1}}
\newcommand{\thmref}[1]{Theorem~\ref{thm:#1}}
\newcommand{\lmmref}[1]{Lemma~\ref{lmm:#1}}
\newcommand{\defref}[1]{Definition~\ref{def:#1}}

\newtheorem{theorem}{Theorem}
\newtheorem{definition}{Definition}
\newtheorem{lemma}{Lemma}

\newcommand{\swinaffiliation}{Centre for Atom Optics and Ultrafast Spectroscopy, Swinburne University of Technology, Hawthorn, VIC 3122, Australia}

\begin{document}
\title{Wigner representation of BEC}

\author{B.~Opanchuk}
\address{\swinaffiliation}

\date{\today}
\begin{abstract}
Abstract goes here.
\end{abstract}

% Uncomment to set PACs
%\pacs{}

% uncomment if the separate page for title is needed
%\maketitle

% =============================================================================
\section{Introduction}
% =============================================================================

\todo{Introductory words.}

\todo{Problem statement.}

\todo{Review of previous works.}

% =============================================================================
\section{Wirtinger differentiation}
% =============================================================================

In this paper we are using differentiation of complex functions extensively.
Instead of classical definition of the differential which only works for holomorphic functions we use Wirtinger differentiation~\cite{Wirtinger1927}.
One can find thorough description of these rules, for example, in~\cite{Kreutz-Delgado2009}; in this section we will only outline the basics.

\begin{definition}
	For a complex variable $z = x + iy$ and a function $f(z) = u(x, y) + iv(x, y)$ the Wirtinger differential is
	\begin{eqn*}
		\frac{df(z)}{dz}
		= \frac{1}{2} \left(
			\frac{\partial f}{\partial x} - i \frac{\partial f}{\partial y}
		\right).
	\end{eqn*}
\end{definition}

One can easily prove that if $f(z)$ is holomorphic, then the above definition coincides with the classical differential for complex functions.
Wirtinger differential obeys sum, product, quotient, and chain differentiation rules (the former one is applied as if $f(z) \equiv f(z, z^*)$).

We will need some lemmas about integration.
For convenience, we will use the following definition:

\begin{definition}
	For a complex variable $z = x + iy$ the integral
	\begin{equation*}
		\int d^2 z \equiv \int_{-\infty}^{\infty} \int_{-\infty}^{\infty} dx\, dy,
	\end{equation*}
	or, in other words, stands for the two-dimensional integral over the complex plane.
\end{definition}

\begin{lemma}
\label{lmm:c-numbers:fourier-of-moments}
	If $\alpha$ and $\lambda$ are complex variables,
	then for any non-negative integers $r$ and $s$:
	\begin{eqn}
		& \int d^2\alpha\, \alpha^r (\alpha^*)^s \exp(-\lambda \alpha^* + \lambda^* \alpha) \\
		& = \pi^2
			\left( -\frac{\partial}{\partial \lambda^*} \right)^r
			\left( \frac{\partial}{\partial \lambda} \right)^s
			\delta(\Real \lambda) \delta(\Imag \lambda)
	\end{eqn}
\end{lemma}
\begin{proof}
First, changing the variables in the integrals and using known Fourier transform relations, we can prove that for real $x$ and $v$, and non-negative integer $n$
\begin{equation*}
	\int\limits_{-\infty}^{\infty} dv\, v^n \exp(\pm 2 i x v)
	= \pi (\mp i / 2)^n \delta^{(n)}(x),
\end{equation*}
Expanding the $\alpha^r (\alpha^*)^s$ term using binomial theorem and using the above property, one can reach the statement of the lemma.
\end{proof}

A notable special case of \lmmref{c-numbers:fourier-of-moments} is
\begin{equation*}
	\int d^2\alpha \exp(-\lambda \alpha^* + \lambda^* \alpha)
	= \pi^2 \delta(\Real \lambda) \delta(\Imag \lambda).
\end{equation*}

\begin{lemma}
\label{lmm:c-numbers:zero-integrals}
	For any non-negative integers $r$, $s$ and complex $\alpha$:
	\begin{eqn*}
		\int d^2\lambda
			\frac{\partial}{\partial \lambda} \left(
				\exp(-\lambda \alpha^* + \lambda^* \alpha)
				\exp(ixy) x^r y^s
			\right)
		& = 0, \\
		\int d^2\lambda
			\frac{\partial}{\partial \lambda^*}
			\left(
				\exp(-\lambda \alpha^* + \lambda^* \alpha)
				\exp(ixy) x^r y^s
			\right)
		& = 0,
	\end{eqn*}
	where $\lambda = x + iy$.
\end{lemma}
\begin{proof}
We will prove the first equation.
First, note that complex-valued integral of derivative is evaluated as
\begin{eqn*}
	\int d^2\lambda \frac{\partial}{\partial \lambda} f(\lambda, \lambda^*)
	& =	\frac{1}{2} \int\limits_{-\infty}^{\infty} dy \left(
			\left. g(x, y) \right|_{x=-\infty}^{\infty}
		\right) \\
	& - \frac{i}{2} \int\limits_{-\infty}^{\infty} dx \left(
			\left. h(x, y) \right|_{y=-\infty}^{\infty}
		\right),
\end{eqn*}
where we expanded $f = g + ih$.
Thus
\begin{eqn*}
	& \int d^2\lambda
		\frac{\partial}{\partial \lambda} \left(
			\exp(-\lambda \alpha^* + \lambda^* \alpha)
			\exp(ixy) x^r y^s
		\right) \\
	& = \left(
			\frac{1}{2} \exp(2ixv) x^r \int dy \exp(iy(x-2u)) y^s
		\right)_{x = -\infty}^\infty \\
	& - \left(
			\frac{i}{2} \exp(-2ixy) y^s \int dx \exp(ix(y+2v)) x^r
		\right)_{y = -\infty}^\infty \\
	& = \left(
			\frac{1}{2} \exp(2ixv) x^r 2 \pi i^s \delta^{(s)}(x-2u)
		\right)_{x = -\infty}^\infty \\
	& - \left(
			\frac{i}{2} \exp(-2ixy) y^s 2 \pi i^r \delta^{(r)}(y+2v)
		\right)_{y = -\infty}^\infty \\
	& = 0,
\end{eqn*}
because any derivative of delta function is zero on the infinity.
\end{proof}

% =============================================================================
\section{Single-mode Wigner representation}
% =============================================================================

We will need the displacement operator which was first introduced by Weyl~\cite{Weyl1950}:

\begin{definition}
\label{def:sm-wigner:displacement-op}
	If $\hat{a}^\dagger$ and $\hat{a}$ are bosonic creation and annihilation operators, displacement operator $\hat{D}$ is
	\begin{equation*}
		\hat{D}(\lambda, \lambda^*) = \exp(\lambda \hat{a}^\dagger - \lambda^* \hat{a}).
	\end{equation*}
\end{definition}

Using Baker-Hausdorff theorem to split non-commuting operators in the exponent, one can find that
\begin{eqn}
\label{eqn:sm-wigner:displacement-derivatives}
	\frac{\partial}{\partial \lambda} \hat{D}(\lambda, \lambda^*)
	= \hat{D}(\lambda, \lambda^*) (\hat{a}^\dagger + \frac{1}{2} \lambda^*)
	= (\hat{a}^\dagger - \frac{1}{2} \lambda^*) \hat{D}(\lambda, \lambda^*), \\
	-\frac{\partial}{\partial \lambda^*} \hat{D}(\lambda, \lambda^*)
	= \hat{D}(\lambda, \lambda^*) (\hat{a} + \frac{1}{2} \lambda)
	= (\hat{a} - \frac{1}{2} \lambda) \hat{D}(\lambda, \lambda^*).
\end{eqn}

Wigner transformation converts an operator $\hat{A}$ on a Hilbert space to a function $\mathcal{W}[\hat{A}](\alpha, \alpha^*)$ on a phase space.
In terms of the displacement operator Wigner transformation $\mathcal{W}$ and Wigner function $W$ can be defined as

\begin{definition}
\label{def:sm-wigner:w-transformation}
	Wigner transformation of an operator $\hat{A}$:
	\begin{eqn*}
		\mathcal{W}[\hat{A}]
		= \frac{1}{\pi^2} \int d^2 \lambda \exp(-\lambda \alpha^* + \lambda^* \alpha)
			\Trace{ \hat{A} \hat{D}(\lambda, \lambda^*) }.
	\end{eqn*}
	Wigner function is a Wigner transformation of a density matrix:
	\begin{eqn*}
		W(\alpha, \alpha^*) \equiv \mathcal{W}[\hat{\rho}].
	\end{eqn*}
\end{definition}

The Wigner function always exists for any density matrix~\cite{Gardiner2004}, and the correspondence $W \leftrightarrow \hat{\rho}$ is a bijection.
In some cases it is convenient to use the Wigner function in form
\begin{equation}
\label{eqn:sm-wigner:w-function}
	W (\alpha, \alpha^*)
	= \frac{1}{\pi^2} \int d^2 \lambda \exp(-\lambda \alpha^* + \lambda^* \alpha)
		\chi_W (\lambda, \lambda^*),
\end{equation}
where $\chi_W (\lambda, \lambda^*)$ is the characteristic function:
\begin{equation}
	\chi_W (\lambda, \lambda^*)
	= \Trace{ \hat{\rho} \hat{D}(\lambda, \lambda^*) }.
\end{equation}

\begin{lemma}
\label{lmm:sm-wigner:zero-integrals}
	For any integer $m$ and $n$
	\begin{eqn*}
		\int d^2\lambda
			\frac{\partial}{\partial \lambda} & \left(
				\exp(-\lambda \alpha^* + \lambda^* \alpha)
				\left( \frac{\partial}{\partial \lambda} \right)^m
				\left( -\frac{\partial}{\partial \lambda^*} \right)^n
				\hat{D}(\lambda, \lambda^*)
			\right)
		= 0, \\
		\int d^2\lambda
			\frac{\partial}{\partial \lambda^*} & \left(
				\exp(-\lambda \alpha^* + \lambda^* \alpha)
				\left( \frac{\partial}{\partial \lambda} \right)^m
				\left( -\frac{\partial}{\partial \lambda^*} \right)^n
				\hat{D}(\lambda, \lambda^*)
			\right)
		= 0.
	\end{eqn*}
\end{lemma}
\begin{proof}
We will prove the first equation.
Expanding $\lambda = x + iy$ and applying Baker-Hausdorff theorem:
\begin{eqn}
\label{eqn:sm-wigner:BH-displacement}
	\hat{D}(\lambda, \lambda^*)
	= \exp(ixy) \exp(x(\hat{a}^\dagger - \hat{a})) \exp(iy(\hat{a}^\dagger + \hat{a}))
\end{eqn}
Expanding derivatives over $\partial/\partial\lambda$ and $\partial/\partial\lambda^*$ in terms of $\partial/\partial x$ and $\partial/\partial y$, and exponents in~\eqnref{sm-wigner:BH-displacement} as power series:
\begin{eqn}
\fl	\int d^2\lambda
		\frac{\partial}{\partial \lambda} \left(
			\exp(-\lambda \alpha^* + \lambda^* \alpha)
			\left( \frac{\partial}{\partial \lambda} \right)^m
			\left( -\frac{\partial}{\partial \lambda^*} \right)^n
			\hat{D}(\lambda, \lambda^*)
		\right) \\
\fl	= \sum_{r=0}^{\infty} \sum_{s=0}^{\infty} \left(
			\int d^2\lambda
			\frac{\partial}{\partial \lambda} \left(
				\exp(-\lambda \alpha^* + \lambda^* \alpha)
				\exp(ixy) f_{mnrs}(x, y)
			\right)
		\right)
		g_{rs}(\hat{a}, \hat{a}^\dagger) \\
\fl	= 0,
\end{eqn}
where $f_{mnrs}(x, y)$ and $g_{rs}(\hat{a}, \hat{a}^\dagger)$ are some finite-order polynomials,
and we used \lmmref{c-numbers:zero-integrals} to evaluate integrals over $\lambda$.
\end{proof}

\begin{lemma}
	\label{lmm:sm-wigner:moments-from-chi}
	For any integer $r$ and $s$
	\begin{eqn*}
		\langle \symprod{ \hat{a}^r (\hat{a}^\dagger)^s } \rangle
		= \left.
			\left( \frac{\partial}{\partial \lambda} \right)^s
			\left( -\frac{\partial}{\partial \lambda^*} \right)^r
			\chi_W (\lambda, \lambda^*)
		\right|_{\lambda=0}.
	\end{eqn*}
\end{lemma}
\begin{proof}
The displacement operator can be expanded into power series as
\begin{eqn}
	\exp (\lambda \hat{a}^\dagger - \lambda^* \hat{a})
	= \sum_{r,s}
		\frac{(-\lambda^*)^r \lambda^s}{r!s!}
		\symprod{ \hat{a}^r (\hat{a}^\dagger)^s }.
\end{eqn}
Thus
\begin{eqn}
	\chi_W(\lambda, \lambda^*)
	& = \sum_{r,s}
		\frac{(-\lambda^*)^r \lambda^s}{r!s!}
		\Trace{
			\hat{\rho} \symprod{ \hat{a}^r (\hat{a}^\dagger)^s }
		} \\
	& = \sum_{r,s}
		\frac{(-\lambda^*)^r \lambda^s}{r!s!}
		\langle \symprod{ \hat{a}^r (\hat{a}^\dagger)^s } \rangle.
\end{eqn}
Apparently, the application of $(\partial / \partial \lambda)^s$ and $(-\partial / \partial \lambda^*)^r$ will eliminate all lower order moments, and setting $\lambda = 0$ afterwards will eliminate all higher order moments, leaving only $\symprod{ \hat{a}^r (\hat{a}^\dagger)^s }$:
\begin{eqn}
	\left.
		\left( \frac{\partial}{\partial \lambda} \right)^s
		\left( -\frac{\partial}{\partial \lambda^*} \right)^r
		\chi_W (\lambda, \lambda^*)
	\right|_{\lambda=0}
	& = r! s! \frac{1}{r! s!}
		\langle \symprod{ \hat{a}^r (\hat{a}^\dagger)^s } \rangle \\
	& = \langle \symprod{ \hat{a}^r (\hat{a}^\dagger)^s } \rangle.
	\qedhere
\end{eqn}
\end{proof}

Now we have all we need to prove two main theorems for working with Wigner representation.
First one allows us to transform operator equations to ordinary differential equations.
Second one gives a way to calculate any moments of the operators, given the solution of the differential equation.

\begin{theorem}[Operator correspondences]
\label{thm:sm-wigner:correspondences}
\begin{eqn*}
	\mathcal{W} [ \hat{a} \hat{A} ]
		& = \left( \alpha + \frac{1}{2} \frac{\partial}{\partial \alpha^*} \right) \mathcal{W}[\hat{A}],
	\quad
	\mathcal{W} [ \hat{a}^\dagger \hat{A} ]
		= \left( \alpha^* - \frac{1}{2} \frac{\partial}{\partial \alpha} \right) \mathcal{W}[\hat{A}], \\
	\mathcal{W} [ \hat{A} \hat{a} ]
		& = \left( \alpha - \frac{1}{2} \frac{\partial}{\partial \alpha^*} \right) \mathcal{W}[\hat{A}],
	\quad
	\mathcal{W} [ \hat{A} \hat{a}^\dagger ]
		= \left( \alpha^* + \frac{1}{2} \frac{\partial}{\partial \alpha} \right) \mathcal{W}[\hat{A}].
\end{eqn*}
\end{theorem}
\begin{proof}
We will prove the first correspondence.
First, let us transform the trace using~\eqnref{sm-wigner:displacement-derivatives}:
\begin{eqn}
	\Trace{ \hat{a} \hat{A} \hat{D} }
	& = \Trace{ \hat{A} \left(
		-\frac{\partial}{\partial \lambda^*}
		-\frac{1}{2} \lambda
	\right) \hat{D}} \\
	& = \left(
		-\frac{\partial}{\partial \lambda^*}
		-\frac{1}{2} \lambda
	\right) \Trace{ \hat{A} \hat{D}}
\end{eqn}
Now we need to move this additional multiplier outside the integral in the expression for Wigner function:
\begin{eqn}
\fl	\mathcal{W} [ \hat{a} \hat{A} ]
	& = \frac{1}{\pi^2} \int d^2 \lambda \exp(-\lambda \alpha^* + \lambda^* \alpha)
		\Trace{ \hat{a} \hat{A} \hat{D}(\lambda, \lambda^*) } \\
\fl	& = \frac{1}{2} \frac{\partial}{\partial \alpha^*} \mathcal{W} [\hat{A}]
	- \frac{1}{\pi^2} \int d^2 \lambda \exp(-\lambda \alpha^* + \lambda^* \alpha)
		\frac{\partial}{\partial \lambda^*}
		\Trace{ \hat{A} \hat{D}(\lambda, \lambda^*) } \\
\fl	& = \frac{1}{2} \frac{\partial}{\partial \alpha^*} \mathcal{W} [\hat{A}]
	+ \frac{1}{\pi^2} \int d^2 \lambda \left(
		\frac{\partial}{\partial \lambda^*} \exp(-\lambda \alpha^* + \lambda^* \alpha)
	\right)
	\Trace{ \hat{A} \hat{D}(\lambda, \lambda^*) } \\
\fl	& = \left( \alpha + \frac{1}{2} \frac{\partial}{\partial \alpha^*} \right) \mathcal{W} [\hat{A}],
\end{eqn}
where we used~\lmmref{sm-wigner:zero-integrals} to move the partial derivative over $\lambda^*$.
\end{proof}

\begin{theorem}[Calculation of moments]
\label{thm:sm-wigner:moments}
	Expectations of symmetrically ordered operator products are moments of the Wigner function:
	\begin{eqn*}
		\langle \symprod{ \hat{a}^r (\hat{a}^\dagger)^s } \rangle
		= \int d^2\alpha\, \alpha^r (\alpha^*)^s W(\alpha, \alpha^*)
	\end{eqn*}
\end{theorem}
\begin{proof}
By definition of the Wigner function:
\begin{eqn}
\int d^2\alpha\, \alpha^r (\alpha^*)^s W(\alpha, \alpha^*) \\
	= \frac{1}{\pi^2} \Trace{ \hat{\rho}
			\int d^2\alpha\, \alpha^r (\alpha^*)^s
			\int d^2\lambda \exp(-\lambda \alpha^* + \lambda^* \alpha)
			\hat{D}(\lambda, \lambda^*)
		}
\end{eqn}
Integrating by parts and eliminating terms which fit \lmmref{sm-wigner:zero-integrals}:
\begin{eqn}
\fl	= \frac{1}{\pi^2} \Trace{ \hat{\rho}
			\int d^2\alpha \int d^2\lambda
			\exp(-\lambda \alpha^* + \lambda^* \alpha)
			\left( \frac{\partial}{\partial \lambda} \right)^s
			\left( -\frac{\partial}{\partial \lambda^*} \right)^r
			\hat{D} (\lambda, \lambda^*)
		}
\end{eqn}
Evaluating integral over $\alpha$ using \lmmref{c-numbers:fourier-of-moments}:
\begin{eqn}
	& = \int d^2\lambda\,
		\delta (\Real \lambda) \delta (\Imag \lambda)
		\left( \frac{\partial}{\partial \lambda} \right)^s
		\left( -\frac{\partial}{\partial \lambda^*} \right)^r
		\Trace{
			\hat{\rho}
			\hat{D}(\lambda, \lambda^*)
		} \\
	& = \left.
		\left( \frac{\partial}{\partial \lambda} \right)^s
		\left( -\frac{\partial}{\partial \lambda^*} \right)^r
		\chi_W (\lambda, \lambda^*)
	\right|_{\lambda=0}.
\end{eqn}
Now, recognising the final expression as a part of \lmmref{sm-wigner:moments-from-chi},
we immideately get the statement of the theorem.
\end{proof}


% =============================================================================
\section{Functional calculus}
% =============================================================================

Phase-space treatment of multimode problems can be simplified by working with multimode field operators instead of single-mode operators.
It was initially introduced by Graham~\cite{Graham1970,Graham1970a}.
Examples of usage can be found in~\cite{Steel1998,Norrie2006a}.
Detailed description of functional calculus is given in~\cite{Dalton2011}.
Here we only provide some important results which are going to be used later on in this paper.

First we must introduce some operations on functions, which will replace common differentials and integrals used in single and multi-mode cases and help encapsulate basis and mode populations inside wave functions and field operators.
In order to do that, we define an orthonormal basis $\phi_{\nvec}$, where $\nvec$ is a state vector with $D$ elements.
Orthonormality and completeness conditions for basis functions are, respectively,
\begin{eqns}
	\int\limits_A \phi_{\nvec}^*(\xvec) \phi_{\mvec}(\xvec) d\xvec = \delta_{\nvec\mvec}, \\
	\sum_{\nvec} \phi_{\nvec}^*(\xvec) \phi_{\nvec}(\xvec^\prime) = \delta(\xvec^\prime - \xvec),
\end{eqns}
where the exact nature of integration area $A$ depends on the nature of the basis set (for example, $A$ is the whole space for harmonic oscillator modes, or a box for plane waves).
Hereinafter we assume that the integration $\int d\xvec$ is always performed over $A$.

Given the basis, we can define the composition transformation
\begin{eqn}
	\mathcal{C} :: \mathbb{C}^{|L|} \rightarrow (\mathbb{R}^D \rightarrow \mathbb{C})_L \\
	\mathcal{C}(\balpha) = \sum_{\nvec \in L} \phi_{\nvec} \alpha_{\nvec},
\end{eqn}
where $L$ is some subset of the basis, and $|L|$ is its cardinality.
Its result is a complex-valued function, which consists only of modes from $L$.
Decomposition transformation is, in turn
\begin{eqn}
	\mathcal{C}^{-1} :: (\mathbb{R}^D \rightarrow \mathbb{C})_L \rightarrow \mathbb{C}^{|L|} \\
	\mathcal{C}^{-1}[f]_m = \int d\xvec \phi_m^*(\xvec) f(\xvec),\,m \in L.
\end{eqn}
Any function can be projected to subset $L$ using the projection transformation
\begin{eqn}
\label{eqn:func-calculus:projector}
	\mathcal{P} ::
	(\mathbb{R}^D \rightarrow \mathbb{C}) \rightarrow (\mathbb{R}^D \rightarrow \mathbb{C})_L \\
	\mathcal{P}[f](\xvec)
	= \sum_{\nvec \in L} \phi_{\nvec} (\xvec) \int
		d\xvec^\prime\, \phi_{\nvec}^*(\xvec^\prime) f(\xvec^\prime),
\end{eqn}
If $L$ is the whole basis, then, apparently, $\mathcal{P} \equiv \mathds{1}$.

If not explicitly stated otherwise, all functions of $\xvec$ are assumed to belong to basis subset $L$ (restricted basis), that is $\mathcal{P}[f] \equiv f$, or $f$ has type $(\mathbb{R}^D \rightarrow \mathbb{C})_L$.
Note that the result of any non-linear transformation of a function is not guaranteed to belong to $L$ and requires explicit projection to be used with other restricted functions.
This applies to the delta function which depends on coordinates.
To avoid confusion with delta function of real or complex number, the restricted delta function is written as $\delta_P$ and defined as
\begin{eqn}
\label{eqn:func-calculus:restricted-delta}
	\delta_P(\xvec^\prime - \xvec)
	= \sum_{\nvec \in L} \phi_{\nvec}^* (\xvec^\prime) \phi_{\nvec} (\xvec).
\end{eqn}
Apparently, restricted delta belongs to required type $(\mathbb{R}^D \rightarrow \mathbb{C})_L$.
Note that $\delta_P$ is a Hermitian function: $\delta_P^*(\xvec^\prime - \xvec) = \delta_P(\xvec - \xvec^\prime)$.

Restricted delta function can be used to rewrite equation for $\mathcal{P}$:
\begin{eqn}
	\mathcal{P}[f](\xvec) = \int d\xvec^\prime \delta_P(\xvec^\prime - \xvec) f(\xvec^\prime).
\end{eqn}
The conjugate of $\mathcal{P}$ is thus defined as
\begin{eqn}
	(\mathcal{P}[f])^*(\xvec)
	= \int d\xvec^\prime \delta_P^*(\xvec^\prime - \xvec) f^*(\xvec^\prime)
	= \mathcal{P}^* [f^*](\xvec).
\end{eqn}

Let $\mathcal{F}[f] :: (\mathbb{R}^D \rightarrow \mathbb{C})_L \rightarrow (\mathbb{R}^D \rightarrow \mathbb{C})$ be some transformation (note that the result is not guaranteed to belong to restricted basis).
Because values of types $(\mathbb{R}^D \rightarrow \mathbb{C})_L$ and $\mathbb{C}^{|L|}$ are interchangeable, $\mathcal{F}$ can be alternatively treated as a function of a vector of complex numbers:
\begin{eqn}
	\mathcal{F} :: \mathbb{C}^{|L|} \rightarrow \mathbb{C}^\infty \\
	\mathcal{F}(\balpha_f) \equiv \mathcal{C}^{-1}[\mathcal{F}[\mathcal{C}(\balpha_f)]],
\end{eqn}
where the subscript $f$ after the vector means that this is the vector which specifies function $f$.

Functional derivative is defined as
\begin{eqn}
\label{eqn:func-aux:func-diff}
	\frac{\delta}{\delta f(\xvec^\prime)} ::
	\left(
		(\mathbb{R}^D \rightarrow \mathbb{C})_L
		\rightarrow
		(\mathbb{R}^D \rightarrow \mathbb{C})
	\right)
	\rightarrow
	\left(
		(\mathbb{R}^D \rightarrow \mathbb{C})_L
		\rightarrow
		(\mathbb{R}^D \rightarrow \mathbb{R}^D \rightarrow \mathbb{C})
	\right) \\
	\frac{\delta \mathcal{F}[f]}{\delta f(\xvec^\prime)}
	= \sum_{\nvec \in L} \phi_{\nvec}^* (\xvec^\prime)
		\frac{\partial \mathcal{F}(\balpha_f)}{\partial \alpha_{f,\nvec}}.
\end{eqn}
Note that the transformation being returned differs from the one which was taken: the result of new transformation is a function depending on two variables from $\mathbb{R}^D$, not one.
The second variable comes from the function we are differentiating by.

Functional derivative definition behaves in many ways similar to common derivative.

\begin{lemma}
	Functional differentiation~\eqnref{func-aux:func-diff} obeys sum, product, quotient, and chain differentiation rules.
\end{lemma}
\begin{proof}
!!! Sum, product and quotient are more or less obvious; but should we prove chain differentiation?
\end{proof}

\begin{lemma}
	If $g(z)$ is a function that can be expanded into power series,
	and functional $\mathcal{F}[f] \equiv g(f)$, then
	\begin{eqn*}
		\frac{\delta \mathcal{F}[f]}{\delta f(\xvec^\prime)} (\xvec)
		= \delta_P(\xvec^\prime - \xvec)
			\left. \frac{\partial g(z)}{\partial z} \right|_{z = f(\xvec)}
	\end{eqn*}
\end{lemma}
\begin{proof}
We will consider $g(z) = z^k$ case first, which will straightforwardly lead to the statement of the lemma.
For $k = 1$, obviously,
\begin{eqn}
	\frac{\delta f}{\delta f(\xvec^\prime)} (\xvec)
	= \delta_P(\xvec^\prime - \xvec)
\end{eqn}
Then for other values of $k$:
\begin{eqn*}
	\frac{\delta \mathcal{F}[f]}{\delta f(\xvec^\prime)} (\xvec)
	& = \frac{\delta f^k}{\delta f(\xvec^\prime)} (\xvec)
	= \sum_{\nvec \in L} \phi_{\nvec}^{\prime*}
		\frac{\partial f^k}{\partial \alpha_{\nvec}} \\
	& = \sum_{\nvec \in L} \phi_{\nvec}^{\prime*}
		\frac{\partial f^k}{\partial f}
		\frac{\partial f}{\partial \alpha_{\nvec}}
	= k f^{k-1}
		\sum_{\nvec \in L} \phi_{\nvec}^{\prime*}
		\frac{\partial f}{\partial \alpha_{\nvec}} \\
	& = k \delta_P(\xvec^\prime - \xvec) f^{k-1}(\xvec)
	= \delta_P(\xvec^\prime - \xvec)
		\left. \frac{\partial z^k}{\partial z} \right|_{z = f(\xvec)}.
	\qedhere
\end{eqn*}
\end{proof}

\begin{lemma}
	If $g(z)$ can be expanded into series of $z^n (z^*)^m$, and functional $\mathcal{F}[f, f^*] \equiv g(f, f^*)$, then $\delta \mathcal{F} / \delta f^\prime$ and $\delta \mathcal{F} / \delta f^{\prime*}$ can be treated as partial differentiation of the functional of two independent variables $f$ and $f^*$.
	In other words:
	\begin{eqn*}
		\frac{\delta \mathcal{F}}{\delta f^\prime}
		= \delta_P(\xvec^\prime - \xvec) \left.
			\frac{\partial g(z, z^*)}{\partial z}
		\right|_{z=f(x)},
		\quad
		\frac{\delta \mathcal{F}}{\delta f^{\prime*}}
		= \delta_P^*(\xvec^\prime - \xvec) \left.
			\frac{\partial g(z, z^*)}{\partial z^*}
		\right|_{z=f^*(x)}
	\end{eqn*}
\end{lemma}
\begin{proof}
Proof is similar to \lmmref{c-numbers:independent-vars}.
\end{proof}

Functional integration is defined as
\begin{eqn}
	\int \delta f ::
	(\mathbb{R}^D \rightarrow \mathbb{C})_L	\rightarrow \mathbb{C} \\
	\int \delta^2 f \mathcal{F}[f]
	= \int d^2\balpha_f \mathcal{F}(\balpha_f)
	= \int \ldots \int d^2\alpha_{f,1} \ldots d^2\alpha_{f,N} \mathcal{F}(\balpha_f).
\end{eqn}
If the basis contains infinite number of modes, the integral is treated as a limit $N \rightarrow \infty$.

We will need delta functional:
\begin{eqn}
	\Delta[\Lambda]
	\equiv \prod_{\nvec \in L} \delta(\Real \lambda_{\nvec}) \delta(\Imag \lambda_{\nvec}).
\end{eqn}
Note that it is really a functional and not a transformation:
it depends only on $\blambda$, but not on coordinate.
It has the same property as common delta function:
\begin{eqn}
	\int \delta^2 \Lambda \mathcal{F}[\Lambda] \Delta[\Lambda]
	= \int \ldots \int d^2\lambda_1 \ldots d^2\lambda_N \mathcal{F}(\blambda)
		\prod_{\nvec \in L} \delta(\Real \lambda_{\nvec}) \delta(\Imag \lambda_{\nvec})
	= \left. \mathcal{F}(\blambda) \right|_{\forall \nvec\, \lambda_{\nvec} = 0}
	= \left. \mathcal{F}[\Lambda] \right|_{\Lambda \equiv 0}
\end{eqn}

\begin{lemma}[Functional extension of \lmmref{c-numbers:fourier-of-moments}]
\label{lmm:func-calculus:fourier-of-moments}
	If $\Psi$ and $\Lambda$ are complex-valued functions of coordinate $\xvec$,
	then for any non-negative integers $r$ and $s$:
	\begin{eqn}
		\int \delta^2\Psi\, \Psi^r (\Psi^*)^s \exp
			\int d\xvec \left( -\Lambda \Psi^* + \Lambda^* \Psi \right)
		= \pi^{2N}
			\left( -\frac{\delta}{\delta \Lambda^*} \right)^r
			\left( \frac{\delta}{\delta \Lambda} \right)^s
			\Delta[\Lambda]
	\end{eqn}
\end{lemma}
\begin{proof}
\begin{eqn}
	& \int \delta^2\Psi\, \Psi^r (\Psi^*)^s \exp
		\int d\xvec \left( -\Lambda \Psi^* + \Lambda^* \Psi \right) \\
	& = \int \ldots \int d^2\alpha_1 \ldots d^2\alpha_N
		\left( \sum_{\nvec \in L} \phi_{\nvec} \alpha_{\nvec} \right)^r
		\left( \sum_{\nvec \in L} \phi^*_{\nvec} \alpha_{\nvec}^* \right)^s
		\prod_{\nvec \in L} \exp(-\lambda_{\nvec} \alpha_{\nvec}^* + \lambda_{\nvec}^* \alpha_{\nvec}) \\
	& = \int \ldots \int d^2\alpha_1 \ldots d^2\alpha_N
		\sum_{u_1 + \ldots + u_N = r} \binom{r}{u_1, \ldots, u_N}
			\prod_{\nvec \in L} \phi_{\nvec}^{u_{\nvec}} \alpha_{\nvec}^{u_{\nvec}} \\
	&	\sum_{v_1 + \ldots + v_N = s} \binom{s}{v_1, \ldots, v_N}
			\prod_{\nvec \in L} (\phi_{\nvec}^*)^{v_{\nvec}} (\alpha_{\nvec}^*)^{v_{\nvec}}
		\prod_{\nvec \in L} \exp(-\lambda_{\nvec} \alpha_{\nvec}^* + \lambda_{\nvec}^* \alpha_{\nvec}) \\
	& = \sum_{u_1 + \ldots + u_N = r}
		\sum_{v_1 + \ldots + v_N = s}
		\binom{r}{u_1, \ldots, u_N}
		\binom{s}{v_1, \ldots, v_N}
		\prod_{\nvec \in L}
			\phi_{\nvec}^{u_{\nvec}} (\phi_{\nvec}^*){v_{\nvec}}
			\int d^2\alpha_{\nvec}
				\alpha_{\nvec}^{u_{\nvec}}
				(\alpha_{\nvec}^*)^{v_{\nvec}}
				\exp(-\lambda_{\nvec} \alpha_{\nvec}^* + \lambda_{\nvec}^* \alpha_{\nvec}) \\
	& = \sum_{u_1 + \ldots + u_N = r}
		\sum_{v_1 + \ldots + v_N = s}
		\binom{r}{u_1, \ldots, u_N}
		\binom{s}{v_1, \ldots, v_N}
		\pi^{2N}
		\prod_{\nvec \in L}
			\phi_{\nvec}^{u_{\nvec}} (\phi_{\nvec}^*)^{v_{\nvec}}
			\left( -\frac{\partial}{\partial \lambda_{\nvec}^*} \right)^{u_{\nvec}}
			\left( \frac{\partial}{\partial \lambda_{\nvec}} \right)^{v_{\nvec}}
			\delta(\Real \lambda_{\nvec}) \delta(\Imag \lambda_{\nvec}) \\
	& = \pi^{2N}
		(-\sum_{\nvec \in L} \phi_{\nvec} \frac{\partial}{\partial \lambda_{\nvec}^*})^r
		(\sum_{\nvec \in L} \phi_{\nvec}^* \frac{\partial}{\partial \lambda_{\nvec}})^s
		\prod_{\nvec \in L} \delta(\Real \lambda_{\nvec}) \delta(\Imag \lambda_{\nvec}) \\
	& = \pi^{2N}
		\left( -\frac{\delta}{\delta \Lambda^*} \right)^r
		\left( \frac{\delta}{\delta \Lambda} \right)^s
		\Delta[\Lambda]
	\qedhere
\end{eqn}
\end{proof}

In order to perform transformations of master equations in the future,
we will need a lemma, which justifies certain operation with Laplacian
(which is a part of kinetic term in Hamiltonian).

\begin{lemma}
\label{lmm:func-calculus:move-laplacian}
	If $\forall n \in L, \xvec \in \partial A$ $\phi_n(\xvec) = 0$, then
	\begin{eqn*}
		\int\limits_A d\xvec \left(
			\nabla^2 \frac{\delta}{\delta \Psi}
		\right) \Psi \mathcal{F}[\Psi, \Psi^*]
		= \int\limits_A d\xvec \frac{\delta}{\delta \Psi}
		( \nabla^2 \Psi ) \mathcal{F}[\Psi, \Psi^*]
	\end{eqn*}
\end{lemma}
\begin{proof}
Integration limits play an important role in this proof,
so we will write them explicitly.
\begin{eqn}
	\int\limits_A d\xvec \left(
		\nabla^2 \frac{\delta}{\delta \Psi}
	\right) \Psi
	= \sum_{\nvec, \mvec} \left(
			\int\limits_A d\xvec ( \nabla^2 \phi_{\nvec}^* ) \phi_{\mvec}
		\right)
		\frac{\partial}{\partial \alpha_{\nvec}} \alpha_{\mvec} \mathcal{F}(\balpha)
	= (*)
\end{eqn}
Using Green's first identity and the fact that eigenfunctions are equal to zero at the boundary of $A$:
\begin{eqn}
	\int\limits_A d\xvec ( \nabla^2 \phi_{\nvec}^* ) \phi_{\mvec}
	& = \oint\limits_{\partial A} \phi_{\mvec} (\nabla \phi_{\nvec}^* \cdot \boldsymbol{v}) dS
	- \int\limits_A d\xvec ( \nabla \phi_{\nvec}^* ) ( \nabla \phi_{\mvec} ) \\
	& = 0 - \int\limits_A d\xvec ( \nabla \phi_{\nvec}^* ) ( \nabla \phi_{\mvec} ) \\
	& = \oint\limits_{\partial A} \phi_{\nvec}^* (\nabla \phi_{\mvec} \cdot \boldsymbol{v}) dS
	- \int\limits_A d\xvec ( \nabla \phi_{\nvec}^* ) ( \nabla \phi_{\mvec} ) \\
	& = \int\limits_A d\xvec \phi_{\nvec}^* ( \nabla^2 \phi_{\mvec} ),
\end{eqn}
where $\boldsymbol{v}$ is the outward pointing unit normal of surface element $dS$.
Thus
\begin{eqn}
	= \sum_{\nvec, \mvec} \left(
			\int\limits_A d\xvec \phi_{\nvec}^* ( \nabla^2 \phi_{\mvec} )
		\right)
		\frac{\partial}{\partial \alpha_{\nvec}} \alpha_{\mvec} \mathcal{F}(\balpha)
	= \int\limits_A d\xvec \frac{\delta}{\delta \Psi}
		( \nabla^2 \Psi ) \mathcal{F}[\Psi, \Psi^*].
	\qedhere
\end{eqn}
\end{proof}

Note that this lemma imposes additional requirement for basis functions,
but in practical applications it is always satisfied.
For example, in plane wave basis eigenfunctions are equal to zero at the border of the bounding box,
and in harmonic oscillator basis they are equal to zero on the infinity
(which can be considered the boundary of their integration area).
Hereinafter we will assume that this condition is true for any basis we work with.

% =============================================================================
\section{Field operators and restricted basis}
% =============================================================================
\label{sec:func-operators}

We define an orthonormal basis $\fullbasis$ consisting of $\phi_{\nvec}(\xvec)$, where $\nvec$ is a state vector, and $\xvec \in \mathbb{R}^D$.
Orthonormality and completeness conditions for basis functions are, respectively,
\begin{eqns}
	\int\limits_A \phi_{\nvec}^*(\xvec) \phi_{\mvec}(\xvec) d\xvec = \delta_{\nvec\mvec}, \\
	\sum_{\nvec} \phi_{\nvec}^*(\xvec) \phi_{\nvec}(\xvec^\prime) = \delta(\xvec^\prime - \xvec),
\end{eqns}
where the exact nature of integration area $A$ depends on the basis set (for example, $A$ is the whole space for harmonic oscillator modes, or a box for plane waves).
Hereinafter we assume that the integration $\int d\xvec$ is always performed over $A$.

Standard bosonic field operators from~\eqnref{master-eqn:commutators} which project coordinates to the full Hilbert space $\mathbb{H}$ can be decomposed as
\begin{eqn}
    \Psiop_j \in (\mathbb{R}^D \rightarrow \mathbb{H}): \quad
	\Psiop_j(\xvec) = \sum_{\nvec \in \fullbasis} \phi_{\nvec}(\xvec) \hat{a}_{j,\nvec},
\end{eqn}
where single mode operators $\hat{a}_{j,\nvec}$ obey bosonic commutation relations, the pair $j,\nvec$ serving as a mode identifier.
The energy cutoff mentioned in the previous section will result in operating with some fixed subset of the basis.
Let $\restbasis$ be this subset with cardinality $|\restbasis|$.
Restricted field operators contain only modes from the subset $\restbasis$:
\begin{eqn}
    \Psiop_j \in (\mathbb{R}^D \rightarrow \mathbb{H}_{\restbasis}): \quad
	\Psiop_j(\xvec)	= \sum_{\nvec \in \restbasis} \phi_{\nvec} (\xvec) \hat{a}_{j,\nvec}.
\end{eqn}
In this paper we will use them without explicit indices (as opposed to full basis field operators) in order to avoid clutter.
Restricted field operators have type $\mathbb{FH}_{\restbasis} \equiv (\mathbb{R}^D \rightarrow \mathbb{H}_{\restbasis})$, where $\mathbb{H}_{\restbasis}$ is the Hilbert space of the restricted subset of modes.

Because of the restricted nature of the operator, commutation relations~\eqnref{master-eqn:commutators} no longer apply.
The following ones should be used instead:
\begin{eqn}
\label{eqn:func-operators:restricted-commutators}
	\left[ \Psiop_j(\xvec), \Psiop_k(\xvec^\prime) \right]
	& = \left[ \Psiop_j^\dagger(\xvec), \Psiop_k^\dagger(\xvec^\prime) \right] = 0, \\
	\left[ \Psiop_j(\xvec), \Psiop_k^\dagger(\xvec^\prime) \right]
	& = \delta_{jk} \delta_{\restbasis}(\xvec^\prime, \xvec),
\end{eqn}
where $\delta_{\restbasis}$ is the restricted delta-function from \defref{func-calculus:restricted-delta}.

% =============================================================================
\section{Functional Wigner representation}
% =============================================================================

The single-mode Wigner transformation of the operator $\hat{A}$ is defined as
\begin{eqn}
	\mathcal{W}_{\mathrm{sm}}[\hat{A}]
	= \frac{1}{\pi^2} \int d^2 \lambda \exp(-\lambda \alpha^* + \lambda^* \alpha)
		\Trace{ \hat{A} \hat{D}(\lambda, \lambda^*) },
\end{eqn}
where the displacement operator $\hat{D}(\lambda, \lambda^*) = \exp(\lambda \hat{a}^\dagger - \lambda^* \hat{a})$ was first introduced by Weyl~\cite{Weyl1950}.
The detailed description of the Wigner function $W(\alpha, \alpha^*) \equiv \mathcal{W}_{\mathrm{sm}}[\hat{\rho}]$ can be found in~\cite{Gardiner2004}.
In this section we will extend this definition to the multimode case.

The important part of the definition is the functional analogue of the displacement operator.
\todo{Need to explain why we write $F[\Lambda, \Lambda^*]$ and not $F[\Lambda]$?}

\begin{definition}
    Functional displacement operator
	\begin{eqn*}
		\hat{D}_j :: \mathbb{F}_{\restbasis_j} \rightarrow \mathbb{H}_{\restbasis_j} \\
		\hat{D}_j[\Lambda, \Lambda^*] = \exp \int d\xvec \left(
			\Lambda \Psiop_j^\dagger - \Lambda^* \Psiop_j
		\right).
	\end{eqn*}
	It is also convenient to define the displacement functional as
	\begin{eqn*}
		D :: \mathbb{F}_{\restbasis_j} \rightarrow \mathbb{F}_{\restbasis_j} \rightarrow \mathbb{C} \\
		D[\Lambda, \Lambda^*, \Psi, \Psi^*] = \exp \int d\xvec \left(
			-\Lambda \Psi^* + \Lambda^* \Psi
		\right).
	\end{eqn*}
\end{definition}

It can be shown that the functional displacement operator has properties similar to its single-mode equivalent.

\begin{lemma}
\label{lmm:func-wigner:displacement-derivatives}
	\begin{eqn*}
		\frac{\delta}{\delta \Lambda^\prime} \hat{D}_j[\Lambda, \Lambda^*]
		= \hat{D}_j[\Lambda, \Lambda^*] (\Psiop_j^{\prime\dagger} + \frac{1}{2} \Lambda^{\prime*})
		= (\Psiop_j^{\prime\dagger} - \frac{1}{2} \Lambda^{\prime*}) \hat{D}_j[\Lambda, \Lambda^*], \\
		-\frac{\delta}{\delta \Lambda^{\prime*}} \hat{D}_j[\Lambda, \Lambda^*]
		= \hat{D}_j(\Lambda, \Lambda^*) (\Psiop_j^\prime + \frac{1}{2} \Lambda^\prime)
		= (\Psiop_j^\prime - \frac{1}{2} \Lambda^\prime) \hat{D}_j[\Lambda, \Lambda^*].
	\end{eqn*}
\end{lemma}
\begin{proof}
Proved using Baker-Hausdorff theorem and evaluating integrals.
\end{proof}

It is convenient to first define a general Wigner transformation.

\begin{definition}
\label{def:func-wigner:w-transformation}
	Multi-component functional Wigner transformation $\mathcal{W}$ is defined as
	\todo{Is the return value real or complex?}
	\begin{eqn*}
		\mathcal{W} :: \left( \mathbb{R}^D \rightarrow \prod_{j=1}^C \mathbb{H}_{\restbasis_j} \right)
			\rightarrow \prod_{j=1}^C \mathbb{F}_{\restbasis_j}
			\rightarrow \mathbb{R} \\
		\mathcal{W}[\hat{A}]
		= \frac{1}{\pi^{2 \sum|\restbasis_j|}} \int \delta^2 \bLambda
			\left( \prod_{j=1}^C D[\Lambda_j, \Lambda_j^*, \Psi_j, \Psi_j^*] \right)
			\Trace{ \hat{A} \prod_{j=1}^C \hat{D}_j[\Lambda_j, \Lambda_j^*] },
	\end{eqn*}
	where $\Lambda_j \in \mathbb{F}_{\restbasis_j}$, and $\int \delta^2 \bLambda \equiv \int \delta^2 \Lambda_1 \ldots \delta^2 \Lambda_C$.
	It transforms a coordinate-dependent operator $\hat{A}$ on a restricted subset of a Hilbert space to a functional $(\mathcal{W}[\hat{A}])[\bPsi, \bPsi^*]$.
	\todo{Add expression for Weyl transformation?}
\end{definition}

Wigner functional is a special case of Wigner transformation.

\begin{definition}
\label{def:func-wigner:w-functional}
	The Wigner functional is
	\begin{eqn*}
		W :: \prod_{j=1}^C \mathbb{F}_{\restbasis_j} \rightarrow \mathbb{R} \\
		W [\bPsi, \bPsi^*]
		\equiv \mathcal{W}[\hat{\rho}]
		= \frac{1}{\pi^{2 \sum|\restbasis_j|}} \int \delta^2 \bLambda
			\left( \prod_{j=1}^C D[\Lambda_j, \Lambda_j^*, \Psi_j, \Psi_j^*] \right) \chi_W,
	\end{eqn*}
	where $\chi_W [\bLambda, \bLambda^*]$ is the characteristic functional
	\begin{eqn*}
		\chi_W [\bLambda, \bLambda^*]
		= \Trace{ \hat{\rho} \prod_{j=1}^C \hat{D}_j[\Lambda_j, \Lambda_j^*] }.
	\end{eqn*}
\end{definition}

The Wigner functional has two important properties analogous to the single-mode case.
First one is used to successively transform operator products.
In order to prove it, we will need an auxiliary lemma.

\begin{lemma}
\label{lmm:func-wigner:zero-integrals}
	For $\Lambda \in \mathbb{F}_j$, $\Psi \in \mathbb{F}_j$:
	\begin{eqn*}
		\int \delta^2\Lambda
			\frac{\delta}{\delta \Lambda^\prime} \left(
				D[\Lambda, \Lambda^*, \Psi_j, \Psi_j^*]
				\left( \frac{\delta}{\delta \Lambda^\prime} \right)^r
				\left( -\frac{\delta}{\delta \Lambda^{\prime*}} \right)^s
				\hat{D}_j[\Lambda, \Lambda^*]
			\right)
		= 0, \\
		\int \delta^2\Lambda
			\frac{\delta}{\delta \Lambda^{\prime*}}
			\left(
				D[\Lambda, \Lambda^*, \Psi, \Psi^*]
				\left( \frac{\delta}{\delta \Lambda^\prime} \right)^r
				\left( -\frac{\delta}{\delta \Lambda^{\prime*}} \right)^s
				\hat{D}_j[\Lambda, \Lambda^*]
			\right)
		= 0.
	\end{eqn*}
\end{lemma}
\begin{proof}
Displacement operator and displacement functional can be represented as functions of vectors:
\begin{eqn}
	\hat{D}[\Lambda, \Lambda^*]
	= \prod_{\nvec \in \restbasis_j} \exp \left(
		\lambda_{\nvec} \hat{a}_{j,\nvec}^\dagger - \lambda_{\nvec}^* \hat{a}_{j,\nvec}
	\right),
\end{eqn}
\begin{eqn}
	D[\Lambda, \Lambda^*, \Psi, \Psi^*]
	= \prod_{\nvec \in \restbasis_j} \exp
		(-\lambda_{\nvec} \alpha_{j,\nvec}^* + \lambda_{\nvec}^* \alpha_{j,\nvec}),
\end{eqn}
The proof consists of substituting these in the equations from the statement and applying \lmmref{c-numbers:zero-integrals}.
\end{proof}

\begin{theorem}[Functional correspondences]
\label{thm:func-wigner:correspondences}
    If $\mathcal{W} [ \hat{A} ] \equiv (\mathcal{W} [ \hat{A} ]) [\bPsi, \bPsi^*]$, then
	\begin{eqn*}
		\mathcal{W} [ \Psiop_j \hat{A} ]
			& = \left( \Psi_j + \frac{1}{2} \frac{\delta}{\delta \Psi_j^*} \right) \mathcal{W}[\hat{A}],
		\quad
		\mathcal{W} [ \Psiop_j^\dagger \hat{A} ]
			= \left( \Psi_j^* - \frac{1}{2} \frac{\delta}{\delta \Psi_j} \right) \mathcal{W}[\hat{A}], \\
		\mathcal{W} [ \hat{A} \Psiop_j ]
			& = \left( \Psi_j - \frac{1}{2} \frac{\delta}{\delta \Psi_j^*} \right) \mathcal{W}[\hat{A}],
		\quad
		\mathcal{W} [ \hat{A} \Psiop_j^\dagger ]
			= \left( \Psi_j^* + \frac{1}{2} \frac{\delta}{\partial \Psi_j} \right) \mathcal{W}[\hat{A}].
	\end{eqn*}
\end{theorem}
\begin{proof}
The proof uses \lmmref{func-wigner:displacement-derivatives} to transform the $\hat{A} \prod_j \hat{D}_j$ product inside the trace, and \lmmref{func-wigner:zero-integrals} to integrate by parts, effectively moving the differentials to intended places.
\end{proof}

The second property complements the first one, providing the way to obtain expectations of operator products given the Wigner function.
Again, it requires a supplementary lemma.

\begin{lemma}
\label{lmm:func-wigner:moments-from-chi}
    For any non-negative integer $r$ and $s$:
	\begin{eqn*}
		\langle \symprod{ (\Psiop_j^\prime)^r (\Psiop_j^{\prime\dagger})^s } \rangle
		= \left.
			\left( \frac{\delta}{\delta \Lambda_j^\prime} \right)^s
			\left( -\frac{\delta}{\delta \Lambda_j^{\prime*}} \right)^r
			\chi_W [\bLambda, \bLambda^*]
		\right|_{\bLambda \equiv 0}.
	\end{eqn*}
\end{lemma}
\begin{proof}
The factor corresponding to $j$-th component in the displacement operator can be expanded as
\begin{eqn}
	\exp (\Lambda_j \Psiop_j^\dagger - \Lambda_j^* \Psiop_j)
	= \sum_{r,s}
		\frac{
			\symprod{
				\left( \int d\xvec \Lambda_j \Psiop_j^\dagger \right)^r
				\left( -\int d\xvec \Lambda_j^* \Psiop_j \right)^s
			}
		}
		{r!s!}.
\end{eqn}
We can swap functional derivative with both integration and multiplication by independent function, so:
\begin{eqn}
	\frac{\delta}{\delta \Lambda_j^\prime} \left( \int d\xvec \Lambda_j \Psiop_j^\dagger \right)^r
	= r \Psiop_j^{\prime\dagger} \left( \int d\xvec \Lambda_j \Psiop_j^\dagger \right)^{r-1},
\end{eqn}
and multiple application of the differential gives us
\begin{eqn}
	\left( \frac{\delta}{\delta \Lambda_j^\prime} \right)^r
	\left( \int d\xvec \Lambda_j \Psiop_j^\dagger \right)^r
	= r! ( \Psiop_j^{\prime\dagger} )^r.
\end{eqn}
Similarly for the other differential:
\begin{eqn}
	\left( -\frac{\delta}{\delta \Lambda_j^{\prime*}} \right)^s
	\left( -\int d\xvec \Lambda_j \Psiop_j^\dagger \right)^s
	= s! ( \Psiop_j^{\prime\dagger} )^s.
\end{eqn}

Thus, differentiation will eliminate all lower order terms in the expansion, and all higher order terms will be eliminated by setting $\Lambda_j \equiv 0$ for every $j$, leaving only one operator product with required order.
\end{proof}

\begin{theorem}[Calculation of expectations]
\label{thm:func-wigner:moments}
	For any non-negative integer $r_j$, $s_j$
	\begin{eqn*}
		\langle \symprod{ \prod_{j=1}^C \Psiop_j^{r_j} (\Psiop_j^\dagger)^{s_j} } \rangle
		= \int \delta^2 \bPsi
			\left( \prod_{j=1}^C \Psi_j^{r_j} (\Psi_j^*)^{s_j} \right) W[\bPsi, \bPsi^*].
	\end{eqn*}
\end{theorem}
\begin{proof}
By definition of the Wigner functional:
\begin{eqn}
\fl	\int \delta^2\bPsi \left( \prod_{j=1}^C \Psi_j^{r_j} (\Psi_j^*)^{s_j} \right) W[\bPsi, \bPsi^*] \\
\fl	= \frac{1}{\pi^{2\sum|\restbasis_j|}} \Trace{ \hat{\rho}
		\prod_{j=1}^C
			\int \delta^2 \Lambda_j \left(
				\int \delta^2 \Psi_j\, \Psi_j^{r_j} (\Psi_j^*)^{s_j}
				D[\Lambda_j, \Lambda_j^*, \Psi_j, \Psi_j^*]
			\right)
			\hat{D}_j[\Lambda_j, \Lambda_j^*]
	}.
\end{eqn}
Evaluating the integral over $\Psi_j$ using \lmmref{func-calculus:fourier-of-moments}:
\begin{eqn}
	= \Trace{ \hat{\rho}
		\prod_{j=1}^C
			\int \delta^2 \Lambda_j \left(
				\left( -\frac{\delta}{\delta \Lambda_j^*} \right)^{r_j}
				\left( \frac{\delta}{\delta \Lambda_j} \right)^{s_j}
				\Delta_{\restbasis_j}[\Lambda_j]
			\right)
			\hat{D}_j[\Lambda_j, \Lambda_j^*]
	}.
\end{eqn}
Integrating by parts for each component in turn and eliminating terms which fit \lmmref{func-wigner:zero-integral}:
\begin{eqn}
	& = \Trace{ \hat{\rho}
		\prod_{j=1}^C \int \delta^2 \Lambda_j
			\Delta_{\restbasis_j}[\Lambda_j]
			\left( -\frac{\delta}{\delta \Lambda_j^*} \right)^{r_j}
			\left( \frac{\delta}{\delta \Lambda_j} \right)^{s_j}
			\hat{D}_j[\Lambda_j, \Lambda_j^*]
	} \\
	& = \left.
		\left(
			\prod_{j=1}^C
			\left( \frac{\delta}{\delta \Lambda_j} \right)^{s_j}
			\left( -\frac{\delta}{\delta \Lambda_j^*} \right)^{r_j}
		\right)
		\chi_W [\bLambda, \bLambda^*]
	\right|_{\bLambda \equiv 0},
\end{eqn}
where $\Delta_{\restbasis_j}[\Lambda_j]$ is a delta functional from \defref{func-calculus:delta-functional}.
Now, recognising the final expression as a part of \lmmref{func-wigner:moments-from-chi},
we immediately get the statement of the theorem.
\end{proof}

% =============================================================================
\section{Specific cases of transformations}
% =============================================================================

In order to Wigner transform the master equation~\eqnref{master-eqn:master-eqn}, we will need several theorems about transformations of specific operator sequences.

First, we will need an expression for high-order commutators of restricted field operators.
They look somewhat similar to those for single-mode bosonic operators, or standard field operators from~\cite{Louisell1990}.

\begin{lemma}
	\begin{eqn*}
		\left[ \Psiop, ( \Psiop^{\prime\dagger} )^l \right]
		& = l \delta_{\restbasis} (\xvec^\prime - \xvec) ( \Psiop^{\prime\dagger} )^{l-1}, \\
		\left[ \Psiop^\dagger, ( \Psiop^\prime )^l \right]
		& = - l \delta_{\restbasis}^* (\xvec^\prime - \xvec) ( \Psiop^\prime )^{l-1}.
	\end{eqn*}
\end{lemma}
\begin{proof}
Proved by induction.
\end{proof}

A further generalisation of these relations is

\begin{lemma}
\label{lmm:func-operators:functional-commutators}
	\begin{eqn*}
		\left[ \Psiop, f( \Psiop^\prime, \Psiop^{\prime\dagger} ) \right]
		& = \delta_{\restbasis} (\xvec^\prime - \xvec) \frac{\partial f}{\partial \Psiop^{\prime\dagger}} \\
		\left[ \Psiop^\dagger, f( \Psiop^\prime, \Psiop^{\prime\dagger} ) \right]
		& = -\delta_{\restbasis}^* (\xvec^\prime - \xvec) \frac{\partial f}{\partial \Psiop^\prime},
	\end{eqn*}
	where $f(z, z^*)$ is a function that can be expanded into the power series of $z$ and $z^*$.
\end{lemma}

First theorem describes the transformation of the linear part of the Hamiltonian~\eqnref{master-eqn:hamiltonian}.

\begin{theorem}
\label{thm:transformations:w-commutator1}
    \begin{eqn*}
    	\mathcal{W} \left[ [\int d\xvec \Psiop_j^\dagger \Psiop_k, \hat{A}] \right]
    	= \int d\xvec \left(
    		- \frac{\delta}{\delta \Psi_j} \Psi_k
    		+ \frac{\delta}{\delta \Psi_k^*} \Psi_j^*
    	\right) \mathcal{W}[\hat{A}].
    \end{eqn*}
\end{theorem}
\begin{proof}
Proved straightforwardly using \thmref{func-wigner:mc-correspondences} and the relation
\begin{eqn}
	\Psi_k \frac{\delta}{\delta \Psi_j} \mathcal{F}
	= \left(
		\frac{\delta}{\delta \Psi_j} \Psi_k
		- \delta_{jk} \delta_{\restbasis}(\xvec, \xvec)
	\right) \mathcal{F}.
\end{eqn}
\end{proof}

Commutators with the Laplacian inside require somewhat special treatment, because it acts on basis functions and, in general, cannot be dragged around like a constant.
But for our purposes we only need one specific case, and, fortunately, in this case it does act like a constant.

\begin{theorem}
\label{thm:transformations:w-laplacian-commutator1}
    \begin{eqn*}
    	\mathcal{W} \left[
    		\int d\xvec [\Psiop^\dagger \nabla^2 \Psiop, \hat{A}]
    	\right]
    	= \int d\xvec \left(
    		- \frac{\delta}{\delta \Psi} \nabla^2 \Psi
    		+ \frac{\delta}{\delta \Psi^*} \nabla^2 \Psi^*
    	\right) \mathcal{W}[\hat{A}].
    \end{eqn*}
\end{theorem}
\begin{proof}
Proved using \thmref{func-wigner:mc-correspondences} and \lmmref{func-calculus:move-laplacian}.
\end{proof}

Next theorem describes the transformation of the non-linear part of the Hamiltonian.
\todo{Can be simplified by assuming the locality of the interaction, thus getting rid of $\xvec^\prime$.}

\begin{theorem}
\label{thm:transformations:w-commutator2}
    \begin{eqn*}
    	\mathcal{W} \left[
    		[
    			\int d\xvec \int d\xvec^\prime
    			\Psiop_j^\dagger \Psiop_k^{\prime\dagger} \Psiop_j^\prime \Psiop_k,
    			\hat{A}
    		]
    	\right] \\
    	= \int d\xvec \int d\xvec^\prime \left(
    		\frac{\delta}{\delta \Psi_j} \left(
    			- \Psi_j^\prime \Psi_k \Psi_k^{\prime*}
    			+ \frac{1}{2} \delta_{jk} \delta_{\restbasis}(\xvec^\prime, \xvec^\prime) \Psi_k
    			+ \frac{1}{2} \delta_{\restbasis}(\xvec, \xvec^\prime) \Psi_j^\prime
    		\right) \right . \\
    	\left. + \frac{\delta}{\delta \Psi_j^{\prime*}} \left(
    			\Psi_j^* \Psi_k \Psi_k^{\prime*}
    			- \frac{1}{2} \delta_{jk} \delta_{\restbasis}(\xvec, \xvec) \Psi_k^{\prime*}
    			- \frac{1}{2} \delta_{\restbasis}(\xvec, \xvec^\prime) \Psi_j^*
    		\right) \right. \\
    	\left. + \frac{\delta}{\delta \Psi_k^\prime} \left(
    			- \Psi_j^\prime \Psi_j^* \Psi_k
    			+ \frac{1}{2} \delta_{jk} \delta_{\restbasis}(\xvec, \xvec) \Psi_j^\prime
    			+ \frac{1}{2} \delta_{\restbasis}(\xvec^\prime, \xvec) \Psi_k
    		\right) \right .\\
    	\left. + \frac{\delta}{\delta \Psi_k^*} \left(
    			\Psi_j^\prime \Psi_j^* \Psi_k^{\prime*}
    			- \frac{1}{2} \delta_{jk} \delta_{\restbasis}(\xvec^\prime, \xvec^\prime) \Psi_j^*
    			- \frac{1}{2} \delta_{\restbasis}(\xvec^\prime, \xvec) \Psi_k^{\prime*}
    		\right) \right. \\
    	\left.
    			+ \frac{\delta}{\delta \Psi_j}
    			\frac{\delta}{\delta \Psi_j^{\prime*}}
    			\frac{\delta}{\delta \Psi_k^\prime}
    			\frac{1}{4} \Psi_k
    			- \frac{\delta}{\delta \Psi_j}
    			\frac{\delta}{\delta \Psi_j^{\prime*}}
    			\frac{\delta}{\delta \Psi_k^*}
    			\frac{1}{4} \Psi_k^{\prime*}
    		\right. \\
    	\left.
    			+ \frac{\delta}{\delta \Psi_k^\prime}
    			\frac{\delta}{\delta \Psi_k^*}
    			\frac{\delta}{\delta \Psi_j}
    			\frac{1}{4} \Psi_j^\prime
    			- \frac{\delta}{\delta \Psi_k^\prime}
    			\frac{\delta}{\delta \Psi_k^*}
    			\frac{\delta}{\delta \Psi_j^{\prime*}}
    			\frac{1}{4} \Psi_j^*
    	\right) \mathcal{W}[\hat{A}].
    \end{eqn*}
\end{theorem}
\begin{proof}
Proof is the same as in case of \thmref{transformations:w-commutator1}.
\end{proof}

Finally, the transformation of loss terms~\eqnref{master-eqn:loss-term} requires more work.
The proof makes use of two auxiliary lemmas.
First one will help us move functional differentials to their intended places (i.e., to the left).

\begin{lemma}
\label{lmm:transformations:swap-differential}
    For $\mathcal{F} \in \mathbb{F}_{\restbasis} \rightarrow \mathbb{F}$ and any non-negative integer $a$, $b$:
    \begin{eqn*}
    	\Psi^a \left( \frac{\delta}{\delta \Psi} \right)^b \mathcal{F}[\Psi, \Psi^*] \\
    	= \sum_{j=0}^{\min(a, b)}
    		\binom{b}{j} \frac{(-1)^j a!}{(a - j)!}
    		\delta_{\restbasis}(\xvec, \xvec)^j
    		\left( \frac{\delta}{\delta \Psi} \right)^{b - j}
    		\Psi^{a - j}
    		\mathcal{F}[\Psi, \Psi^*]
    \end{eqn*}
\end{lemma}
\begin{proof}
Proved straightforwardly by induction.
\end{proof}

Second lemma gives a way to simplify sums obtained from the application of the previous lemma.

\begin{lemma}[Sum rearrangement]
\label{lmm:transformations:sum-rearrangement}
    For any non-negative integer $l$, $u$:
    \begin{eqn*}
    	\sum_{j=0}^l \sum_{k=0}^{\min(l-u,j)} x^{j-k} Q(j, k)
    	= \sum_{v=0}^l x^v \sum_{k=0}^{l-\max(u,v)} Q(v + k, k).
    \end{eqn*}
\end{lemma}
\begin{proof}
Can be proved either by formal manipulation with sets, or by drawing a picture.
\end{proof}

Finally, the loss transformation theorem can be proved.

\begin{theorem}
\label{thm:transformations:w-losses}
    The Wigner transformation of loss term~\eqnref{master-eqn:loss-term} is
    \begin{eqn*}
\fl    	\mathcal{W} \left[ \int d\xvec \hat{\mathcal{L}}_{\lvec} [\hat{A}] \right]
    	= \int d\xvec
    		\sum_{j_1=0}^{l_1} \sum_{k_1=0}^{l_1} \ldots
    		\sum_{j_C=0}^{l_C} \sum_{k_C=0}^{l_C}
    			\left(
    				\prod_{c=1}^C
    					\left( \frac{\delta}{\delta \Psi_c^*} \right)^{j_c}
    					\left( \frac{\delta}{\delta \Psi_c} \right)^{k_c}
    			\right)
    			L_{\lvec, \jvec, \kvec}
    		\mathcal{W}[\hat{A}],
    \end{eqn*}
    where
    \begin{eqn*}
\fl    	L_{\lvec, \jvec, \kvec}
    	= \left( 2 - (-1)^{\sum_c j_c} - (-1)^{\sum_c k_c} \right) \\
    		\prod_{c=1}^C \left(
    			\sum_{m_c=0}^{l_c - \max(j_c, k_c)}
    			Q(l_c, j_c, k_c, m_c)
    			\delta_{\restbasis}(\xvec, \xvec)^{m_c}
    			\Psi_c^{l_c - j_c - m_c}
    			(\Psi_c^*)^{l_c - k_c - m_c}
    		\right),
    \end{eqn*}
    and
    \begin{eqn*}
        Q(l, j, k, m)
    	= (-1)^m \left( \frac{1}{2} \right)^{j + k + m}
    		\frac{(l!)^2}{m! j! k! (l - k - m)! (l - j - m)!}.
    \end{eqn*}
\end{theorem}
\begin{proof}
Proved by applying \thmref{func-wigner:mc-correspondences}, expanding products using binomial theorem, using \lmmref{transformations:swap-differential} to move differentials to front, and applying \lmmref{transformations:sum-rearrangement} to transform summations.
\end{proof}

% =============================================================================
\section{Master equation}
% =============================================================================

The basic Hamiltonian is easily expressed using quantum fields $\Psiop_j^{\dagger}(\xvec)$ and $\Psiop_j(\xvec)$,
where $\xvec$ is $D$-dimensional coordinate vector,
$\Psiop_j^{\dagger}(\xvec)$ creates a boson of component $j$ at a location $\xvec$,
and $\Psiop_j(\xvec)$ destroys one;
the commutators are defined by~\eqnref{func-operators:commutators}.
Second-quantized Hamiltonian for the system looks like:
\begin{eqn}
\label{eqn:master-eqn:hamiltonian}
	\hat{H} / \hbar = \int d\xvec \left\{
		\Psiop_j^{\dagger} K_{jk} \Psiop_k
		+ \frac{1}{2} \int d\xvec^\prime
			\Psiop_j^\dagger (\xvec) \Psiop_k^\dagger (\xvec^\prime)
			U_{jk}(\xvec - \xvec^\prime)
			\Psiop_j (\xvec^\prime) \Psiop_k (\xvec)
	\right\}.
\end{eqn}
Here we use the Einstein summation convention of summing over repeated indices.
$U_{jk}$ is the two-body scattering potential, and $K_{jk}$ is the single-particle Hamiltonian:
\begin{eqn}
	K_{jk} = \left(
			-\frac{\hbar}{2m} \nabla^2 + \omega_j + V_j(\xvec) / \hbar
		\right) \delta_{jk}
		+ \tilde{\Omega}_{jk}(t),
\end{eqn}
where $V_j$ is the external trapping potential for spin $j$,
$\omega_j$ is the internal energy of spin $j$,
and $\tilde{\Omega}_{jk}$ represents a time-dependent coupling that is used to rotate one spin projection into another.

If we impose an energy cutoff $\ecut$ and only take into account low-energy modes,
the general scattering potential $U_{jk}(\xvec - \xvec^\prime)$ can be replaced by contact potential $U_{jk} \delta(\xvec - \xvec^\prime)$~\cite{Morgan2000}, giving the effective Hamiltonian
\begin{eqn}
\label{eqn:master-eqn:effective-H}
	\hat{H} / \hbar = \int d\xvec \left\{
		\Psiop_j^{\dagger} K_{jk} \Psiop_k
		+ \frac{U_{jk}}{2} \Psiop_j^\dagger \Psiop_k^\dagger \Psiop_j \Psiop_k
	\right\}.
\end{eqn}

For $s$-wave scattering in three dimensions the coefficient is $U_{jk} = 4 \pi \hbar a_{jk} / m$,
where $a_{jk}$ is the scattering length.
Note that in general case the coefficient must be renormalised depending on the grid~\cite{Sinatra2002},
but the change is small if $dx \gg a_{jk}$.


Hereinafter field operators and wave functions will be assumed to be defined in restricted basis, unless explicitly stated otherwise.
The Markovian master equation for the system with the inclusion of losses can be written as~\cite{Jack2002}
\begin{eqn}
\label{eqn:master-eqn:master-eqn}
	\frac{d\hat{\rho}}{dt} =
		- \frac{i}{\hbar} \left[ \hat{H}, \hat{\rho} \right]
		+ \sum_{\lvec} \kappa_{\lvec} \int d\xvec
			\mathcal{L}_{\lvec} \left[ \hat{\rho} \right],
\end{eqn}
where $\lvec = (l_1, l_2, \ldots, l_n)$ is a vector indicating the spins that are coupled,
$n$ being the number of interacting particles,
and we have introduced local Liouville loss terms,
\begin{eqn}
	\mathcal{L}_{\lvec} \left[ \hat{\rho} \right] =
		2\hat{O}_{\lvec} \hat{\rho} \hat{O}_{\lvec}^\dagger
		- \hat{O}_{\lvec}^\dagger \hat{O}_{\lvec} \hat{\rho}
		- \hat{\rho} \hat{O}_{\lvec}^\dagger \hat{O}_{\lvec}.
\end{eqn}
The reservoir coupling operators $\hat{O}_{\lvec}$ are the distinct $n$-fold products of local field annihilation operators,
$\hat{O}_{\lvec} = \hat{O}_{\lvec} (\Psiopvec) =
	\Psiop_{l_{1}} (\xvec)
	\Psiop_{l_{2}} (\xvec) \ldots
	\Psiop_{l_{n}} (\xvec),$
describing local $n$-body collision losses.

The master equation allows us to derive an important property.

\begin{theorem}
    \begin{eqn*}
        \frac{d}{dt} \langle \Psiop_j \rangle
    	= P \left[
    		\langle
    			-\frac{i}{\hbar} \left(
    				K_{jm} \Psiop_m
    				+ U_{jm} \Psiop_m^\dagger \Psiop_m \Psiop_j
    			\right)
    			- \sum_{\lvec} \kappa_{\lvec}
    				\frac{\partial \hat{O}_{\lvec}^\dagger}{\partial \Psiop_j^\dagger} \hat{O}_{\lvec}
    		\rangle
    	\right]
    \end{eqn*}
\end{theorem}
\begin{proof}
Obviously,
\begin{eqn}
    \frac{d}{dt} \langle \Psiop_j \rangle
    = \frac{d}{dt} \Trace{ \hat{\rho} \Psiop_j }
    = \Trace{ \frac{d\hat{\rho}}{dt} \Psiop_j }
\end{eqn}
Substituting the right part of the master equation and applying \lmmref{func-operators:functional-commutators} to simplify the loss term, one gets the statement of the lemma.
\end{proof}

% =============================================================================
\section{Wigner truncation and Fokker-Planck equation}
% =============================================================================

In order to solve operator equation~\eqnref{master-eqn:master-eqn} numerically, we will transform it to ordinary differential equation using Wigner transformation~\eqnref{func-wigner:w-transformation}.

Namely, the term with $K_j$ is transformed using \thmref{transformations:w-commutator1} and \thmref{transformations:w-laplacian-commutator1} (since $K_j$ is basically a sum of Laplacian operator and functions of $\xvec$):
\begin{eqn}
	\mathcal{W} \left[ [ \int d\xvec \Psiop_j^\dagger K_{jk} \Psiop_k, \hat{\rho} ] \right]
	= \int d\xvec \left(
			- \frac{\delta}{\delta \Psi_j} K_{jk} \Psi_k
			+ \frac{\delta}{\delta \Psi_k^*} K_{jk} \Psi_j^*
		\right)
		W,
\end{eqn}
where Wigner function $W = \mathcal{W}[\hat{\rho}]$.
Nonlinear term is transformed with \thmref{transformations:w-commutator2} (minding the locality of interaction and assuming $U_{kj} = U_{jk}$):
\begin{eqn}
\fl	\mathcal{W} \left[
		[
			\int d\xvec \frac{U_{jk}}{2}
				\Psiop_j^\dagger \Psiop_k^\dagger \Psiop_j \Psiop_k,
			\hat{\rho}
		]
	\right]
	= & \int d\xvec U_{jk} \left(
		\frac{\delta}{\delta \Psi_j} \left(
			- \Psi_j \Psi_k \Psi_k^*
			+ \frac{\delta_{\restbasis}(\xvec, \xvec)}{2} ( \delta_{jk} \Psi_k + \Psi_j )
		\right) \right. \\
	&	\left. + \frac{\delta}{\delta \Psi_j^*} \left(
			\Psi_j^* \Psi_k \Psi_k^*
			- \frac{\delta_{\restbasis}(\xvec, \xvec)}{2} ( \delta_{jk} \Psi_k^* + \Psi_j^* )
		\right) \right. \\
	&	\left.
			+ \frac{\delta}{\delta \Psi_j}
			\frac{\delta}{\delta \Psi_j^*}
			\frac{\delta}{\delta \Psi_k}
			\frac{1}{4} \Psi_k
			- \frac{\delta}{\delta \Psi_j}
			\frac{\delta}{\delta \Psi_j^*}
			\frac{\delta}{\delta \Psi_k^*}
			\frac{1}{4} \Psi_k^*
		\right) W.
\end{eqn}

Loss operator is transformed with \thmref{transformations:w-losses}.
Not writing the resulting expression here, because with the absence of truncation it is too long,
and is practically the same as in theorem statement.

Transformation to SDE is a very convenient way to solve FPE numerically, but we can only do that if it does not have functional derivatives of order more than 2.
In the assumption of the state being coherent, the condition for truncation can be shown to be~\cite{Sinatra2002}
\begin{eqn}
    N \gg |\restbasis|,
\end{eqn}
where $N$ is the number of atoms.
This condition is equivalent to~\cite{Norrie2006}
\begin{eqn}
    \delta_L(\xvec, \xvec) \ll | \Psi_j |^2.
\end{eqn}

\begin{lemma}
    Assuming the conditions for Wigner truncation are satisfied,
    the result of Wigner transformation of the nonlinear term can be written as
    \begin{eqn*}
    	\mathcal{W} \left[
    		[
    			\frac{U_{jk}}{2}
    				\Psiop_j^\dagger \Psiop_k^\dagger \Psiop_j \Psiop_k,
    			\hat{\rho}
    		]
    	\right]
    	= U_{jk} \left(
    		\frac{\delta}{\delta \Psi_j^*} \Psi_j^* \Psi_k \Psi_k^*
    		- \frac{\delta}{\delta \Psi_j} \Psi_j \Psi_k \Psi_k^*
    	\right) W.
    \end{eqn*}
\end{lemma}

\begin{theorem}
    Assuming the conditions for Wigner truncation are satisfied, the result of Wigner transformation of the loss term can be written as
    \begin{eqn*}
\fl    	\mathcal{W}[\mathcal{L}_{\lvec}[\hat{\rho}]]
    	= \sum_{n=1}^C
    			\frac{\delta}{\delta \Psi_n^*} \frac{\partial O_{\lvec}}{\partial \Psi_n} O_{\lvec}^*
    	+ \sum_{n=1}^C
    		\frac{\delta}{\delta \Psi_n} \frac{\partial O_{\lvec}^*}{\partial \Psi_n^*} O_{\lvec}
    	+ \sum_{n=1}^C \sum_{p=1}^C
    		\frac{\delta^2}{\delta \Psi_n^* \delta \Psi_p}
    		\frac{\partial O_{\lvec}}{\partial \Psi_n}
    		\frac{\partial O_{\lvec}^*}{\partial \Psi_p^*}.
    \end{eqn*}
    where $O_{\lvec} \equiv O_{\lvec}(\Psivec) = \prod_{c=1}^C \Psi_c^{l_c}$.
\end{theorem}
\begin{proof}
The proof is basically a simplification of the result of \thmref{transformations:w-losses} under certain conditions.
First, we are neglecting all occurrences of $\delta_{\restbasis}$, which means setting $m_c = 0$ for every $c$.
Second, we are dropping all terms with high order differentials,
which can be expressed as limiting $\sum j_c + \sum k_c \le 2$.
The only combinations of $j_c$ and $k_c$ for which $Z(\jvec, \kvec)$ is not zero are thus
$\{ j_c = \delta_{cn}, k_c = 0, n \in [1, C] \}$,
$\{ j_c = 0, k_c = \delta_{cn}, n \in [1, C] \}$ and
$\{ j_c = \delta_{cn}, k_c = \delta_{cp}, n \in [1, C], p \in [1, C] \}$.
These combinations produce terms with $\delta/\delta \Psi_n^*$,
$\delta/\delta \Psi_n$ and
$\delta^2/\delta \Psi_p \delta \Psi_n^*$ respectively.
Applying these conditions one can get the statement of the theorem.
\end{proof}

Assuming that $K_{jk}$, $U_{jk}$ and $\kappa_{\lvec}$ are real-valued, resulting Wigner-transformed master equation is
\begin{eqn}
\fl	\frac{dW}{dt}
	= \int d\xvec \left(
		- \sum_{j=1}^C \frac{\delta}{\delta \Psi_j} A_j
		- \sum_{j=1}^C \frac{\delta}{\delta \Psi_j^*} A_j^*
		+ \sum_{j=1}^C \sum_{k=1}^C \frac{\delta^2}{\delta \Psi_j^* \delta \Psi_k} D_{jk}
	\right) W,
\end{eqn}
where
\begin{eqn}
	A_j = -\frac{i}{\hbar} \left(
			\sum_{k=1}^C K_{jk} \Psi_k
			+ \sum_{k=1}^C U_{jk} \Psi_j \Psi_k \Psi_k^*
		\right)
		- \sum_{\lvec} \kappa_{\lvec} \frac{\partial O_{\lvec}^*}{\partial \Psi_j^*} O_{\lvec},
\end{eqn}
and
\begin{eqn}
	D_{jk} = \sum_{\lvec} \kappa_{\lvec}
		\frac{\partial O_{\lvec}}{\partial \Psi_j}
		\frac{\partial O_{\lvec}^*}{\partial \Psi_k^*}.
\end{eqn}

According to \thmref{app-fpe:fpe-sde-func}, this equation is equivalent to a set of stochastic differential equations in It\^{o} form
\begin{eqn}
	d\Psi_j = \mathcal{P} \left[
		A^{(j)} dt + \sum_{\lvec} B_{\lvec}^{(j)} Q_{\lvec}
	\right],
\end{eqn}
where
\begin{eqn}
    B_{\lvec}^{(j)} = \sqrt{\kappa_{\lvec}} \frac{\partial O_{\lvec}}{\partial \Psi_j},
\end{eqn}
and $Q_{\lvec}$ is a functional Wiener process:
\begin{eqn}
	Q_{\lvec} = \sum_{\nvec \in \fullbasis} \phi_j Z_{\lvec,\nvec},
\end{eqn}
and $Z_{\lvec,\nvec}$ are, in turn, independent complex-valued Wiener processes.

% =============================================================================
\section{Initial states}
% =============================================================================

Initial values for the numerical integration of equations~\eqnref{fpe:sdes} are obtained by finding the Wigner transformation of the density matrix for the desired initial state, and then sampling the initial values according to the resulting Wigner function.
As an example, consider the simple case with a coherent initial state.

\begin{theorem}
    The Wigner distribution for a multi-mode coherent state with with expectation values
    $\alpha_{\nvec}(0) = \alpha_{\nvec}^{(0)}$, $\nvec \in \restbasis$ is
    \begin{eqn*}
    	W_c (\balpha^{(0)})
    	= \left( \frac{2}{\pi} \right)^{|\restbasis|} \prod_{\nvec \in \restbasis}
    		\exp(-2 |\alpha_{\nvec} - \alpha_{\nvec}^{(0)}|^2).
    \end{eqn*}
\end{theorem}
\begin{proof}
The density matrix of the state is
\begin{eqn}
	\hat{\rho}
	= \vert \alpha_{\nvec}^{(0)},\, \nvec \in \restbasis \rangle
		\langle \alpha_{\nvec}^{(0)},\, \nvec \in \restbasis \vert
	= \left( \prod_{\nvec \in \restbasis} \vert \alpha_{\nvec}^{(0)} \rangle \right)
		\left( \prod_{\nvec \in \restbasis} \langle \alpha_{\nvec}^{(0)} \vert \right).
\end{eqn}
Then the characteristic function is
\begin{eqn}
	\chi_W (\balpha^{(0)})
	= \prod_{\nvec \in \restbasis}
		\langle \alpha_{\nvec}^{(0)} \vert
		\hat{D}_{\nvec} (\lambda_{\nvec}, \lambda_{\nvec}^*)
		\vert \alpha_{\nvec}^{(0)} \rangle
\end{eqn}
Using the properties of the displacement operator, this can be transformed to
\begin{eqn}
	\chi_W (\balpha^{(0)})
	= \prod_{\nvec \in \restbasis}
		\exp(
			- \lambda_{\nvec}^* \alpha_{\nvec}^{(0)}
			+ \lambda_{\nvec} (\alpha_{\nvec}^{(0)})^*
			- \frac{1}{2} |\lambda|^2
		).
\end{eqn}
Finally, Wigner function is
\begin{eqn}
\fl	W_c (\balpha^{(0)})
	= \frac{1}{\pi^{2|\restbasis|}} \prod_{\nvec \in \restbasis} \left(
		\int d^2\lambda_{\nvec}
			\exp(
				- \lambda_{\nvec} (\alpha_{\nvec}^* - (\alpha_{\nvec}^{(0)})^*)
				+ \lambda_{\nvec}^* (\alpha_{\nvec} - \alpha_{\nvec}^{(0)})
				- \frac{1}{2} |\lambda|^2
			)
	\right) \\
	= \left( \frac{2}{\pi} \right)^{|\restbasis|} \prod_{\nvec \in \restbasis}
		\exp(-2 |\alpha_{\nvec} - \alpha_{\nvec}^{(0)}|^2).
	\qedhere
\end{eqn}
\end{proof}

The resulting Wigner distribution is a product of independent complex-valued Gaussian distributions for each mode,
with the expectation value equal to the expectation value of the mode,
and the variance equal to $\frac{1}{2}$.
Therefore the initial state can be sampled as
\begin{eqn}
	\alpha_{\nvec} = \alpha_{\nvec}^{(0)} + \frac{1}{\sqrt{2}} \eta_{\nvec},
\end{eqn}
where $\eta_{\nvec}$ are normally distributed complex random numbers with zero mean,
$\langle \eta_{\mvec} \eta_{\nvec} \rangle = 0$ and
$\langle \eta_{\mvec} \eta_{\nvec}^* \rangle = \delta_{\mvec,\nvec}$
(in other words, with components distributed independently with variance $\frac{1}{2}$).
This looks like adding half a ``vacuum particle'' to each mode.
In functional form this can be written as
\begin{eqn}
	\Psi_j(\xvec, 0)
	= \Psi_j^{(0)}(\xvec, 0)
		+ \sum_{\nvec \in \restbasis} \frac{\eta_{j,\nvec}}{2} \phi_{\nvec},
\end{eqn}
where $\Psi_j^{(0)}(\xvec, 0)$ is the ``classical'' ground state of the system.

\todo{We can describe the Wigner function for the number state here, as an example of non-positive Wigner function.}

% =============================================================================
\section{Conclusion}
% =============================================================================

\todo{Conclusion goes here.}


\appendix

% =============================================================================
\section{Functional Fokker-Planck equation}
% =============================================================================

The general approach to numerical solution of the Fokker-Planck equation is to transform it to the equivalent set of stochastic differential equations (SDEs).
In the textbooks this transformation is defined for real variables only~\cite{Risken1996}, while we have functional FPE with complex-valued functions.

Our starting point is the reformulation of the theorem for real-valued multivariable FPE from~\cite{Risken1996}:

\begin{lemma}[FPE--SDEs correspondence in convenient form.]
\label{lmm:app-fpe:fpe-sde-real}
    If $\zvec^T \equiv (z_1 \ldots z_M)$ is a set of real-valued variables,
    Fokker-Planck equation
    \begin{eqn*}
    	\frac{dW}{dt}
    	= -\boldsymbol{\partial}_{\zvec}^T \boldsymbol{a} W
    	+ \frac{1}{2} \Trace{ \boldsymbol{\partial}_{\zvec} \boldsymbol{\partial}_{\zvec}^T B B^T } W
    \end{eqn*}
    is equivalent to a set of stochastic differential equations in It\^{o} form
    \begin{eqn*}
    	d\zvec = \boldsymbol{a} dt + B d\Zvec
    \end{eqn*}
    and to a set of stochastic differential equations in Stratonovich form
    \begin{eqn*}
    	d\zvec = (\boldsymbol{a} - \boldsymbol{c})dt + B d\Zvec,
    \end{eqn*}
    where the noise-induced (or spurious) drift vector $\boldsymbol{c}$ has elements
    \begin{eqn*}
    	c_i
    	= \sum_{k,j} B_{kj} \frac{\partial}{\partial z_k} B_{ij}
    	= \Trace{B^T \boldsymbol{\partial}_z \boldsymbol{e}_i^T B},
    \end{eqn*}
    $\boldsymbol{e}_i$ being the unit vector with elements $(\boldsymbol{e}_i)_j = \delta_{ij}$.
    Here $W \equiv W(\zvec)$ is a probability distribution,
    $\boldsymbol{a} \equiv \boldsymbol{a}(\zvec)$ is a vector function,
    $B \equiv B(\zvec)$ is a matrix function ($B$ having size $M \times L$, where $L$ corresponds to the number of noise sources),
    $\partial_{\zvec}^T \equiv (\partial_{z_1} \ldots \partial_{z_M})$ is a vector differential,
    and $d\Zvec$ is a standard $L$-dimensional real-valued Wiener process.
\end{lemma}
\begin{proof}
For details see~\cite{Risken1996}, sections 3.3 and 3.4.
\end{proof}

\begin{theorem}
\label{thm:app-fpe:fpe-sde-complex}
    If $\boldsymbol{\alpha}^T \equiv (\alpha_1 \ldots \alpha_M)$ is a set of complex-valued variables,
    Fokker-Planck equation
    \begin{eqn*}
    	\frac{dW}{dt}
    	= -\boldsymbol{\partial}_{\boldsymbol{\alpha}}^T \boldsymbol{a} W - \boldsymbol{\partial}_{\boldsymbol{\alpha}^*}^T \boldsymbol{a}^* W
    	+ \Trace{ \boldsymbol{\partial}_{\boldsymbol{\alpha}} \boldsymbol{\partial}_{\boldsymbol{\alpha}^*}^T B B^H } W
    \end{eqn*}
    is equivalent to a set of stochastic differential equations in It\^{o} form
    \begin{eqn*}
    	d\boldsymbol{\alpha} = \boldsymbol{a} dt + B d\Zvec,
    \end{eqn*}
    or to Stratonovich form
    \begin{eqn*}
    	d\boldsymbol{\alpha} = (\boldsymbol{a} - \boldsymbol{c}) dt + B d\Zvec,
    \end{eqn*}
    where noise-induced drift term is
    \begin{eqn*}
    	c_j = \Trace{ B^H \boldsymbol{\partial}_{\boldsymbol{\alpha}^*} \boldsymbol{e}_j^T B },
    \end{eqn*}
    and $d\Zvec = (d\boldsymbol{X} + id\boldsymbol{Y}) / \sqrt{2}$ is an $M$-dimensional complex-valued Wiener process,
    containing two standard $M$-dimensional Wiener processes $d\boldsymbol{X}$ and $d\boldsymbol{Y}$.
\end{theorem}
\begin{proof}
Proved straightforwardly by transforming the equation to real variables and applying \lmmref{app-fpe:fpe-sde-real}.
\end{proof}

\begin{theorem}[Multi-component extension of \thmref{app-fpe:fpe-sde-complex}]
\label{thm:app-fpe:mc-fpe-sde}
    If $\boldsymbol{\alpha}^{(c)},\, c = 1..C$ are $C$ sets of complex variables $\boldsymbol{\alpha}^{(c)} \equiv (\alpha_1^{(c)} \ldots \alpha_M^{(c)})$,
    then Fokker-Planck equation
    \begin{eqn}
    	\frac{dW}{dt}
    	= - \sum_{c=1}^C \boldsymbol{\partial}_{\boldsymbol{\alpha}^{(c)}}^T \boldsymbol{a}^{(c)} W
    	- \sum_{c=1}^C \boldsymbol{\partial}_{(\boldsymbol{\alpha}^{(c)})^*}^T (\boldsymbol{a}^{(c)})^* W
    	+ \sum_{m=1}^c \sum_{n=1}^c
    		\Trace{
    			\boldsymbol{\partial}_{\boldsymbol{\alpha}^{(m)}}
    			\boldsymbol{\partial}_{(\boldsymbol{\alpha}^{(n)})^*}^T
    			B^{(n)} (B^{(m)})^H
    		} W
    \end{eqn}
    is equivalent to a set of stochastic differential equations in It\^{o} form
    \begin{eqn}
    	d\boldsymbol{\alpha}^{(c)} = \boldsymbol{a}^{(c)} dt + B^{(c)} d\Zvec,\, c = 1..C
    \end{eqn}
    or to Stratonovich form
    \begin{eqn*}
    	d\boldsymbol{\alpha}^{(c)} = (\boldsymbol{a}^{(c)} - \boldsymbol{c}^{(c)}) dt + B^{(c)} d\Zvec,
    \end{eqn*}
    where noise-induced drift term is
    \begin{eqn*}
    	c_j^{(c)} = \sum_{n=1}^C
    		\Trace{ (B^{(n)})^H \boldsymbol{\partial}_{(\boldsymbol{\alpha}^{(n)})^*} \boldsymbol{e}_j^T B^{(c)} },
    \end{eqn*}
    and $d\Zvec$ is an $L$-dimensional complex-valued Wiener process.
\end{theorem}
\begin{proof}
Proved by joining vectors from all components into one vector and applying \thmref{app-fpe:fpe-sde-complex}.
\end{proof}

\begin{theorem}
\label{thm:app-fpe:fpe-sde-func}
    Functional FPE
    \begin{eqn*}
    	\frac{dW}{dt}
    	= \int d\xvec \left(
    		- \sum_{j=1}^C \frac{\delta}{\delta \Psi_j} A_j
    		- \sum_{j=1}^C \frac{\delta}{\delta \Psi_j^*} A_j^*
    		+ \sum_{j=1}^C \sum_{k=1}^C \frac{\delta^2}{\delta \Psi_j \delta \Psi_k^*}
    			\sum_{\lvec} B_{\lvec}^{(k)} (B_{\lvec}^{(j)})^*
    	\right) W
    \end{eqn*}
    is equivalent to the set of SDEs in It\^{o} form
    \begin{eqn*}
    	d\Psi_j = \mathcal{P} \left[
    		A^{(j)} dt + \sum_{\lvec} B_{\lvec}^{(j)} d\boldsymbol{Q}_{\lvec}
    	\right],
    \end{eqn*}
    or in Stratonovich form
    \begin{eqn*}
    	d\Psi_j = \mathcal{P} \left[
    		(A^{(j)} - C^{(j)}) dt + \sum_{\lvec} B_{\lvec}^{(j)} d\boldsymbol{Q}_{\lvec}
    	\right],
    \end{eqn*}
    where
    \begin{eqn*}
    	C^{(j)} = \sum_{n=1}^C \sum_{\lvec}
    		(B_{\lvec}^{(n)})^*
    		\frac{\delta}{\delta \Psi_n^*}
    		B_{\lvec}^{(j)},
    \end{eqn*}
    and $\boldsymbol{Q}_{\lvec}$ is a functional Wiener process:
    \begin{eqn*}
    	\boldsymbol{Q}_{\lvec} = \sum_{\nvec \in L \cup H} \phi_j \Zvec_{\lvec,\nvec}.
    \end{eqn*}
\end{theorem}
\begin{proof}
Proved by expanding functional derivatives and applying \thmref{app-fpe:mc-fpe-sde}.
The diffusion term has to be transformed in order to conform to the theorem:
\begin{eqn}
	\int d\xvec \phi_{\nvec} \phi_{\mvec}^* \sum_{\lvec} B_{\lvec}^{(k)} (B_{\lvec}^{(j)})^*
	& = \int d\xvec \int d\xvec^\prime
			\phi_{\nvec}^\prime \phi_{\mvec}^*
			\sum_{\lvec} B_{\lvec}^{(k)} (B_{\lvec}^{(j)})^{\prime*}
			\delta(\xvec - \xvec^\prime) \\
	& = \int d\xvec \int d\xvec^\prime
			\phi_{\nvec}^\prime \phi_{\mvec}^*
			\sum_{\lvec} B_{\lvec}^{(k)} (B_{\lvec}^{(j)})^{\prime*}
			\sum_{\pvec \in L \cup H} \phi_{\pvec}^{\prime*} \phi_{\pvec} \\
	& = \sum_{\pvec \in L \cup H, \lvec}
		\int d\xvec
			\phi_{\nvec}^* B_{\lvec}^{(k)} \phi_{\pvec}
		\int d\xvec
			\phi_{\mvec} (B_{\lvec}^{(j)})^* \phi_{\pvec}^*
\end{eqn}
Grouping terms back and recognising the definition of projection transformation, one gets the statement of the theorem.
\end{proof}


\section*{References}
\bibliographystyle{unsrt}
\bibliography{qsim-long}

\end{document}