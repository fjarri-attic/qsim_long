% =============================================================================
\section{Single-mode Wigner representation}
% =============================================================================

We will need the displacement operator which was first introduced by Weyl~\cite{Weyl1950}:

\begin{definition}
\label{def:sm-wigner:displacement-op}
	If $\hat{a}^\dagger$ and $\hat{a}$ are bosonic creation and annihilation operators, displacement operator $\hat{D}$ is
	\begin{equation*}
		\hat{D}(\lambda, \lambda^*) = \exp(\lambda \hat{a}^\dagger - \lambda^* \hat{a}).
	\end{equation*}
\end{definition}

Using Baker-Hausdorff theorem to split non-commuting operators in the exponent, one can find that
\begin{eqn}
\label{eqn:sm-wigner:displacement-derivatives}
	\frac{\partial}{\partial \lambda} \hat{D}(\lambda, \lambda^*)
	= \hat{D}(\lambda, \lambda^*) (\hat{a}^\dagger + \frac{1}{2} \lambda^*)
	= (\hat{a}^\dagger - \frac{1}{2} \lambda^*) \hat{D}(\lambda, \lambda^*), \\
	-\frac{\partial}{\partial \lambda^*} \hat{D}(\lambda, \lambda^*)
	= \hat{D}(\lambda, \lambda^*) (\hat{a} + \frac{1}{2} \lambda)
	= (\hat{a} - \frac{1}{2} \lambda) \hat{D}(\lambda, \lambda^*).
\end{eqn}

Wigner transformation converts an operator $\hat{A}$ on a Hilbert space to a function $\mathcal{W}[\hat{A}](\alpha, \alpha^*)$ on a phase space.
In terms of the displacement operator Wigner transformation $\mathcal{W}$ and Wigner function $W$ can be defined as

\begin{definition}
\label{def:sm-wigner:w-transformation}
	Wigner transformation of an operator $\hat{A}$:
	\begin{eqn*}
		\mathcal{W}[\hat{A}]
		= \frac{1}{\pi^2} \int d^2 \lambda \exp(-\lambda \alpha^* + \lambda^* \alpha)
			\Trace{ \hat{A} \hat{D}(\lambda, \lambda^*) }.
	\end{eqn*}
	Wigner function is a Wigner transformation of a density matrix:
	\begin{eqn*}
		W(\alpha, \alpha^*) \equiv \mathcal{W}[\hat{\rho}].
	\end{eqn*}
\end{definition}

The Wigner function always exists for any density matrix~\cite{Gardiner2004}, and the correspondence $W \leftrightarrow \hat{\rho}$ is a bijection.
In some cases it is convenient to use the Wigner function in form
\begin{equation}
\label{eqn:sm-wigner:w-function}
	W (\alpha, \alpha^*)
	= \frac{1}{\pi^2} \int d^2 \lambda \exp(-\lambda \alpha^* + \lambda^* \alpha)
		\chi_W (\lambda, \lambda^*),
\end{equation}
where $\chi_W (\lambda, \lambda^*)$ is the characteristic function:
\begin{equation}
	\chi_W (\lambda, \lambda^*)
	= \Trace{ \hat{\rho} \hat{D}(\lambda, \lambda^*) }.
\end{equation}

\begin{lemma}
\label{lmm:sm-wigner:zero-integrals}
	For any integer $m$ and $n$
	\begin{eqn*}
		\int d^2\lambda
			\frac{\partial}{\partial \lambda} & \left(
				\exp(-\lambda \alpha^* + \lambda^* \alpha)
				\left( \frac{\partial}{\partial \lambda} \right)^m
				\left( -\frac{\partial}{\partial \lambda^*} \right)^n
				\hat{D}(\lambda, \lambda^*)
			\right)
		= 0, \\
		\int d^2\lambda
			\frac{\partial}{\partial \lambda^*} & \left(
				\exp(-\lambda \alpha^* + \lambda^* \alpha)
				\left( \frac{\partial}{\partial \lambda} \right)^m
				\left( -\frac{\partial}{\partial \lambda^*} \right)^n
				\hat{D}(\lambda, \lambda^*)
			\right)
		= 0.
	\end{eqn*}
\end{lemma}
\begin{proof}
We will prove the first equation.
Expanding $\lambda = x + iy$ and applying Baker-Hausdorff theorem:
\begin{eqn}
\label{eqn:sm-wigner:BH-displacement}
	\hat{D}(\lambda, \lambda^*)
	= \exp(ixy) \exp(x(\hat{a}^\dagger - \hat{a})) \exp(iy(\hat{a}^\dagger + \hat{a}))
\end{eqn}
Expanding derivatives over $\partial/\partial\lambda$ and $\partial/\partial\lambda^*$ in terms of $\partial/\partial x$ and $\partial/\partial y$, and exponents in~\eqnref{sm-wigner:BH-displacement} as power series:
\begin{eqn}
\fl	\int d^2\lambda
		\frac{\partial}{\partial \lambda} \left(
			\exp(-\lambda \alpha^* + \lambda^* \alpha)
			\left( \frac{\partial}{\partial \lambda} \right)^m
			\left( -\frac{\partial}{\partial \lambda^*} \right)^n
			\hat{D}(\lambda, \lambda^*)
		\right) \\
\fl	= \sum_{r=0}^{\infty} \sum_{s=0}^{\infty} \left(
			\int d^2\lambda
			\frac{\partial}{\partial \lambda} \left(
				\exp(-\lambda \alpha^* + \lambda^* \alpha)
				\exp(ixy) f_{mnrs}(x, y)
			\right)
		\right)
		g_{rs}(\hat{a}, \hat{a}^\dagger) \\
\fl	= 0,
\end{eqn}
where $f_{mnrs}(x, y)$ and $g_{rs}(\hat{a}, \hat{a}^\dagger)$ are some finite-order polynomials,
and we used \lmmref{c-numbers:zero-integrals} to evaluate integrals over $\lambda$.
\end{proof}

\begin{lemma}
	\label{lmm:sm-wigner:moments-from-chi}
	For any integer $r$ and $s$
	\begin{eqn*}
		\langle \symprod{ \hat{a}^r (\hat{a}^\dagger)^s } \rangle
		= \left.
			\left( \frac{\partial}{\partial \lambda} \right)^s
			\left( -\frac{\partial}{\partial \lambda^*} \right)^r
			\chi_W (\lambda, \lambda^*)
		\right|_{\lambda=0}.
	\end{eqn*}
\end{lemma}
\begin{proof}
The displacement operator can be expanded into power series as
\begin{eqn}
	\exp (\lambda \hat{a}^\dagger - \lambda^* \hat{a})
	= \sum_{r,s}
		\frac{(-\lambda^*)^r \lambda^s}{r!s!}
		\symprod{ \hat{a}^r (\hat{a}^\dagger)^s }.
\end{eqn}
Thus
\begin{eqn}
	\chi_W(\lambda, \lambda^*)
	& = \sum_{r,s}
		\frac{(-\lambda^*)^r \lambda^s}{r!s!}
		\Trace{
			\hat{\rho} \symprod{ \hat{a}^r (\hat{a}^\dagger)^s }
		} \\
	& = \sum_{r,s}
		\frac{(-\lambda^*)^r \lambda^s}{r!s!}
		\langle \symprod{ \hat{a}^r (\hat{a}^\dagger)^s } \rangle.
\end{eqn}
Apparently, the application of $(\partial / \partial \lambda)^s$ and $(-\partial / \partial \lambda^*)^r$ will eliminate all lower order moments, and setting $\lambda = 0$ afterwards will eliminate all higher order moments, leaving only $\symprod{ \hat{a}^r (\hat{a}^\dagger)^s }$:
\begin{eqn}
	\left.
		\left( \frac{\partial}{\partial \lambda} \right)^s
		\left( -\frac{\partial}{\partial \lambda^*} \right)^r
		\chi_W (\lambda, \lambda^*)
	\right|_{\lambda=0}
	& = r! s! \frac{1}{r! s!}
		\langle \symprod{ \hat{a}^r (\hat{a}^\dagger)^s } \rangle \\
	& = \langle \symprod{ \hat{a}^r (\hat{a}^\dagger)^s } \rangle.
	\qedhere
\end{eqn}
\end{proof}

Now we have all we need to prove two main theorems for working with Wigner representation.
First one allows us to transform operator equations to ordinary differential equations.
Second one gives a way to calculate any moments of the operators, given the solution of the differential equation.

\begin{theorem}[Operator correspondences]
\label{thm:sm-wigner:correspondences}
\begin{eqn*}
	\mathcal{W} [ \hat{a} \hat{A} ]
		& = \left( \alpha + \frac{1}{2} \frac{\partial}{\partial \alpha^*} \right) \mathcal{W}[\hat{A}],
	\quad
	\mathcal{W} [ \hat{a}^\dagger \hat{A} ]
		= \left( \alpha^* - \frac{1}{2} \frac{\partial}{\partial \alpha} \right) \mathcal{W}[\hat{A}], \\
	\mathcal{W} [ \hat{A} \hat{a} ]
		& = \left( \alpha - \frac{1}{2} \frac{\partial}{\partial \alpha^*} \right) \mathcal{W}[\hat{A}],
	\quad
	\mathcal{W} [ \hat{A} \hat{a}^\dagger ]
		= \left( \alpha^* + \frac{1}{2} \frac{\partial}{\partial \alpha} \right) \mathcal{W}[\hat{A}].
\end{eqn*}
\end{theorem}
\begin{proof}
We will prove the first correspondence.
First, let us transform the trace using~\eqnref{sm-wigner:displacement-derivatives}:
\begin{eqn}
	\Trace{ \hat{a} \hat{A} \hat{D} }
	& = \Trace{ \hat{A} \left(
		-\frac{\partial}{\partial \lambda^*}
		-\frac{1}{2} \lambda
	\right) \hat{D}} \\
	& = \left(
		-\frac{\partial}{\partial \lambda^*}
		-\frac{1}{2} \lambda
	\right) \Trace{ \hat{A} \hat{D}}
\end{eqn}
Now we need to move this additional multiplier outside the integral in the expression for Wigner function:
\begin{eqn}
\fl	\mathcal{W} [ \hat{a} \hat{A} ]
	& = \frac{1}{\pi^2} \int d^2 \lambda \exp(-\lambda \alpha^* + \lambda^* \alpha)
		\Trace{ \hat{a} \hat{A} \hat{D}(\lambda, \lambda^*) } \\
\fl	& = \frac{1}{2} \frac{\partial}{\partial \alpha^*} \mathcal{W} [\hat{A}]
	- \frac{1}{\pi^2} \int d^2 \lambda \exp(-\lambda \alpha^* + \lambda^* \alpha)
		\frac{\partial}{\partial \lambda^*}
		\Trace{ \hat{A} \hat{D}(\lambda, \lambda^*) } \\
\fl	& = \frac{1}{2} \frac{\partial}{\partial \alpha^*} \mathcal{W} [\hat{A}]
	+ \frac{1}{\pi^2} \int d^2 \lambda \left(
		\frac{\partial}{\partial \lambda^*} \exp(-\lambda \alpha^* + \lambda^* \alpha)
	\right)
	\Trace{ \hat{A} \hat{D}(\lambda, \lambda^*) } \\
\fl	& = \left( \alpha + \frac{1}{2} \frac{\partial}{\partial \alpha^*} \right) \mathcal{W} [\hat{A}],
\end{eqn}
where we used~\lmmref{sm-wigner:zero-integrals} to move the partial derivative over $\lambda^*$.
\end{proof}

\begin{theorem}[Calculation of moments]
\label{thm:sm-wigner:moments}
	Expectations of symmetrically ordered operator products are moments of the Wigner function:
	\begin{eqn*}
		\langle \symprod{ \hat{a}^r (\hat{a}^\dagger)^s } \rangle
		= \int d^2\alpha\, \alpha^r (\alpha^*)^s W(\alpha, \alpha^*)
	\end{eqn*}
\end{theorem}
\begin{proof}
By definition of the Wigner function:
\begin{eqn}
\int d^2\alpha\, \alpha^r (\alpha^*)^s W(\alpha, \alpha^*) \\
	= \frac{1}{\pi^2} \Trace{ \hat{\rho}
			\int d^2\alpha\, \alpha^r (\alpha^*)^s
			\int d^2\lambda \exp(-\lambda \alpha^* + \lambda^* \alpha)
			\hat{D}(\lambda, \lambda^*)
		}
\end{eqn}
Integrating by parts and eliminating terms which fit \lmmref{sm-wigner:zero-integrals}:
\begin{eqn}
\fl	= \frac{1}{\pi^2} \Trace{ \hat{\rho}
			\int d^2\alpha \int d^2\lambda
			\exp(-\lambda \alpha^* + \lambda^* \alpha)
			\left( \frac{\partial}{\partial \lambda} \right)^s
			\left( -\frac{\partial}{\partial \lambda^*} \right)^r
			\hat{D} (\lambda, \lambda^*)
		}
\end{eqn}
Evaluating integral over $\alpha$ using \lmmref{c-numbers:fourier-of-moments}:
\begin{eqn}
	& = \int d^2\lambda\,
		\delta (\Real \lambda) \delta (\Imag \lambda)
		\left( \frac{\partial}{\partial \lambda} \right)^s
		\left( -\frac{\partial}{\partial \lambda^*} \right)^r
		\Trace{
			\hat{\rho}
			\hat{D}(\lambda, \lambda^*)
		} \\
	& = \left.
		\left( \frac{\partial}{\partial \lambda} \right)^s
		\left( -\frac{\partial}{\partial \lambda^*} \right)^r
		\chi_W (\lambda, \lambda^*)
	\right|_{\lambda=0}.
\end{eqn}
Now, recognising the final expression as a part of \lmmref{sm-wigner:moments-from-chi},
we immideately get the statement of the theorem.
\end{proof}

