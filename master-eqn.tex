% =============================================================================
\section{Master equation}
% =============================================================================

The basic Hamiltonian is easily expressed using quantum fields $\Psiop_j^{\dagger}(\xvec)$ and $\Psiop_j(\xvec)$,
where $\xvec$ is $D$-dimensional coordinate vector,
$\Psiop_j^{\dagger}(\xvec)$ creates a boson of component $j$ at a location $\xvec$,
and $\Psiop_j(\xvec)$ destroys one;
the commutators are defined by~\eqnref{func-operators:commutators}.
Second-quantized Hamiltonian for the system looks like:
\begin{eqn}
\label{eqn:master-eqn:hamiltonian}
	\hat{H} / \hbar = \int d\xvec \left\{
		\Psiop_j^{\dagger} K_{jk} \Psiop_k
		+ \frac{1}{2} \int d\xvec^\prime
			\Psiop_j^\dagger (\xvec) \Psiop_k^\dagger (\xvec^\prime)
			U_{jk}(\xvec - \xvec^\prime)
			\Psiop_j (\xvec^\prime) \Psiop_k (\xvec)
	\right\}.
\end{eqn}
Here we use the Einstein summation convention of summing over repeated indices.
$U_{jk}$ is the two-body scattering potential, and $K_{jk}$ is the single-particle Hamiltonian:
\begin{eqn}
	K_{jk} = \left(
			-\frac{\hbar}{2m} \nabla^2 + \omega_j + V_j(\xvec) / \hbar
		\right) \delta_{jk}
		+ \tilde{\Omega}_{jk}(t),
\end{eqn}
where $V_j$ is the external trapping potential for spin $j$,
$\omega_j$ is the internal energy of spin $j$,
and $\tilde{\Omega}_{jk}$ represents a time-dependent coupling that is used to rotate one spin projection into another.

If we impose an energy cutoff $\ecut$ and only take into account low-energy modes,
the general scattering potential $U_{jk}(\xvec - \xvec^\prime)$ can be replaced by contact potential $U_{jk} \delta(\xvec - \xvec^\prime)$~\cite{Morgan2000}, giving the effective Hamiltonian
\begin{eqn}
\label{eqn:master-eqn:effective-H}
	\hat{H} / \hbar = \int d\xvec \left\{
		\Psiop_j^{\dagger} K_{jk} \Psiop_k
		+ \frac{U_{jk}}{2} \Psiop_j^\dagger \Psiop_k^\dagger \Psiop_j \Psiop_k
	\right\}.
\end{eqn}

For $s$-wave scattering in three dimensions the coefficient is $U_{jk} = 4 \pi \hbar a_{jk} / m$,
where $a_{jk}$ is the scattering length.
Note that in general case the coefficient must be renormalised depending on the grid~\cite{Sinatra2002},
but the change is small if $dx \gg a_{jk}$.


Hereinafter field operators and wave functions will be assumed to be defined in restricted basis, unless explicitly stated otherwise.
The Markovian master equation for the system with the inclusion of losses can be written as~\cite{Jack2002}
\begin{eqn}
\label{eqn:master-eqn:master-eqn}
	\frac{d\hat{\rho}}{dt} =
		- \frac{i}{\hbar} \left[ \hat{H}, \hat{\rho} \right]
		+ \sum_{\lvec} \kappa_{\lvec} \int d\xvec
			\mathcal{L}_{\lvec} \left[ \hat{\rho} \right],
\end{eqn}
where $\lvec = (l_1, l_2, \ldots, l_n)$ is a vector indicating the spins that are coupled,
$n$ being the number of interacting particles,
and we have introduced local Liouville loss terms,
\begin{eqn}
	\mathcal{L}_{\lvec} \left[ \hat{\rho} \right] =
		2\hat{O}_{\lvec} \hat{\rho} \hat{O}_{\lvec}^\dagger
		- \hat{O}_{\lvec}^\dagger \hat{O}_{\lvec} \hat{\rho}
		- \hat{\rho} \hat{O}_{\lvec}^\dagger \hat{O}_{\lvec}.
\end{eqn}
The reservoir coupling operators $\hat{O}_{\lvec}$ are the distinct $n$-fold products of local field annihilation operators,
$\hat{O}_{\lvec} = \hat{O}_{\lvec} (\Psiopvec) =
	\Psiop_{l_{1}} (\xvec)
	\Psiop_{l_{2}} (\xvec) \ldots
	\Psiop_{l_{n}} (\xvec),$
describing local $n$-body collision losses.

The master equation allows us to derive an important property.

\begin{theorem}
    \begin{eqn*}
        \frac{d}{dt} \langle \Psiop_j \rangle
    	= P \left[
    		\langle
    			-\frac{i}{\hbar} \left(
    				K_{jm} \Psiop_m
    				+ U_{jm} \Psiop_m^\dagger \Psiop_m \Psiop_j
    			\right)
    			- \sum_{\lvec} \kappa_{\lvec}
    				\frac{\partial \hat{O}_{\lvec}^\dagger}{\partial \Psiop_j^\dagger} \hat{O}_{\lvec}
    		\rangle
    	\right]
    \end{eqn*}
\end{theorem}
\begin{proof}
Obviously,
\begin{eqn}
    \frac{d}{dt} \langle \Psiop_j \rangle
    = \frac{d}{dt} \Trace{ \hat{\rho} \Psiop_j }
    = \Trace{ \frac{d\hat{\rho}}{dt} \Psiop_j }
\end{eqn}
Substituting the right part of the master equation and applying \lmmref{func-operators:functional-commutators} to simplify the loss term, one gets the statement of the lemma.
\end{proof}
